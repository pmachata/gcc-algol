\documentclass[a4paper,11pt]{article}
\usepackage{a4wide}
\usepackage[latin2]{inputenc}
\usepackage[
  bookmarks=true,
  bookmarksopen=true,
  colorlinks=false,
  urlbordercolor={0 0 0},
  pdfhighlight=/I,
  pdftitle=The\ GCC-Algol\ Demo,
  pdfauthor=Petr\ Machata,
  pdfstartview=FitH,
  bookmarksopenlevel=2
  ]{hyperref}
\pagestyle{empty}
\topmargin 0in
\begin{document}

\def\Algol{{\sc Algol}\space}

% for various computer-related terms
\def\literal#1{{\sffamily{}#1}}
% functions names
\def\function#1{{\sffamily{}#1}}
% file names
\def\file#1{{\sffamily{}#1}}
% command line options
\def\option#1{{\sffamily{}#1}}
% variables, types, and similar
\def\variable#1{{\sffamily{}#1}}

% program listings
\def\program#1{{\sffamily\begin{tabbing}#1\end{tabbing}}}
% keywords in program listings
\def\keyw#1{{\sffamily\bfseries {#1}}}
% comments
\def\comment#1{{\sffamily\it {#1}}}
% strings
\def\litstr#1{{\sffamily\it {#1}}}
% indentation
\def\ind{\hspace{0.5cm}}
% access through pointer (x->y)
\def\ptr{$-\!\!>$}
% GENERIC tree type
\def\xtree{\literal{\keyw{tree}\space}}

\def\note#1{{{\it Note.\space}#1}}
\def\output#1{{\ttfamily\begin{tabbing}#1\end{tabbing}}}

\def\term#1{{\it #1}}


\title{The GCC-Algol Demo}
\author{Petr Machata}
\date{2007-05-21}

\maketitle

\section{Requirements}

The \Algol 60 GCC front end has following build requirements:

\begin{itemize}
\item GCC 4.2.x with resolved dependencies.  GCC 4.1.x or earlier will
  NOT work\footnote{\file{gcc-algol} uses few functions that were only
  introduced in GCC 4.2.x.  Their number is not high, and the
  dependence is certainly for no fundamental reason.  My guess is
  \file{gcc-algol} could be ported to GCC 4.1.x with a bit of
  effort.}.  You will need either complete GCC package
  (e.g. \file{gcc-4.2.0.tar.bz2}), or core package and test suite
  (e.g. \file{gcc-core-4.2.0.tar.bz2} and
  \file{gcc-testsuite-4.2.0.tar.bz2}).
\item Flex 2.5.31 or newer.  Warning: this is a newer version than
  what GCC requires.
\item Bison 2.3 or newer.
\item \file{libga60} wraps around target-native \file{libc} and
  \file{libm}, which have to be available on target site.
\end{itemize}

\subsection{Supported Arches}

As of this writing, \file{gcc-algol} was successfully compiled and
test suite passed on systems with following triplets:

\begin{itemize}
\item i686-pc-linux-gnu
\item ia64-unknwon-linux-gnu
\item mips-sgi-irix6.5
\end{itemize}

This is quite diverse set of arches: little and big endians, 32 and 64
bits, two unrelated operating systems, though no really obscure
system.  This illustrates that the front end is probably written in
enough hardware and OS independent way to be readily portable.

\section{Build Process}

Only a straight build (i.e. no Canadian Crosses) is supported at the
moment. (Or at least tested.  Chances are the compiler will work also
for non-native setups.)  First, prepare a source tree:

\program{\ind\=\+
\$ ls\\
{\em gcc-4.2.0.tar.bz2 gcc-algol60-0.3.tar.bz2}\\
\$ tar xjf gcc-4.2.0.tar.bz2\\
\$ pushd gcc-4.2.0\\
\$ tar xjf ../gcc-algol60-0.3.tar.bz2\\
\$ patch -p1 $<$ gcc-4.2.0-toplevel.patch\\
{\em patching file configure}\\
{\em patching file Makefile.in}\\
{\em patching file Makefile.def}\\
\$ popd}

Then prepare for the build:

\program{\ind\=\+
\$ mkdir build-4.2.0\\
\$ cd build-4.2.0\\
\$ ../gcc-4.2.0/configure \=-{}-prefix=`pwd`/../inst-4.2.0/ $\backslash$\\
                          \>-{}-enable-version-specific-runtime-libs $\backslash$\\
			  \>-{}-enable-languages=c,algol60 $\backslash$\\
			  \>-{}-disable-werror}

The \option{-{}-disable-werror} part is unfortunate, but necessary:
the parser and lexer \file{.c} files, generated from yacc and flex
respectively, currently generate a lot of warnings.

Finally, do a build itself:

\program{\ind\=\+
\$ gmake -j 4\\
{\em ...lots of output...}\\
{\em gmake[1]: Leaving directory `/tmp/build-4.2.0'}}

(The \option{-j} constant determines how many parallel processes make
should launch.  You will want to adjust this value depending on number
of CPUs or cores of your computer.)

You can test GCC as a whole with \literal{gmake check}, or just \Algol
60 front end with \literal{gmake -C gcc check-algol60}.  If you so
wish, you can install the front end with \literal{gmake install}, it
will end up in configured \option{-{}-prefix} directory.

\section{Using \file{gcc-algol}}

From now on, you can use it as an ordinary GCC
command\footnote{Actually, in the case as illustrated, the compiler
would likely end up in a directory outside \literal{PATH} and
\literal{LD\_LIBRARY\_PATH}.  You may need to adjust these
variables.}:

\program{\ind\=\+
\$ which ga60-4.2.0\\
{\em /tmp/inst-4.2.0/bin/ga60-4.2.0}\\
\$ ga60-4.2.0 --version $|$ head -n 1\\
{\em ga60-4.2.0 (GCC) 4.2.0}\\
\$ cat foo.a60\\
{\em 'begin' puts(`Yay, it works!'); 'end';}\\
\$ ga60-4.2.0 foo.a60 -o x\\
\$ ./x\\
{\em Yay, it works!}}

Let's have a look at the compiled binary:

\program{\ind\=\+
\$ readelf -a x $|$ grep NEEDED\\
{\em 0x0000000000000001} \={\em{}(NEEDED)} \={\em{}Shared library: [libga60.so.1]}\\
{\em 0x0000000000000001} \>{\em{}(NEEDED)} \>{\em{}Shared library: [libm.so.6.1]}\\
{\em 0x0000000000000001} \>{\em{}(NEEDED)} \>{\em{}Shared library: [libc.so.6.1]}}

The binary itself depends on \file{libm} and \file{libc}.  These
dependencies are injected into the binary by the linking hook
\function{lang\_specific\_driver}.  If we link the binary with gcc
driver, our hook will not be called, and only \file{libc} will be
linked in:

\program{\ind\=\+
\$ gcc-4.2.0 x.a60 -o x -lga60\\
\$ readelf -a x $|$ grep NEEDED\\
{\em{}0x0000000000000001} \={\em{}(NEEDED)} \={\em{}Shared library: [libga60.so.1]}\\
{\em{}0x0000000000000001} \>{\em{}(NEEDED)} \>{\em{}Shared library: [libc.so.6.1]}}

Under current setup, where all \Algol 60 services are provided by {\em
shared} \file{libga60}, explicit dependence on \file{libm} is
unnecessary, and even wrong---\file{libga60} itself should have these
dependencies (and has!).  However, \file{libga60} doesn't have to be
shared, and in that case \file{libm} has to be brought in explicitly.

\file{gcc-algol} uses a preprocessor, and understands traditional
preprocessing directives.  The preprocessor is not run by default,
only when the file extension is \file{.A60} or \file{.ALG} (i.e. upper
case).  For example, let's have the two files as shown on figures
\ref{Figure++PreprocessX} and \ref{Figure++PreprocessY}.

\begin{figure}
\begin{verbatim}
#if !defined(_ALGOL60_) || !defined(USK)
# error That does not compute.
#else
'begin'
        /* funky C-like comment */
#       include "y.ALG"
'end';
#endif
\end{verbatim}
\caption{Preprocessing: Example file x.ALG}
\label{Figure++PreprocessX}
\end{figure}

\begin{figure}
\begin{verbatim}


'string' a;
\end{verbatim}
\caption{Preprocessing: Example file y.ALG}
\label{Figure++PreprocessY}
\end{figure}

\program{\ind\=\+
\$ ga60-4.2.0 ./x.ALG\\
{\em{}./x.ALG:2: error: \#error That does not compute.}\\
\$ ga60-4.2.0 -DUSK ./x.ALG\\
{\em{}./y.ALG:1: error: type 'string' is invalid in this context.}\\
\$ ga60-4.2.0 -DUSK -M ./x.ALG\\
{\em{}x.o: x.ALG y.ALG}\\
\$ ga60-4.2.0 -DUSK -E ./x.ALG\\
{\em{}\# 1 "./x.ALG"}\\
{\em{}\# 1 "$<$built-in$>$"}\\
{\em ... more preprocessed output ... }}

The errors at the first and second commands are intentional.
The first is here to display that it's possible to pass in your own
defines.  The second shows that \file{gcc-algol} correctly tracks
files and line numbers: even though the erroneous line appears as
twelfth line of preprocessed output, error message is located
properly.

Finally, let's look that the binary supports debugging:

\program{\ind\=\+
\$ cat foo.a60\\
{\em 'begin'}\\
\ind{\em  puts (`ahoy');}\\
\ind{\em exit (9);}\\
{\em 'end'}\\
\$ ga60-4.2.0 foo.a60 -o x -ggdb3 \\
\$ gdb -q ./x \\
(gdb) break main\\
{\em Breakpoint 1 at 0x804846a: file foo.a60, line 2.}\\
(gdb) run\\
{\em Starting program: /tmp/x}\\
{\em Breakpoint 1, main () at foo.a60:2}\\
{\em2}\ind{\em{}puts (`ahoy');}\\
(gdb) cont\\
{\em Continuing.}\\
{\em ahoy}\\
{\em Program exited with code 010.}}

\end{document}

% Local Variables:
% compile-command: "make show-dp"
% End:
