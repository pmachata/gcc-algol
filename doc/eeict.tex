\documentclass{eeict}
\usepackage[latin2]{inputenc}
\usepackage[bf]{caption2}

\title{Construction of GNU Compiler Collection Front End}
\author{Petr Machata}
\programme{Master Degree Programme (2), FIT BUT}
\emails{xmacha31@stud.fit.vutbr.cz}

\supervisor{Milo� Eysselt, Luk� Szemla}
\emailv{eysselt@fit.vutbr.cz, lukas.szemla@siemens.com}

\begin{document}
% -- Hlavi�ka pr�ce --

\maketitle

\def\term#1{{\it #1}}
\def\option#1{{\sffamily{}#1}}
\def\literal#1{{\sffamily{}#1}}
\def\file#1{{\sffamily{}#1}}

\selectlanguage{english}

\abstract{ The entry barrier to the GCC development got considerably
  lower during the last years.  With work on documentation and
  separation of internal modules, writing for GCC becomes accessible
  to wide community of industry and academia developers.

  This paper provides an overview of GCC high-level work-flow, with
  emphasis on information necessary for front end developer. }

\section{Introduction}

GCC can translate from variety of source languages into variety of
assemblers.  With one command, decorated with various flags, it is
able to do preprocessing, compilation, assembly and linking.  How does
it do this?

In following sections, we delve into successively deeper levels of
overall GCC architecture.

\section{Compilation Driver and Compiler Proper}

On the outermost level, GCC is divided into a \term{compilation
  driver}, and a \term{compiler proper}.  Besides this, GCC uses
assembler and a linker from the host tool chain.

Compilation driver is a user-interfacing application.  It knows about
all languages that GCC supports.  Given a source file, it can guess
what to do based on suffix of the file.  After optional preprocessing,
it launches compiler proper, then passes its output to assembler, and,
eventually, linker.  Depending on command line switches in effect the
process can be cut short in any of the stages.

There is one compiler proper for each language, each in own
executable.  You can run the compiler by hand, if you so wish, it
accepts much the same command line arguments as \file{gcc} command,
and turns the source file into assembly.

Let us now look at the compiler proper closer in next section.

\section{Front End, Middle End, and Back End}

The compiler proper itself is composed from three components: a
\term{front end}, a \term{back end}, and a \term{middle end}.  Front
end contains language-processing logic, and together with middle end
it makes up platform independent part of the compiler.  Back end is
then the platform dependent part.

Just like the compilation process done by the driver, the compilation
of a source file can be viewed as a pipeline that converts one program
representation into another.  Source code enters the front end and
flows through the pipeline, being converted at each stage into
successively lower-level representation forms until final code
generation in the form of assembly code \cite{LDS:2006:Novillo}.

There are two intermediate languages that are used on the interfaces
between the three ``ends'' of GCC.  The higher level one, used between
front end and middle end, is called \term{GENERIC}.  The lower level
one, used between middle end and back end, is called RTL, or Register
Transfer Language.  Both middle end and back end do various
optimizations on their intermediate representation before they turn it
into yet lower level one.

Both interfaces mentioned are {\em unidirectional}: front end feeds
GENERIC into middle end, middle end feeds RTL into back end.  But
sometimes the other direction is also necessary.  For example, during
alias analysis, middle end has to know whether two objects of
different data types may occupy the same memory location
\cite{LDS:2006:Novillo}.  Each language has its own rules for that,
and front end is the place where language-dependent things happen.
For this purpose, GCC has a mechanism of \term{language hooks} or
\term{langhooks}, which provide a way to involve front end in lower
layers of compilation process.

The goal of front end is to analyze source program, and ensure all
types are correct and all constraints required by the language
definition hold.  If everything is sound, it has to provide GENERIC
representation of the program.  You need to know GENERIC to write a
front end, but you do not have to know anything about RTL.

\section{GENERIC and GIMPLE}

The important intermediate form is called GENERIC.  From
expressiveness point of view, it is similar to C.  From notation point
of view, it is similar to Lisp.  GENERIC is capable of representing
whole functions, i.e. it supports everything there is to represent in
a typical C function: variables, loops, conditionals, function calls,
etc.

GENERIC is a tree language (hence the Lisp qualities).  As any well
behaving tree, it is recursive in nature, having both internal and
leaf nodes, with internal nodes capable of holding other internal
nodes.  Typical leaves are identifier references, integer numbers,
etc.  Internal nodes are then unary or binary operations, block
containers, etc.

For optimization purposes, GENERIC is still too high level a
representation.  During a course of compilation, it is lowered.  The
intermediate code that it is lowered into is called \term{GIMPLE}.
The process of lowering is thus inevitably called
\term{gimplification}.  GIMPLE is a subset of GENERIC.  Nesting
structures are still represented as containers in GIMPLE, but all
expressions are broken down to three address code, using temporaries
to store intermediate results\cite{GDS:2003:Merill}.  There are
actually two GIMPLE forms: high GIMPLE and low GIMPLE.  In low GIMPLE
containers are further transformed into \literal{goto}s and
labels\cite{LDS:2006:Novillo}.

Apart from predefined GENERIC nodes, GCC provides a mechanism to
define your own node types.  You have to provide a langhook for the
purpose of gimplifying these.  For example C++ front end actually does
not use pure GENERIC, but extends it with its own node types.

\section{Fronted ASTs}

While it is possible to use GENERIC for representation of programs in
your front end, it is recommended not to do so \cite{LDS:2006:Novillo}
\cite{LJ:2005:Tromey}.  Your own AST representation can suit the
language in hand better, and furthermore you are better shielded from
the changes in GCC core.  Besides, the language analysis tools that
you write are then shielded from {\em GCC itself}, which makes them
reusable in other tasks: e.g. as a syntax checker in smart editor.

This approach was taken by the Java front end, and also the
experimental front end of mine, which compiles Algol.

\section{Garbage Collector}

Internally, GCC uses garbage collector \cite{TR:GCCInt} for its memory
management.  The objects with indeterminable lifetime, which includes
trees, are not managed explicitly, but instead garbage-collected.
The collector used is of mark \& sweep kind.  Pointers (variables,
fields, ...) that should be collected are explicitly tagged, the tags
are gathered during the build, and marking and scanning routines are
generated.

The garbage collector data are also used for implementation of
precompiled headers.  The precompiled header mechanism can only save
static variables if they are scalar. Complex data structures must be
allocated in garbage-collected memory.

\section{Summary}

In my thesis\cite{2006:Machata}, I am describing the methodology of
custom front end integration.  The work is half exploratory, and half
synthesizing in nature: there is some documentation and several
papers, but in the end, few people know how to write for GCC.  The
thesis will provide useful how-to for anyone who wishes to write their
own GCC front end.


\bibliographystyle{plain}
\bibliography{eeict}


\end{document}

% Local Variables:
% compile-command: "make show-eeict"
% End:
