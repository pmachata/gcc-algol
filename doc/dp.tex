\documentclass[a4paper,11pt]{report}
\usepackage{a4wide}
\usepackage[latin2]{inputenc}
\usepackage[
  bookmarks=true,
  bookmarksopen=true,
  colorlinks=false,
  urlbordercolor={0 0 0},
  pdfhighlight=/I,
  pdftitle=,
  pdfauthor=Petr\ Machata,
  pdfstartview=FitH,
  bookmarksopenlevel=2
  ]{hyperref}
\usepackage{subscript}
\pagestyle{empty}
\topmargin 0in
\begin{document}

This is a series of notes regarding what I'm going to want to have
there included.

 + Introduction: the purpose of work, goals, how to read, how is it
 written.  Write at the very end of work.

 + GCC the platform: How is GCC organized internally, how the
 compilation process works, what is required on side of future
 frontend implementor.  What is GENERIC and how is it used.  Concrete
 stuff: the way GCC works, and how to use it when you want to gccify
 your parser.  I don't think it makes a sense to include whole GENERIC
 reference here.  Provide few examples and leave the rest to next
 chapter.  Debugging -- how to debug GCC itself.  Dejagnu testing
 suite.  Attributes -- how do frontend handle attributes, how do they
 define their own.  -std=XXX, -pedantic -- how to support various
 language versions.

 + ALGOL 60: the language.  Why ALGOL 60.  High level overview of how
 the language looks like and what's inside.  Pick interesting points,
 if they are interesting from language implementor's POV.  Show
 extensions to the language (IO, structures, separate compilation).
 Provide UML model of ALGOL representation.  Methods employed to
 implement lexical and syntactic analysis. (Just a short note.  Most
 probably flex and yacc, but maybe we would rather stick with
 recursive descend.)

 + ALGOL 60 over GCC: Case study.  Suppose we have the parser library
 ready and waiting.  How to plug it into GCC?  Let's see... step by
 step from creation of apropriate directory, to setting up configures
 (don't forget about the optional keyword distinguisher (probably only
 case-sensitivity flag)), to initialization and to translation of
 ALGOL 60 representation data structures to GENERIC.  How to plug
 testsuite to gcc.

 + Conclusion: Why this work simply rocks.  What has been done, what
 could have been done better, etc.

 + Appendix A: steps to install ALGOL 60 compiler on end user system.

 + Appendix B: source codes.

 + Appendix C: ALGOL 60 spec (don't attach to work? maybe just use it
 as a reference)

\end{document}
