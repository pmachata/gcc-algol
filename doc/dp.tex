\documentclass[a4paper,11pt]{report}
\usepackage{a4wide}
\usepackage[latin2]{inputenc}
\usepackage[
  bookmarks=true,
  bookmarksopen=true,
  colorlinks=false,
  urlbordercolor={0 0 0},
  pdfhighlight=/I,
  pdftitle=,
  pdfauthor=Petr\ Machata,
  pdfstartview=FitH,
  bookmarksopenlevel=2
  ]{hyperref}
\usepackage{subscript}
\pagestyle{empty}
\topmargin 0in
\begin{document}

\title{Construction of GNU Compiler Collection Frontend}
\author{Petr Machata\\
  \texttt{xmacha31@stud.fit.vutbr.cz}}
\date{2006-09-08}
\maketitle

\def\mychapter#1{{\Huge \bf \vspace{1em}#1}\vspace{1.0em}}
\def\mysection#1{{\Large \bf \vspace{1em}#1}\vspace{1.0em}}
\def\mysubsection#1{{\large \bf \vspace{1em}#1}\vspace{1.0em}}

\def\Algol{{\sc Algol}\space}
\def\GCC{{\sc GCC}\space}
\def\C{{\sc C}\space}

% for various computer-related terms
\def\literal#1{{\sffamily{}#1}}
% functions names
\def\function#1{{\sffamily{}#1}}
% file names
\def\file#1{{\sffamily{}#1}}
% commandline options
\def\option#1{{\sffamily{}#1}}
% variables, types, and similar
\def\variable#1{{\sffamily{}#1}}

% program listings
\def\program#1{{\sffamily\begin{tabbing}#1\end{tabbing}}}
% keywords in program listings
\def\keyw#1{{\sffamily\bfseries {#1}}}
% indentation
\def\ind{\hspace{0.5cm}}

\def\note#1{{{\it Note.\space}#1}}
\def\output#1{{\ttfamily\begin{tabbing}#1\end{tabbing}}}
\def\program#1{{\tt\begin{tabbing}#1\end{tabbing}}}

% ====================================================================

\chapter*{Prohl�en�}

Prohla�uji �e jsem tuto bakal��skou pr�ci vypracoval s�m pod odborn�m
veden�m Ing. Luk�e Szemly.  Ve�ker� prameny, z nich� jsem p�i pr�ci
�erpal, jsou ��dn� citov�ny, a v p��loze uvedeny.

% ====================================================================

\chapter*{Acknowledgements}

Thanks to... (FIT BUT that they gave up their rights)

% ====================================================================

\chapter*{Abstrakt}

\mysection{Kl��ov� slova}

% ====================================================================

\chapter*{Abstract}

\mysection{Key words}

% ====================================================================

\tableofcontents

% ====================================================================

\chapter{Introduction}

Describe what this thesis is about, summarize chapters, provide
insight to how to read it.

\chapter{Why GCC, Why Now}

(Reference: Tom Tromey's Java paper) Summarize options available when
one writes new language.  Interpreters and compilers, hand-rolling
compiler, compiling via C (this sucks mostly because C compiler has to
recover the higher level model from C, which of course isn't possible;
and debugging--GCC will preserve symbol names where necessary, while
in C, you'd have to mangle them without chance to give a clue to
target C compiler (you want to emit \#line directives to point the user
to the place where the error originated, and in debugger you have no
way to hide the fact that you are going through C)), compiling via GCC
(what if it's dynamic language, what if static).

Personal opinions aside, writing GCC frontend is probably best choice.
GCC is ported to dozens of platforms, has tons of optimizations, has
necessary community, corporation and academia drive, and finally
writing such frontend isn't nearly as difficult as rolling your own
backend (fingers crossed it's true).

Describe briefly that the author has written \Algol 60 parser to get
himself familiar with the stuff he describes.  Maybe drop a few notes
about how the parser (independent from GCC) and GCC frontend got
joined.

\chapter{GCC Architecture}

How does GCC work.  What's driver, what's the frontend proper.  Bird's
eye view of compilation passes, internal structure, references to
other documents with more detailed description.  Main source will be
Diego Novillo's OLS paper.  Mention garbage collector (only mention).
Also
mention hooks, but don't go into depth (go as far as Tom Tromey's
Linux Journal article).

Mention that there are two approaches: targeting GENERIC, and using
GENERIC as AST rep.  This work aims at the former scenario.  Under
this setup, there are four levels actually: your parser, GCC frontend,
GCC middleend, GCC backend.  The frontend of the parser is completely
independent of GCC, backend of the parser is frontend of the GCC.
With the latter approach, one defines special tree codes.

\chapter[Hello World]{Hello, World!}

What's Make-lang.in, config-lang.in, lang-specs.h, algol60.texi.

Compilation process: why language subdirectory, what happens with
Make-lang.in and config-lang.in, what's bootstrapping.  Defining
hooks, defining options.
Note: recompilation after the change of lang.opt takes forever.
Note 2: correction: it takes *aeons*.
On my box:
real	162m16.769s
user	116m27.430s
sys	2m49.900s
i.e. almost three hours





\section{Other section}

This work includes frontend template.  That's the smallest possible
frontend, that even doesn't emit any code, just writes out some stuff
(but it has all inits in place and ready for real thing).  Go through
the files and code in that template, and briefly describe hooks and
interesting things.  Leave the full catalogues to subsequent chapters.

About integration.  Mention slow round trip time: edit->compile->test
loop tends to be quite lengthy with GCC; you really want fast machine.
You can use IN\_GCC to special case your frontend to use GCC services
where available.

GCC uses garbage collector.  Describe GCC poisoning.  Probably go
through some GC codes, if apropriate.

%=====================================================================%

\chapter{Targeting GCC}

This chapter should describe how to generate GCC intermediate language
called GENERIC.  Also use hooks if possible.

\section{Symbol Handling}

Symbol handling: probably has to be frontend specific, although GCC
can store declarations.  See how other frontends do their thing.  Name
mangling, C and C++-compatible interfacing (extern "C").

\section{Variables and Types}

How to do typechecking, representing various types (including arrays,
functions, structures).  About variables: function-static, automatic,
file-static and global.  Notes on initialization of misc variable
kinds/types (e.g. how does initialization of file-static struct differ
from initialization of automatic array).

Arrays: variable sized arrays, multidimensional arrays.

\section{Expressions}

Variable reference, constants, various expression kinds.

\section{Structured Programming}

Nested blocks; for, while, do; if, switch; goto and labels.

\section{Functions}

Declarations/definitions, function calls, global, local (nested),
file-static, overloaded, builtin, anonymous. Pass by value, reference
and name, varargs.

\section{Modules}

Variable/function visibility, module ctors and dtors.  Namespaces.

\section{Object-oriented programming}

Objects, dynamic typing (look at OBJ\_TYPE\_REF), virtual functions.

\section{Numbers}

Elaborate on numeric support in \GCC.  Including big integers, long
floats, what's necessary big/little endian-wise, if there's anything
to know about bits-per-byte, complex numbers (and what are possible
types of complex number components).  Floating point, integer and
string literals.

\section{Extending GENERIC}

What are custom GENERIC codes, how to declare them, how to handle
them, what's gimplification, how to gimplify them.

%=====================================================================%

\chapter{GCC Services}

\section{Commandline options}

Each \GCC frontend has the capability of processing commandline
options.  Moreover it inherits all the options from \GCC proper, so
e.g. \option{-O3}, \option{-o file} and others are available for all
frontends with no work.  The only work is necessary for definition of
options peculiar to given frontend, and even there is the tedium of
commandline parsing left off programmer's sight.

\GCC will support both positive and negative variant of \option{-f},
\option{-W} and \option{-m} options.  E.g. when your frontend supports
\option{-fdump-ast}, \GCC will understand also \option{-fno-dump-ast}.
Furthermore, each option can be parametrized.  Thus you can support
e.g. \option{-{}-output-pch=} option for output of precompiled headers,
and the part after ``='' is delivered to option handler as an
argument.

Of course, programmer has to write the handler for frontend-specific
options herself.  All work takes place in
\function{LANG\_HOOKS\_HANDLE\_OPTION} hook, \GCC calls this function
each time it hits an option that the frontend understands.  The
communication isn't done through option strings, though.  Instead, \GCC
associates each option to a symbolic identifier with unique integer
value.  When option is handled, simple \literal{switch} statement can
be used to decide what should be done.  Option strings are transformed
to identifiers in a straightforward manner: each non-alphanumeric
character in a string is replaced with underscore, and \literal{OPT}
is prepended before the resulting string.  Thus
e.g. \option{-{}-output-pch=} is referred-to by identifier
\variable{OPT\_\_output\_pch\_}.

Other arguments in option-handling hook are \variable{argument} and
\variable{value}.  Variable \variable{argument} is either
\variable{NULL}, or it holds a string with the extra option argument
(as was described few paragraphs up).  The variable \variable{value}
is 1 if an option is used in its positive variant, and 0 for
\option{no-} variant.

All frontend-specific options are defined in \file{lang.opt}.  This
file gives, through build magic, rise to \file{options.h}.  Format and
features of \file{lang.opt} are to be found in gcc internals
documentation
\footnote{\url{http://gcc.gnu.org/onlinedocs/gccint/Options.html}}.

\section{Cross-compilation Issues}

Elaborate on build-host-target trichotomy.  Assure ga60 is
cross-compilation capable and describe what problems there were.

\section{Runtime Libraries}

How to add runtime for your language.  How to use/avoid glibc.

\section{Debuginfo}

How to add debuginfos to binaries that your frontend produces.  How to
use gdb for your language.

\section{Attributes}

Employing attributes.  How does frontend handle attributes, how does
it define its own.

\section{Preprocessing}

Employing libcpp.  How to use preprocessor for your language.

\section{Inline Assembly}

Employing assembly.  How to add GCC-compatible inline assembly into
your language.

\section{Diagnostics}

Employing GCC diagnostic messages.  Btw, you really want to do that,
otherwise your buggy programs will be passed and a binary generated,
or GCC crashed, depending on how is your error handling done.
Describe the `pedantic' variable.  Describe pedwarn, warning, error
and friends.  report\_diagnostic, diagnostic\_info and friends.
Report code excerpts with verbatim.  It seems that GCC uses it's own
printing functions, with their own formatting.  See e.g. comment
before pp\_base\_format.  Describe that.

\section{Testsuite}

What's dejagnu testsuite, how to write tests, how to integrate them
into GCC testsuite.  You need acurate error reporting to use the
dejagnu GCC framework.  In fortran, they are doing transformations of
error messages to format expected by the framework, describe that.

%=====================================================================%

\chapter{Conclusions}

Conclusions: ``short, concise statements of the inferences that you
have made because of your work. It helps to organize these as short
numbered paragraphs, ordered from most to least important.''

Summary of Contributions: ``The Summary of Contributions will be much
sought and carefully read by the examiners.  Here you list the
contributions of new knowledge that your thesis makes. Of course, the
thesis itself must substantiate any claims made here. There is often
some overlap with the Conclusions, but that's okay. Concise numbered
paragraphs are again best. Organize from most to least important.''

Future Research: How feasible would be to have a frontend, that just
hands all the calls to dynamically loaded external parser?  This would
allow for out-of-tree builds, wouldn't it?

\chapter*{Appendix: Frontend-to-Middleend Interface}

Describe all data structures that frontend touches.
http://gcc.gnu.org/projects/documentation.html, project "Fully
document the interface of front ends to GCC"

\chapter*{Appendix: \Algol 60 Compiler}

Installation, supported GCCs, usual stuff.



\chapter{Notes}

Functions need a return statements.  Look how it's done for noreturn
functions.

COND\_EXPR breaks when other value than 0 or 1 is used as condition.


\end{document}
