\documentclass{beamer}
\usetheme{default}

\usepackage[english]{babel}
\usepackage[latin1]{inputenc}
\title[Construction of GNU Compiler Collection Front End]{Construction of GNU Compiler Collection\\Front End}

\author[Petr Machata]{Petr Machata \\ \texttt{xmacha31@stud.fit.vutbr.cz}}
\date{2007-01-03}

\def\Algol{{\sc Algol}\space}

\begin{document}

\frame{\titlepage}

\frame
{
  \frametitle{About My Project}

  \begin{itemize}
  \item The goal: describe how to write GCC front ends
  \item \Algol 60 for GCC
  \item General usefulness: English, Open Source
  \item Exploit GCC: inline assembly, attributes, ...
  \end{itemize}
}

\frame
{
  \frametitle{GCC Architecture}

  \begin{itemize}
  \item Driver vs. Proper; lang-specific driver
  \item Front end, Middle end, Back end; hooks
  \item GENERIC, GIMPLE; C expressiveness; AST
  \item Runtime library, testsuite, preprocessor
  \item Garbage collector
  \end{itemize}
}

\frame
{
  \frametitle{Summay of a Work}

  \begin{itemize}
  \item {\tt ga60} compiles simple programs
  \item Implement functions
  \item Cover bootstrapping; both library and frontend
  \item Cover crosscompilation
  \end{itemize}
}

\end{document}

% Local Variables:
% compile-command: "make show-demo-1"
% End:
