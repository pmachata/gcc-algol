\documentclass[a4paper,11pt]{article}
\usepackage{a4wide}
\usepackage[latin2]{inputenc}
\usepackage[
  bookmarks=true,
  bookmarksopen=true,
  colorlinks=false,
  urlbordercolor={0 0 0},
  pdfhighlight=/I,
  pdftitle=Revised\ Report\ on\ the\ Algorithmic\ Language\ Algol\ 60,
  pdfauthor=Petr\ Machata,
  pdfstartview=FitH,
  bookmarksopenlevel=2
  ]{hyperref}
\pagestyle{empty}
\topmargin 0in
\begin{document}
\setcounter{secnumdepth}{5}

\title{Revised Report on the Algorithmic Language Algol 60}

\author{J.W. Backus, F.L. Bauer, J.Green, C. Katz, J. McCarthy\\
   P. Naur, A.J. Perlis, H. Rutishauser, K. Samuelson, B. Vauquois\\
   J.H. Wegstein, A. van Wijngaarden, M. Woodger; \\
   Edited by Peter Naur\\
   Dedicated to the memory of William Turanski}

\footnotetext{This copy of the ``Revised Report on the Algorithmic
Language Algol 60'' was typed to texi source by Erik Schoenfelder at
October 1990, and retyped to \LaTeX \space by Petr Machata at February
2006.}

\date{January 1963}

\maketitle

\def\keyw#1{{\tt \bf \underline{#1}}}
\def\blankchar{\leavevmode\hbox{\tt\char`\ }}

\def\mleqv{$\equiv$}
\def\mlnot{$\neg$}
\def\mland{$\wedge$}
\def\mlor{$\vee$}
\def\mlimp{$\Rightarrow$}
\def\mlt{$<$}
\def\mle{$\leq$}
\def\mgt{$>$}
\def\mge{$\geq$}
\def\meq{$=$}
\def\mne{$\neq$}
\def\mpow{$\uparrow$}
\def\mtim{$\times$}
\def\mdiv{$\div$}
\def\ten{$_{10}$}

\def\mychapter#1{{\Huge \bf \vspace{1em}#1}\vspace{1.0em}}
\def\mysection#1{{\Large \bf \vspace{1em}#1}\vspace{1.0em}}
\def\mysubsection#1{{\large \bf \vspace{1em}#1}\vspace{1.0em}}



\mychapter{Summary}

The report gives a complete defining description of the international
algorithmic language Algol 60. This is a language suitable for
expressing a large class of numerical processes in a form sufficiently
concise for direct automatic translation into the language of
programmed automatic computers.

The introduction contains an account of the preparatory work leading
up to the final conference, where the language was defined. In
addition the notions reference language, publication language, and
hardware representations are explained.

In the first chapter a survey of the basic constituents and features
of the language is given, and the formal notation, by which the
syntactic structure is defined, is explained.

The second chapter lists all the basic symbols, and the syntactic
units known as identifiers, numbers, and strings are defined. Further
some important notions such as quantity and value are defined.

The third chapter explains the rules for forming expressions and the
meaning of these expressions. Three different types of expressions
exist: arithmetic, Boolean (logical), and designational.

The fourth chapter describes the operational units of the language,
known as statements. The basic statements are: assignment statements
(evaluation of a formula), go to statements (explicit break of the
sequence of execution of statements), dummy statements, and procedure
statements (call for execution of a closed process, defined by a
procedure declaration). The formation of more complex structures,
having statement character, is explained. These include: conditional
statements, for statements, compound statements, and blocks.

In the fifth chapter the units known as declarations, serving for
defining permanent properties of the units entering into a process
described in the language, are defined.

The report ends with two detailed examples of the use of the language
and an alphabetic index of definitions.


\mychapter{Introduction}


\mysection{Background}


After the publication \footnote{Preliminary report - International
Algebraic Language, Comm.  Assoc. Comp. Mach. 1, No. 12 (1958), 8.}
\footnote{Report on the Algorithmic Language Algol by the ACM
Committee on Programming Languages and the GAMM Committee on
Programming, edited by A. J. Perlis and K. Samuelson, Numerische
Mathematik Bd. 1, S. 41-60 (1959).} of a preliminary report on the
algorithmic language Algol, as prepares at the conference in Zuerich
in 1958, much interest in the Algol language developed.


As a result of an informal meeting held at Mainz in November 1958, about
forty interested persons from several European countries held an Algol
implementation conference in Copenhagen in February 1959.  A ``hardware
group'' was formed for working cooperatively right down to the level of
the paper tape code.  This conference also led to the publication by
Regnecentralen, Copenhagen, of an `Algol Bulletin', edited by Peter
Naur, which served as a forum for further discussion.  During the June
1959 ICIP Conference in Paris several meetings, both formal and informal
ones, were held.  These meetings revealed some misunderstandings as to
the intent of the group which was primarily responsible for the
formulation of the language, but at the same time made it clear that
there exists a wide appreciation of the effort involved.  As a result of
the discussions it was decided to hold an international meeting in
January 1959 for improving the Algol language and preparing a final
report.  At a European Algol Conference in Paris in November 1959 which
was attended by about fifty people, seven European representatives were
selected at attend the January 1960 Conference, and they represent the
following organisations: Association Fran\c caise de Calcul, British
Computer Society, Gesellschaft f\"ur Angewandte Mathematik und
Mechanik, and the Nederlands Rekenmachine Genootschap.  The seven
representatives held a final preparatory meeting at Mainz in December
1959.

Meanwhile, in the United States, anyone who wished to suggest changes or
corrections to Algol was requested to send his comments to the
`Communications of the ACM', where they were published.  These comments
then became the basis of consideration for changes in the Algol
language.  Both the SHARE and USE organisations established Algol
working groups, and both organisations were represented on the ACM
Committee on Programming Languages.  The ACM Committee met in Washington
in November 1959 and considered all comments on Algol that had been sent
to the ACM `Communications'.  Also, seven representatives were selected
to attend the January 1960 international conference.  The seven
representatives held a final preparatory meeting in Boston in December
1959. 



\mysection{January 1960 Conference}


The thirteen representatives \footnote{William Turanski of the
American group was killed by an automobile just prior to the January
1960 Conference.}, from Denmark, England, France, Germany, Holland,
Switzerland, and the United States, conferred in Paris from January 11
to 16, 1960.

Prior to this meeting a completely new draft report was worked out from
the preliminary report and the recommendations of the preparatory
meetings by Peter Naur and the Conference adopted this new form as the
basis for its report. The Conference then proceeded to work for
agreement on each item of the report. The present report represents the
union of the Committee's concepts and the intersection of its agreements.



\mysection{April 1962 Conference [Edited by M. Woodger]}

A meeting of some of the authors of Algol 60 was held on 2nd - 3rd April
in Rome, Italy, through the facilities and courtesy of the International
Computation Centre.  The following were present:

\vspace{1em}
\begin{tabular*}{15.0cm}{lll}
Authors             & Advisers          & Observer \\
F. L. Bauer         & M. Paul           & W. L. van der Poel\\
J. Green            & R. Franciotti     & (Chairman, IFIP TC 2.1 \\
C. Katz             & P. Z. Ingerman    & Working Group Algol) \\
R. Kogon\\(representing J.W. Backus)   &&\\
P. Naur                                &&\\
K. Samuelson        & G. Seegem\"uller  &\\
J. H. Wegstein      & R.E. Utman        &\\
A. van Wijngaarden                     &&\\
M. Woodger          & P. Landin         &\\
\end{tabular*}
\vspace{1em}

The purpose of the meeting was to correct known errors in, attempt to
eliminate apparent ambiguities in, and otherwise clarify the Algol 60
Report.  Extensions to the language were not considered at the meeting. 
Various proposals for correction and clarification that were submitted
by interested parties in response to the Questionnaire in Algol Bulletin
No. 14 were used as a guide. 

This report constitutes a supplement to the Algol 60 Report which should
resolve a number of difficulties therein.  Not all of the questions
raised concerning the original report could be resolved.  Rather than
risk of hastily drawn conclusions on a number of subtle points, which
might create new ambiguities, the committee decided to report only those
points which they unanimously felt could be stated in clear and
unambiguous fashion. 

Questions concerned with the following areas left for further consideration by
Working Group 2.1 of IFIP, in the expectation that current
work on advanced programming languages will lead to
better resolution:

\begin{enumerate}

\item Side effects of functions.

\item The call by name concept.

\item \keyw{own}: static or dynamic.

\item For statement: static or dynamic.

\item Conflict between specification and declaration.

\end{enumerate}

The authors of the Algol 60 Report present at the Rome Conference,
being aware of the formation of a Working Group on Algol by IFIP,
accepted that any collective responsibility which they might have with respect to the development, specification, and refinement of the Algol language
will from now on be transferred to that body.

This report has been reviewed by IFIP TC 2 on Programming Languages in August 1962 and
has been approved by the Council of the International Federation for
Information Processing.

As with the preliminary Algol report, three different levels of language
are recognized, namely a Reference Language, a Publication Language, and
several Hardware Representations. 


\mysubsection{Reference Language}

\begin{enumerate}

\item It is the working language of the committee.

\item It is the defining language.

\item The characters are determined by ease of mutual understanding and
      not by any computer limitations, coders notation, or pure
      mathematical notation.

\item It is the basic reference and guide for compiler builders.

\item It is the guide for all hardware representations.

\item It is the guide for transliterating from publication language to
      any locally appropriate hardware representations.

\item The main publications of the Algol language itself will use the
      reference representation.
\end{enumerate}


\mysubsection{Publication Language}

\begin{enumerate}
\item The publication language admits variations of the reference
      language according to usage of printing and handwriting (e.g.
      subscripts, spaces, exponents, Greek letters).

\item It is used for stating and communicating process.

\item The characters used may be different in different countries, but
      univocal correspondence with reference representation must be
      secured.
\end{enumerate}


\mysubsection{Hardware Representations}

\begin{enumerate}
\item Each of these is a condensation of the reference language enforced
      by the limited number of characters on the standard input
      equipment.

\item Each one of these uses the character set of a particular computer
      and is the language accepted by a translator for that computer.

\item Each of these must by accompanied by a special set of rules for
      transliterating from publication or reference language.
\end{enumerate}

For transliteration between the reference language and a language
suitable for publications, among others, the following rules are
recommended.


\begin{tabular*}{15.0cm}{lp{10.0cm}}
\\
Reference Language            & Publication Language \\ \\


Subscript brackets [ ]        & Lowering of the line between the
                                brackets and removal of the brackets. \\

Exponentiation $\uparrow$     & Raising the exponent. \\

Parentheses ()                & Any form of parentheses, brackets,
                                braces. \\

Basis of ten $_{10}$           &      Raising of the ten and of the following
                                integral number, inserting of the
                                intended multiplication sign. \\
\end{tabular*}



\mychapter{Description of the reference language}

\hspace*{7cm} Was sich \"u{}berhaupt sagen l\"a{}\ss{}t, l\"a{}\ss{}t sich\\
\hspace*{7cm} klar sagen; und wovon man nicht reden\\
\hspace*{7cm} kann, dar\"u{}ber mu\ss{} man schweigen.\\
\hspace*{9cm} \textsc{Ludwig Wittgenstein}

\section{Structure of the language}
\label{LblStructureOfTheLanguage}


As stated in the introduction, the algorithmic language has three
different kinds of representations -- reference, hardware, and
publication -- and the development described in the sequel is in terms
of the language are represented by a given set of symbols -- and it is
only in the choice of symbols that the other two representations may
differ. Structure and content must be the same for all
representations.

The purpose of the algorithmic language is to describe computational
processes. The basic concept used for the description of calculating
rules is the well known arithmetic expression containing as constituents
numbers, variables, and functions. From such expressions are compounded,
by applying rules of arithmetic composition, self-contained units of the
language -- explicit formulae -- called assignment statements.

To show the flow of computational processes, certain non-arithmetic
statements and statement clauses are added which may describe e.g.,
alternatives, or iterative repetitions of computing statements. Since it
is necessary for the function of the statements that one statement
refers to another, statements may be provided with labels. A sequence of
statements may be enclosed between the statement brackets \keyw{begin}
and \keyw{end} to form a compound statement.

Statements are supported by declarations which are not themselves
computing instructions, but inform the translator of the existence and
certain properties of objects appearing in statements, such as the class
of numbers taken on as values by a variable, the dimension of an array
of numbers, or even the set of rules defining a function. A sequence of
declarations followed by a sequence of statements and enclosed between
\keyw{begin} and \keyw{end} constitutes a block. Every declaration
appears in a block in this way and is valid only for that block.

A program is a block or compound statement which is not contained within
another statement and which makes no use of other statements not
contained within it.

In the sequel the syntax and semantics of the language will be given
\footnote{Whenever the precision of arithmetic is stated as being in
general not specified, or the outcome of a certain process is left
undefined or said to be undefined, this is to be interpreted in the
sense that a program only fully defines a computational process if the
accompanying information specifies the precision assumed, the kind of
arithmetic assumed, and the course of action to be taken in all such
cases as may occur during the execution of the computation.}.



\subsection{Formalism for syntactic description.}
\label{FormalismForSyntacticDescription}


The syntax will be described with the aid of metalinguistic formulae
\footnote{Cf. J. W. Backus, The syntax and semantics of the proposed
    international algebraic language of the Zuerich ACM-GRAMM
    conference. ICIP Paris, June 1959.}.

Their interpretation is best explained by an example:

\begin{flushleft}
\vspace{0.2em}\texttt{<ab> ::= $($ | $[$ | <ab> $($ | <ab> <d>}\\
\end{flushleft}

Sequences of characters enclosed in the bracket \texttt{<>} represent
metalinguistic variables whose values are sequences of symbols. The
marks \texttt{::=} and \texttt{|} (the latter with the meaning of
\b{or}) are metalinguistic connectives. Any mark in a formula, which
is not a variable or a connective, denotes itself (or the class of
marks which are similar to it). Juxtaposition of marks and/or
variables in a formula signifies juxtaposition of the sequences
denoted. Thus the formula above gives a recursive rule for the
formation of values of the variable \texttt{<ab>}. It indicates that
\texttt{<ab>} may have the value \texttt{(} or \texttt{[} or that
given some legitimate value of \texttt{<ab>}, another may be formed by
following it with the character \texttt{(} or by following it with
some value of the variable \texttt{<d>}. If the values of \texttt{<d>}
are the decimal digits, some values of \texttt{<ab>} are:

\begin{verbatim}
[(((1(37(
(12345(
(((
[86
\end{verbatim}


In order to facilitate the study, the symbols used for distinguishing
the metalinguistic variables (i.e. the sequence of characters
appearing within the brackets \texttt{<>} as \texttt{ab} in the above
example) have been chosen to be words describing approximately the
nature of the corresponding variable. Where words which have appeared
in this manner are used elsewhere in the text they will refer to the
corresponding syntactic definition. In addition some formulae have
been given in more than one place.

    Definition:
\begin{flushleft}
\vspace{0.2em}\texttt{<empty> ::=}\\
\end{flushleft}

(i.e. the null string of symbols).



\section[Basic symbols, identifiers, numbers, and strings]{Basic symbols, identifiers, numbers, and strings.\\
  Basic concepts}
\label{LblBasicSymbols}



The reference language is built up from the following basic symbols:

\begin{flushleft}
\vspace{0.2em}\texttt{<basic symbol> ::= <letter> | <digit> | <logical value> | <delimiter>}\\
\end{flushleft}



\subsection{Letters}
\label{LblBasicSymbolsLetters}


\begin{flushleft}
\vspace{0.2em}\texttt{<letter> ::= a | b | c | d | e | f | g | h | i | j | k | l |\\
  \hspace{1cm}m | n | o | p | q | r | s | t | u | v | w | x | y | z |\\
  \hspace{1cm}A | B | C | D | E | F | G | H | I | J | K | L | M | N |\\
  \hspace{1cm}O | P | Q | R | S | T | U | V | W | X | Y | Z}\\
\end{flushleft}


This alphabet may be arbitrarily restricted, or extended with any other
distinctive character (i.e. character not coinciding with any digit,
logical value or delimiter).

Letters do not have individual meaning. They are used for forming
identifiers and strings \footnote{It should be particularly noted that
throughout the reference language underlining is used for defining
independent basic symbols (see sections \ref{LblBasicSymbolsLogical}
and \ref{LblBasicSymbolsDelimiters}). These are understood to have no
relation to the individual letters of which they are composed. Within
the present report underlining will be used for no other purposes.}
(cf. sections \ref{LblBasicSymbolsIdentifiers} Identifiers,
\ref{LblBasicSymbolsStrings} Strings).

\subsection{Digits and logical values}

\subsubsection{Digits}
\label{LblBasicSymbolsDigits}

\begin{flushleft}
\vspace{0.2em}\texttt{<digit> ::= 0 | 1 | 2 | 3 | 4 | 5 | 6 | 7 | 8 | 9}\\
\end{flushleft}

Digits are used for forming numbers, identifiers, and strings.


\subsubsection{Logical values}
\label{LblBasicSymbolsLogical}

\begin{flushleft}
\vspace{0.2em}\texttt{<logical value> ::= \keyw{true} | \keyw{false}}\\
\end{flushleft}

The logical values have a fixed obvious meaning.



\subsection{Delimiters}
\label{LblBasicSymbolsDelimiters}

\begin{flushleft}
\vspace{0.2em}\texttt{<delimiter> ::= <operator> | <separator> | <bracket> |\\
  \hspace{1cm}<declarator> | <specificator>}\\

\vspace{0.2em}\texttt{<operator> ::= <arithmetic operator> | <relational operator> |\\
  \hspace{1cm}<logical operator> | <sequential operator>}\\

\vspace{0.2em}\texttt{<arithmetic operator> ::= $+$ | $-$ | $\times$ | $/$ | $\div$ | $\uparrow$}\\

\vspace{0.2em}\texttt{<relational operator> ::= $<$ | $\leq$ | $=$ | $\geq$ | $>$ | $\neq$}\\

\vspace{0.2em}\texttt{<logical operator> ::= $\equiv$ | $\Rightarrow$ | $\vee$ | $\wedge$ | $\neg$}\\

\vspace{0.2em}\texttt{<sequential operator> ::= \keyw{goto} | \keyw{if} | \keyw{then} |
  \keyw{else} | \keyw{for} | \keyw{do}\footnotemark}\\

\vspace{0.2em}\texttt{<separator> ::= $,$ | $.$ | $_{10}$ | $:$ | $;$ | $:=$ | \blankchar{} | \keyw{step} |\\
  \hspace{1cm}\keyw{until} | \keyw{while} | \keyw{comment}}\\

\vspace{0.2em}\texttt{<bracket> ::= $($ | $)$ | $[$ | $]$ | $`$ | $'$ | \keyw{begin} | \keyw{end}}\\

\vspace{0.2em}\texttt{<declarator> ::= \keyw{own} | \keyw{Boolean} | \keyw{integer} |\\
  \hspace{1cm}\keyw{real} | \keyw{array} | \keyw{switch} | \keyw{procedure}}\\

\vspace{0.2em}\texttt{<specificator> ::= \keyw{string} | \keyw{label} | \keyw{value}}\\

\end{flushleft}

\footnotetext{\keyw{do} is used in for statements. It has no relation
    to the \i{do} of the preliminary report, which is not included in
    Algol60.}


Delimiters have a fixed meaning which for the most part is obvious or
else will be given at the appropriate place in the sequel.

Typographical features such as blank space or change to a new line
have no significance in the reference language. They, however, be used
freely for facilitating reading.

For the purpose of including text among the symbols of a program the
following "comment" conventions hold:

\begin{flushleft}
\begin{tabular*}{15.0cm}{p{11.0cm}l}
The sequence of basic symbols:          &is equivalent to\\

\texttt{; \keyw{comment} <any sequence not containing ;>;}                & ;\\
\texttt{\keyw{begin} \keyw{comment} <any sequence not containing ;>;}     & \keyw{begin}\\
\texttt{\keyw{end} <any sequence not containing \keyw{end} or ; or \keyw{else}>;} & \keyw{end}\\
\end{tabular*}
\end{flushleft}

By equivalence is here meant that any of the three structures shown in
the left hand column may be replaced, in any occurrence outside of
strings, by the symbol shown in the same line in the right hand
column without any effect on the action of the program. It is further
understood that the comment structure encountered first in the text
when reading from left to right has precedence in being replaced over
later structures contained in the sequence.



\subsection{Identifiers}
\label{LblBasicSymbolsIdentifiers}


\subsubsection{Syntax}
\label{LblBasicSymbolsIdentifiersSyntax}


\begin{flushleft}
\vspace{0.2em}\texttt{<identifier> ::= <letter> | <identifier> <letter> | <identifier> <digit>}\\
\end{flushleft}



\subsubsection{Examples}

\begin{verbatim}
q
Soup
V17a
a34kTMNs
MARILYN
\end{verbatim}


\subsubsection{Semantics}
\label{LblBasicSymbolsIdentifiersSemantics}

Identifiers have no inherent meaning, but serve for the identification
of simple variables, arrays, labels, switches, and procedures. They
may be chosen freely (cf. however section \ref{LblStandardFunctions}
Standard functions).

The same identifiers cannot be used to denote two different quantities
except when these quantities have disjoint scopes as defined by the
declarations of the program (cf section \ref{LblQuantitiesKindsAndScopes}
Quantities, kinds and scopes and section \ref{LblDeclarations}
Declarations).



\subsection{Numbers}
\label{LblBasicSymbolsNumbers}


\subsubsection{Syntax}
\label{LblBasicSymbolsNumbersSyntax}

\begin{flushleft}
\vspace{0.2em}\texttt{<unsigned integer> ::= <digit> | <unsigned integer> <digit>}\\

\vspace{0.2em}\texttt{<integer> ::= <unsigned integer> | $+$ <unsigned integer> |\\
  \hspace{1.0cm}$-$ <unsigned integer>}\\

\vspace{0.2em}\texttt{<decimal fraction> ::= $.$ <unsigned integer>}\\

\vspace{0.2em}\texttt{<exponential part> ::= $_{10}$ <integer>}\\

\vspace{0.2em}\texttt{<decimal number> ::= <unsigned integer> | <decimal fraction> | \\
  \hspace{1.0cm}<unsigned integer> <decimal fraction>}\\

\vspace{0.2em}\texttt{<unsigned number> ::= <decimal number> | <exponential part> |\\
  \hspace{1.0cm}<decimal number> <exponential part>}\\

\vspace{0.2em}\texttt{<number> ::= <unsigned number> | $+$ <unsigned number> |\\
  \hspace{1.0cm}$-$ <unsigned number>}\\
\end{flushleft}


\subsubsection{Examples}

\begin{tabular*}{16.0cm}{lll}
\texttt{0}       &\texttt{-200.084}       &\texttt{-.083$_{10}$-02} \\
\texttt{177}     &\texttt{+07.43$_{10}$8} &\texttt{-$_{10}$7}       \\
\texttt{.5384}   &\texttt{9.34$_{10}$+10} &\texttt{$_{10}$-4}       \\
\texttt{+0.7300} &\texttt{2$_{10}$-4}     &\texttt{+$_{10}$+5}      \\
\end{tabular*}

\subsubsection{Semantics}
\label{LblBasicSymbolsNumbersSemantics}

Decimal numbers have their conventional meaning.  The exponent part is
scale factor expressed as an integral power of 10.

\subsubsection{Types}
\label{LblBasicSymbolsNumbersTypes}

Integers are of the type \keyw{integer}. All other numbers are of type
\keyw{real} (cf. section 5.1 Type declarations).



\subsection{Strings}
\label{LblBasicSymbolsStrings}


\subsubsection{Syntax}
\label{LblBasicSymbolsStringsSyntax}

\begin{flushleft}
\vspace{0.2em}\texttt{<proper string> ::= \\
  \hspace{1.0em}<any sequence of symbols not containing ` or ' >\\
  \hspace{1.0em}| <empty>}\\

\vspace{0.2em}\texttt{<open string> ::= <proper string> `<open string>' |\\
  \hspace{1.0em}<open string><open string>}\\

\vspace{0.2em}\texttt{<string> ::= `<open string>'}\\
\end{flushleft}


\subsubsection{Examples}

\begin{flushleft}
\texttt{`5k,,-`[[[`$\vee$=/:'Tt''\\
`This\blankchar{}is\blankchar{}a\blankchar{}`string''\\
}\end{flushleft}

\subsubsection{Semantics}
\label{LblBasicSymbolsStringsSemantics}

In order to enable the language to handle arbitrary sequences of basic
symbols the string quotes \texttt{`} and \texttt{'} are introduced.
The Symbol \blankchar denotes a space. It has no significance outside
strings.  Strings are used as actual parameters of procedures
(cf. sections \ref{LblFunctionDesignators} Function designators and
\ref{LblProcedureStatements} Procedure Statements).


\subsection{Quantities, kinds and scopes}
\label{LblQuantitiesKindsAndScopes}



The following kinds of quantities are distinguished: simple variables,
arrays, labels, switches, and procedures.

The scope of a quantity is the set of statements and expressions in
which the declaration of the identifier associated with that quantity
is valid.  For labels see section \ref{LblCompoundStatementsSemantics}.



\subsection{Values and types}
\label{LblValuesAndTypes}

A value is an ordered set of numbers (special case: a single number),
an ordered set of logical values (special case: a single logical
value), or a label.

Certain of the syntactic units are said to possess values.  These
values will in general change during the execution of the program The
values of expressions and their constituents are defined in section 3.
The value of an array identifier is the ordered set of values of the
corresponding array of subscripted variables (cf. section
\ref{LblVariableSubscriptsP1}).

The various ``types'' (\keyw{integer}, \keyw{real}, \keyw{Boolean})
basically denote properties of values.  The types associated with
syntactic units refer to the values of these units.



\section{Expressions}
\label{LblExpressions}


In the language the primary constituents of the programs describing
algorithmic processes are arithmetic, Boolean, and designational
expressions.  Constituents of the expressions, except for certain
delimiters, are logical values, numbers, variables, function designators,
and elementary arithmetic, relational, logical, and sequential
operators.  Since the syntactic definition of both variables and
function designators contains expressions, the definition of
expressions, and their constituents, is necessarily recursive.


\begin{flushleft}
\vspace{0.2em}\texttt{<expression> ::= <arithmetic expression> |\\
  \hspace{1.0cm}<Boolean expression> | <designational expression>}\\
\end{flushleft}



\subsection{Variables}
\label{LblVariables}


\subsubsection{Syntax}
\label{LblVariablesSyntax}

\begin{flushleft}
\vspace{0.2em}\texttt{<variable identifier> ::= <identifier>}\\

\vspace{0.2em}\texttt{<simple variable> ::= <variable identifier>}\\

\vspace{0.2em}\texttt{<subscript expression> ::= <arithmetic expression>}\\

\vspace{0.2em}\texttt{<subscript list> ::= <subscript expression> |\\
  \hspace{1.0cm}<subscript list> , <subscript expression>}\\

\vspace{0.2em}\texttt{<array identifier> ::= <identifier>}\\

\vspace{0.2em}\texttt{<subscripted value> ::= <array identifier> [ <subscript list> ]}\\

\vspace{0.2em}\texttt{<variable> ::= <simple variable> | <subscripted variable>}\\
\end{flushleft}


\subsubsection{Examples}

\begin{flushleft}
\texttt{epsilon\\
detA\\
a17\\
Q[7,2]\\
x[sin(n$\times$pi/2),Q[3,n,4]]\\
}\end{flushleft}


\subsubsection{Semantics}
\label{LblVariablesSemantics}

A variable is a designation given to a single value.  This value may
be used in expressions for forming other values and may be changed at
will by means of assignment statements (section 4.2).  The type of the
value of a particular variable is defined in the declaration for the
variable itself (cf. section \ref{LblTypeDeclarations} Type
declarations) or for the corresponding array identifier (cf. section
\ref{LblArrayDeclarations} Array declarations),

\subsubsection{Subscripts}
\label{LblVariableSubscripts}

\paragraph{} \label{LblVariableSubscriptsP1}
Subscripted variables designate values which are components of
multidimensional arrays (cf. section \ref{LblArrayDeclarations} Array
declarations).  Each arithmetic expression of the subscript list
occupies one subscript position of the subscripted variable and is
called a subscript.  The complete list of subscripts is enclosed in
the subscript brackets [ ].  The array component referred to by a
subscripted variable is specified by the actual numerical value of its
subscripts (cf. section \ref{LblArithmeticExpressions} Arithmetic
expressions).

\paragraph{} \label{LblVariableSubscriptsP2}
Each subscript position acts like a variable of type \keyw{integer}
and the evaluation of the subscript is understood to be equivalent to
an assignment to this fictitious variable (cf. section
\ref{LblAssignmentTypes}).  The value of the subscripted variable is
defined only if the value of the subscript expression is within the
subscript bounds of the array (cf. section \ref{LblArrayDeclarations}
Array declarations).



\subsection{Function designators}
\label{LblFunctionDesignators}


\subsubsection{Syntax}
\label{LblFunctionDesignatorsSyntax}

\begin{flushleft}
\vspace{0.2em}\texttt{<procedure identifier> ::= <identifier>}\\

\vspace{0.2em}\texttt{<actual parameter> ::= <string> | <expression> |\\
  \hspace{1.0cm}<array identifier> | <switch identifier> |\\
  \hspace{1.0cm}<procedure identifier>}\\

\vspace{0.2em}\texttt{<letter string> ::= <letter> | <letter string> <letter>}\\

\vspace{0.2em}\texttt{<parameter delimiter> ::= , | ) <letter string> : (}\\

\vspace{0.2em}\texttt{<actual parameter list> ::= <actual parameter> |\\
  \hspace{1.0cm}<actual parameter list> <parameter delimiter>\\
  \hspace{1.0cm}<actual parameter>}\\

\vspace{0.2em}\texttt{<actual parameter part> ::= <empty> |\\
  \hspace{1.0cm}( <actual parameter list> )}\\

\vspace{0.2em}\texttt{<function designator> ::= <procedure identifier>\\
  \hspace{1.0cm}<actual parameter part>}\\
\end{flushleft}


\subsubsection{Examples}

\begin{flushleft}
\texttt{sin(a-b)\\
J(v+s,n)\\
R\\
S(s-5) Temperature: (T) Pressure: (P)\\
Compile (`:=') Stack: (Q)}\\
\end{flushleft}

\subsubsection{Semantics}
\label{LblFunctionDesignatorsSemantics}

Function designators define single numerical or logical values which
result through the application of given sets of rules defined by a
procedure declaration (cf. section \ref{LblProcedureDeclarations}
Procedure declarations) to fixed sets of actual parameters.  The rules
governing specification of actual parameters are given in section
\ref{LblProcedureStatements} Procedure statements.  Not every
procedure declaration defines the value of a function designator.

\subsubsection{Standard functions}
\label{LblStandardFunctions}

Certain identifiers should be reserved for the standard functions of
analysis, which will be expressed as procedures.  It is recommended
that this reserved list should contain:


\begin{description}

\item[abs (E)] for the modulus (absolute value) of the value of the
        expression E

\item[sign (E)] for the sign of the value of E (+1 for E$>$0, 0 for E$=$0,
        -1 for E$<$0)

\item[sqrt (E)] for the square root of the value of E

\item[sin (E)] for the sine of the value of E

\item[cos (E)] for the cosine of the value of E

\item[arctan (E)] for the principal value of the arctangent of the value
        of E

\item[ln (E)] for the natural logarithm of the value of E

\item[exp (E)] for the exponential function of the value
        of E (e $\uparrow$ E)
\end{description}

These functions are all understood to operate indifferently on
arguments both of type \keyw{real} and \keyw{integer}.  They will all
yield values of type \keyw{real}, except for sign (E) which will have
values of type \keyw{integer}.  In a particular representation these
function may be available without explicit declarations (cf. section
5.  Declarations).

\subsubsection{Transfer functions}
\label{LblTransferFunctions}

It is understood that transfer functions between any pair of
quantities and expressions my be defined.  Among the standard
functions it is recommended that there be one, namely

\begin{flushleft}\texttt{
    entier (E),\\
}\end{flushleft}

which ``transfers'' an expression of real type to one of integer type,
and assigns to it the value which is the largest integer not greater
than the value of E.


\subsection{Arithmetic expressions}
\label{LblArithmeticExpressions}

\subsubsection{Syntax}
\label{LblArithmeticExpressionsSyntax}

\begin{flushleft}
\vspace{0.2em}\texttt{<adding operator> ::= $+$ | $-$}\\

\vspace{0.2em}\texttt{<multiplying operator> ::= $\times$ | $/$ | $\div$}\\

\vspace{0.2em}\texttt{<primary> ::= <unsigned number> | <variable> |\\
  \hspace{1.0cm}<function designator> | $($ <arithmetic expression> $)$}\\

\vspace{0.2em}\texttt{<factor> ::= <primary> | <factor> |\\
  \hspace{1.0cm}<factor> $\uparrow$ <primary>}\\

\vspace{0.2em}\texttt{<term> ::= <factor> | <term> <multiplying operator> <factor>}\\

\vspace{0.2em}\texttt{<simple arithmetic expression> ::= <term> |\\
  \hspace{1.0cm}<adding operator> <term> |\\
  \hspace{1.0cm}<simple arithmetic expression> <adding operator> <term>}\\

\vspace{0.2em}\texttt{<if clause> ::= \keyw{if} <Boolean expression> \keyw{then}}\\

\vspace{0.2em}\texttt{<arithmetic expression> ::= <simple arithmetic expression> |\\
  \hspace{1.0cm}<if clause> <simple arithmetic expression> \keyw{else} <arithmetic expression>}\\

\end{flushleft}


\subsubsection{Examples}

Primaries:

\begin{flushleft}
\texttt{7.394$_{10}$-8\\
sum\\
w[i+2,8]\\
cos(y+z$\times$3)\\
(a-3/y+vu$\uparrow$8)\\
}\end{flushleft}

Factors:

\begin{flushleft}
\texttt{omega\\
sum$\uparrow$cos(y+z$\times$3)\\
7.394$_{10}$-8$\uparrow$w[i+2,8]$\uparrow$(a-3/y+vu$\uparrow$8)
}\end{flushleft}

Terms:

\begin{flushleft}
\texttt{U\\
omega$\times$sum$\uparrow$cos(y+z$\times${}3)/7.394$_{10}$-8$\uparrow$w[i+2,8]$\uparrow$(a-3/y+vu$\uparrow$8)\\
}\end{flushleft}

Simple arithmetic expressions:

\begin{flushleft}
\texttt{U-Yu+omega$\times$sum$\uparrow$cos(y+z$\times$3)/7.394$_{10}$-8$\uparrow$w[i+2,8]$\uparrow$(a-3/y+vu$\uparrow$8)\\
}\end{flushleft}

Arithmetic expressions:

\begin{flushleft}
\texttt{w$\times$u-Q(S+Cu)$\uparrow$2\\
\keyw{if} q$>$0 \keyw{then} S+3$\times$Q/A \keyw{else} 2$\times$S+3$\times$q\\
\keyw{if} a$<$0 \keyw{then} U+V \keyw{else} \keyw{if} a$\times$b$>$17 \keyw{then} U/V \keyw{else} \keyw{if} k$\neq$y \keyw{then}
    V/U \keyw{else} 0\\
a$\times$sin(omega$\times$t)\\
0.57$_{10}$12$\times$a[N$\times$(N-1)/2,0]\\
(A$\times$arctan(y)+Z)$\uparrow$(7+Q)\\
\keyw{if} q \keyw{then} n-1 \keyw{else} n\\
\keyw{if} a$<$0 \keyw{then} A/B \keyw{else} \keyw{if} b=0 \keyw{then} B/A \keyw{else} z\\
}\end{flushleft}


\subsubsection{Semantics}
\label{LblArithmeticExpressionsSemantics}

An arithmetic expression is a rule for computing a numerical value.
In case of simple arithmetic expressions this value is obtained by
executing the indicated arithmetic operations on the actual numerical
values of the primaries of the expression, as explained in detail in
section \ref{LblOperatorsAndTypes} below.  The actual numerical value
for a primary is obvious in the case of numbers.  For variables it is
the current value (assigned last in the dynamic sense), and for
function designators it is the value arising from the computing rules
defining the procedure (cf. section \ref{LblValuesOfFuncDesignators}
Values of function designators) when applied to the current values of
the procedure parameters given in the expression.  Finally, for
arithmetic expressions enclosed in parentheses the value must through
a recursive analysis be expressed in terms of the values of primaries
of the other three kinds.

In the more general arithmetic expression, which include if clauses,
one out of several simple arithmetic expressions is selected on the
basis of the actual values of the Boolean expression (cf. section
\ref{LblBooleanExpressions} Boolean expressions).  This selection is
made as follows: The Boolean expressions of the if clauses are
evaluated one by one in the sequence from left to right until one
having the value \keyw{true} is found.  The value of the arithmetic
expression is then the value of the first arithmetic expression
following this Boolean (the largest arithmetic expression found in
this position is understood).  The construction:

\begin{flushleft}
\texttt{\keyw{else} <simple arithmetic expression>}\\
\end{flushleft}

is equivalent to the construction:

\begin{flushleft}
\texttt{\keyw{else} \keyw{if} \keyw{true} \keyw{then} <simple arithmetic expression>}\\
\end{flushleft}


\subsubsection{Operators and types}
\label{LblOperatorsAndTypes}

Apart from the Boolean expressions of if clauses, the constituents of
simple arithmetic expressions must be of types \keyw{real} or
\keyw{integer} (cf. section \ref{LblTypeDeclarations} Type
declarations).  The meaning of the basic operators and the types of
the expressions to which they lead are given by the following rules:

\paragraph{} \label{LblOperatorsAndTypesP1}
The operators +, -, and $\times$ have the conventional meaning
(addition, subtraction, and multiplication).  The type of the
expression will by \keyw{integer} if both of the operands are of
\keyw{integer} type, otherwise \keyw{real}.

\paragraph{} \label{LblOperatorsAndTypesP2}
The operations \texttt{<term> / <factor> and <term> $\div$ <factor>}
both denote division, to be understood as a multiplication of the term
by the reciprocal of the factor with due regard to the rules of
precedence (cf. section \ref{LblArithmeticExpressionsPrecedence}).
Thus for example

\begin{flushleft}
\texttt{a/b$\times$7/(p-q)$\times$v/s}\\
\end{flushleft}

means

\begin{flushleft}
\texttt{$((((a\times(b^{-1}))\times7)\times((p-q)^{-1}))\times{}v)\times(s^{-1})$}\\
\end{flushleft}

The operator \texttt{/} is defined for all four combinations of types
\keyw{real} and \keyw{integer} and will yield results of \keyw{real}
type in any case.  The operator $\div$ is defined only for two
operands of type \keyw{integer} and will yield a result of type
\keyw{integer}, mathematically defined as follows:

\begin{flushleft}
\texttt{a $\div$ b = sign(a/b) $\times$ entier(abs(a/b))}\\
\end{flushleft}

(cf. sections \ref{LblStandardFunctions} and \ref{LblTransferFunctions}).

\paragraph{} \label{LblOperatorsAndTypesP3}
The operation \texttt{<factor> $\uparrow$ <primary>} denotes
exponentiation, where the factor is the base and the primary is the
exponent.  Thus for example

\begin{flushleft}
\texttt{2 $\uparrow$ n $\uparrow$ k means $(2^n)^k$}\\
\end{flushleft}

while

\begin{flushleft}
\texttt{2 $\uparrow$ (n $\uparrow$ m)  means  $2^{(n^m)}$}\\
\end{flushleft}


Writing i for a number of \keyw{integer} type, r for a number of
\keyw{real} type, and a for a number of ether \keyw{integer} or
\keyw{real} type, the result is given by the following rules:


\begin{enumerate}
\item[$a \uparrow i$]
  \begin{flushleft}
    if $i > 0$:\\
    \hspace{1.0cm}$a \times a \times \cdots \times a$ (i times), of the same type as a.\\

    if $i = 0$:\\

    \hspace{1.0cm}if $a \neq 0$:\\
    \hspace{2.0cm}$1$, of the same type as a.\\

    \hspace{1.0cm}if $a = 0$:\\
    \hspace{2.0cm}undefined.\\

    if $i < 0$:\\

    \hspace{1.0cm}if $a \neq 0$:\\
    \hspace{2.0cm}$1/(a \times a \times a \times \cdots \times a)$\\
    \hspace{2.0cm}(the denominator has -i factors), of type \keyw{real}.\\

    \hspace{1.0cm}if $a = 0$:\\
    \hspace{2.0cm}undefined.\\
  \end{flushleft}

\item[$a \uparrow r$]

  \begin{flushleft}
    if $a > 0$:\\
    \hspace{1.0cm}$exp(r \times ln(a))$, of type \keyw{real}.\\

    if $a = 0$\\
    \hspace{1.0cm}if $r > 0$:\\
    \hspace{2.0cm}$0.0$, of type \keyw{real}.\\
    \hspace{1.0cm}if $r < 0$:\\
    \hspace{2.0cm}undefined.\\

    if $a < 0$:\\
    \hspace{1.0cm}always undefined.\\
  \end{flushleft}

\end{enumerate}


\subsubsection{Precedence of operators}
\label{LblArithmeticExpressionsPrecedence}

The sequence of operations within one expression is generally from
left to right, with the following additional rules:

\paragraph{} \label{LblArithmeticExpressionsPrecedenceP1}
According to the syntax given in section
\ref{LblArithmeticExpressionsSyntax} the following rules of precedence
hold:

\begin{tabular*}{10cm}{rl}
  first:  & $\uparrow$\\
  second: & $\times / \div$\\
  third:  & $+ -$\\
\end{tabular*}

\paragraph{} \label{LblArithmeticExpressionsPrecedenceP2}
The expression between a left parenthesis and the matching right
parenthesis is evaluated by itself and this value is used in
subsequent calculations.  Censequently the desired order of execution
of operations within an expression can always be arranged by
appropriate positioning of parenthesis.

\subsubsection[Arithmetics of real quantities]{Arithmetics of \keyw{real} quantities}
\label{LblArithmeticExpressionsRealQuantities}

Numbers and variables of type \keyw{real} must be interpreted in the
sense of numerical analysis, i.e.  as entities defined inherently with
only a finite accuracy.  Similarly, the possibility of the occurrence
of a finite deviation from the mathematically defined result in any
arithmetic expression is explicitly understood.  No exact arithmetic
will be specified, however, and it is indeed understood that different
hardware representations may evaluate arithmetic expressions
differently.  The control of the possible consequences of such
differences must be carried out by the methods of numerical analysis.
This control must be considered a part of the process to be described,
and will therefore be expressed in terms of the language
itself.



\subsection{Boolean expressions}
\label{LblBooleanExpressions}


\subsubsection{Syntax}
\label{LblBooleanExpressionsSyntax}

\begin{flushleft}

\vspace{0.2em}\texttt{<relational operator> ::= \mlt{} | \mle{} | \meq{} | \mge{} | \mgt{} | \mne{}}\\

\vspace{0.2em}\texttt{<relation> ::= <simple arithmetic expression> \\
  \hspace{1.0cm}<relational operator> <simple arithmetic expression>}\\

\vspace{0.2em}\texttt{<Boolean primary> ::= <logical value> | <variable> |\\
  \hspace{1.0cm}<function designator> | <relation> | ( <Boolean expression> )}\\

\vspace{0.2em}\texttt{<Boolean secondary> ::= <Boolean primary> | \mlnot{} <Boolean primary>}\\

\vspace{0.2em}\texttt{<Boolean factor> ::= <Boolean secondary> |\\
  \hspace{1.0cm}<Boolean factor> \mland{} <Boolean secondary>}\\

\vspace{0.2em}\texttt{<Boolean term> ::= <Boolean factor> |\\
  \hspace{1.0cm}<Boolean term> \mlor{} <Boolean factor>}\\

\vspace{0.2em}\texttt{<implication> ::= <Boolean term> |\\
  \hspace{1.0cm}<implication> \mlimp{} <Boolean term>}\\

\vspace{0.2em}\texttt{<simple Boolean> ::= <implication> |\\
  \hspace{1.0cm}<simple Boolean> \mleqv{} <implication>}\\

\vspace{0.2em}\texttt{<Boolean expression> ::= <simple Boolean> |\\
  \hspace{1.0cm}<if clause> <simple Boolean> \keyw{else} <Boolean expression>}\\

\end{flushleft}


\subsubsection{Examples}

\begin{flushleft}
\texttt{x=-2\\
Y\mgt{}V\mlor{}z\mlt{}q\\
a+b\mgt{}-5\mland{}z-d\mgt{}q\mpow{}2\\
p\mland{}q\mlor{}x\mne{}y \\
g\mleqv{}\mlnot{}a\mland{}b\mland{}\mlnot{}c\mlor{}d\mlor{}e\mlimp{}\mlnot{} f\\
\keyw{if} k\mlt{}1 \keyw{then} s\mlt{}w \keyw{else} h \mle{} c\\
\keyw{if} \keyw{if} \keyw{if} a \keyw{then} b \keyw{else} c \keyw{then} d \keyw{else} f \keyw{then} g \keyw{else} h \mlt{} k\\
}\end{flushleft}


\subsubsection{Semantics}
\label{LblBooleanExpressionsSemantics}

A Boolean expression is a rule for computing a logical value.  The
principles of evaluation are entirely analogous to those given for
arithmetic expressions in section \ref{LblArithmeticExpressionsSemantics}.

\subsubsection{Types}
\label{LblBooleanExpressionsTypes}

Variables and function designators entered as Boolean primaries must
be declared \keyw{Boolean} (cf. section \ref{LblTypeDeclarations} Type
declarations and section \ref{LblValuesOfFuncDesignators} Value of
function designators).

\subsubsection{The operators}
\label{LblBooleanExpressionsOperators}

Relations take on the value \keyw{true} whenever the corresponding
relation is satisfied for the expressions involved, otherwise
\keyw{false}.

The meaning of the logical operators \mlnot{} (not), \mland{} (and),
\mlor{} (or), \mlimp{} (implies), and \mleqv{} (equivalent), is given
by the following function table.

\vspace{1em}
\begin{tabular*}{16.0cm}{l|cccc}
b1       & \keyw{false} & \keyw{false} & \keyw{true}  & \keyw{true}\\
b2       & \keyw{false} & \keyw{true}  & \keyw{false} & \keyw{true}\\
\cline{1-5}
\mlnot{} b1     & \keyw{true}  & \keyw{true}  & \keyw{false} & \keyw{false}\\
b1 \mland{} b2  & \keyw{false} & \keyw{false} & \keyw{false} & \keyw{true}\\
b1 \mlor{} b2   & \keyw{false} & \keyw{true}  & \keyw{true}  & \keyw{true}\\
b1 \mlimp{} b2  & \keyw{true}  & \keyw{true}  & \keyw{false} & \keyw{true}\\
b1 \mleqv{} b2  & \keyw{true}  & \keyw{false} & \keyw{false} & \keyw{true}\\
\end{tabular*}


\subsubsection{Precedence of operators}
\label{LblBooleanExpressionsOperatorsPrecedence}

The sequence of operations within one expression is generally from
left to right, with the following additional rules:

According to the syntax given in section
\ref{LblBooleanExpressionsSyntax} the following rules of precedence
hold:

\begin{tabular*}{10cm}{rl}
  first:  & arithmetic expressions according to section \ref{LblArithmeticExpressionsPrecedence}.\\
  second: & \mlt{} \mle{} \meq{} \mge{} \mgt{} \mne{}\\
  third:  & \mlnot{}\\
  fourth: & \mland{}\\
  fifth:  & \mlor{}\\
  sixth:  & \mlimp{}\\
  seventh:& \mleqv{}\\
\end{tabular*}

The use of parentheses will be interpreted in the sense given in
section \ref{LblArithmeticExpressionsPrecedence}.



\subsection{Designational expressions}
\label{LblDesignationalExpressions}


\subsubsection{Syntax}
\label{LblDesignationalExpressionsSyntax}

\begin{flushleft}
\vspace{0.2em}\texttt{<label> ::= <identifier> | <unsigned integer>}\\

\vspace{0.2em}\texttt{<switch identifier> ::= <identifier>}\\

\vspace{0.2em}\texttt{<switch designator> ::= <switch identifier> [<subscript expression>]}\\

\vspace{0.2em}\texttt{<simple designational expression> ::= <label> |\\
  \hspace{1.0cm}<switch designator> | (<designational expression>)}\\

\vspace{0.2em}\texttt{<designational expression> ::= <simple designational expression> |\\
  \hspace{1.0cm}<if clause> <simple designational expression>\\
  \hspace{1.0cm}\keyw{else} <designational expression>}\\
\end{flushleft}

\subsubsection{Examples}

\begin{flushleft}
\texttt{17\\
p9\\
Coose[n-1]\\
Town [\keyw{if} y\mlt{}0 \keyw{then} N \keyw{else} N+1]\\
\keyw{if} Ab\mlt{}c \keyw{then} 17 \keyw{else} q[\keyw{if} w \mle{} 0 \keyw{then} 2 \keyw{else} n]}\\
\end{flushleft}

\subsubsection{Semantics}
\label{LblDesignationalExpressionsSemantics}

A designational expression is a rule for obtaining a label of a
statement (cf. section 4.  Statements).  Again the principle of the
evaluation is entirely analogous to that of arithmetic expressions
(section \ref{LblArithmeticExpressionsSemantics}).  In the general
case the Boolean expression of the if clauses will select a simple
designational expression.  If this is a label the desired result is
already found.  A switch designator refers refers to the corresponding
switch declaration (cf. section \ref{LblSwitchDeclarations} Switch
declarations) and by the actual numerical value of its subscript
expression selects one of the designational expressions listed in the
switch declaration by counting these from left to right.  Since the
designational expression thus selected may again by a switch
designator this evaluation is obviously a recursive process.

\subsubsection{The subscript expression}

The evaluation of the subscript expression is analogous to that of
subscripted variables (cf. section \ref{LblVariableSubscriptsP2}).
The value of a switch designator is defined only if the subscript
expression assumes one of the positive values 1, 2, 3, ..., n, where n
is the number of entries in the switch list.

\subsubsection{Unsigned integers as labels}

Unsigned integers used as labels have the property that leading zeroes
do not affect their meaning, e.g. \texttt{00127} denotes the same
label as \texttt{217}.



\section{Statements}
\label{LblStatements}


The units of operation within the language are called statements.
They will normally be executed consecutively as written.  However,
this sequence of operations may be broken by go to statements, which
define their successor explicitly, and shortened by conditional
statements, which may cause certain statements to be skipped.

In order to make it possible to define a specific dynamic succession,
statements may be provided with labels.

Since sequences of statements may be grouped together into compound
statements and blocks the definition of statement must necessarily be
recursive.  Also since declarations, described in section 5, enter
fundamentally into the syntactic structure, the syntactic definition
of statements must suppose declarations to be already defined.



\subsection{Compound statements and blocks}


\subsubsection{Syntax}
\label{LblCompoundStatementsSyntax}

\begin{flushleft}
\vspace{0.2em}\texttt{<unlabelled basic statement> ::= <assignment statement> |\\
  \hspace{1.0cm}<go to statement> | <dummy statement> | <procedure statement>}\\

\vspace{0.2em}\texttt{<basic statement> ::= <unlabelled basic statement> |\\
  \hspace{1.0cm}<label>: <basic statement>}\\

\vspace{0.2em}\texttt{<unconditional statement> ::= <basic statement> |\\
  \hspace{1.0cm}<compound statement> | <block>}\\

\vspace{0.2em}\texttt{<statement> ::= <unconditional statement> |\\
  \hspace{1.0cm}<conditional statement> | <for statement>}\\

\vspace{0.2em}\texttt{<compound tail> ::= <statement> \keyw{end} |\\
  \hspace{1.0cm}<statement> ; <compound tail>}\\

\vspace{0.2em}\texttt{<block head> ::= \keyw{begin} <declaration> |\\
  \hspace{1.0cm}<block head> ; <declaration>}\\

\vspace{0.2em}\texttt{<unlabelled compound> ::= \keyw{begin} <compound tail>}\\

\vspace{0.2em}\texttt{<unlabelled block> ::= <block head> ; <compound tail>}\\

\vspace{0.2em}\texttt{<compound statement> ::= <unlabelled compound> |\\
  \hspace{1.0cm}<label>: <compound statement>}\\

\vspace{0.2em}\texttt{<block> ::= <unlabelled block> | <label>: <block>}\\

\vspace{0.2em}\texttt{<program> ::= <block> | <compound statement>}\\
\end{flushleft}

This syntax may be illustrated as follows: Denoting arbitrary
statements, declarations, and labels, by the letters S, D, L,
respectively, the basic syntactic units take the forms:

Compound statement:
\begin{flushleft}
\texttt{L:L: ... \keyw{begin} S; S; ... S; S \keyw{end}}
\end{flushleft}

Block:
\begin{flushleft}
\texttt{L:L: ... \keyw{begin} D; D; .. D; S; S; ... S; S \keyw{end}}
\end{flushleft}

It should by kept in mind that each of the statements S may again be a
complete compound statement or a block.


\subsubsection{Examples}

Basic statements:

\begin{flushleft}
\texttt{a:=p+q\\
\keyw{goto} Naples\\
Start: Continue: W:=7.993}\\
\end{flushleft}

Compound statements:

\begin{flushleft}
\texttt{\keyw{begin} x:=0; \keyw{for} y:=1 \keyw{step} 1 \keyw{until} n \keyw{do}\\
  \hspace{2.0cm}x:=x+A[y];\\
  \hspace{1.25cm}\keyw{if} x\mgt{}q \keyw{then} \keyw{goto} STOP \keyw{else} \keyw{if} x\mgt{}w-2 \keyw{then}\\
  \hspace{2.0cm}\keyw{goto} S;\\
  \hspace{1.25cm}Aw: St: W:=x+bob \keyw{end}}\\
\end{flushleft}

Block:

\begin{flushleft}
\texttt{Q: \keyw{begin} \keyw{integer} i, k; \keyw{real} w;\\
  \hspace{1.85cm}\keyw{for} i:=1 \keyw{step} 1 \keyw{until} m \keyw{do}\\
  \hspace{1.85cm}\keyw{for} k:=i+1 \keyw{step} 1 \keyw{until} m \keyw{do}\\
  \hspace{1.85cm}\keyw{begin} w:=A[i,k];\\
  \hspace{3.1cm}A[i,k]:=A[k,i];\\
  \hspace{3.1cm}A[k,i]:=w \keyw{end} for i and k\\
  \hspace{0.65cm}\keyw{end} block Q}\\
\end{flushleft}


\subsubsection{Semantics}
\label{LblCompoundStatementsSemantics}

Every block automatically introduces a new level of nomenclature.
This is realized as follows: Any identifier occurring within the block
my through a suitable declaration (cf. section 5.  Declarations) be
specified to be local to the block in question.  This means (a) that
the entity represented by this identifier inside the blocks has no
existence outside it and (b) that any entity represented by this
identifier outside the block is completely inaccessible inside the
block.

Identifiers (except those representing labels) occurring within a
block and not being declared to this block will be non-local to it,
i.e.  will represent the same entity inside the block and in the level
immediately outside it.  A label separated by a colon from a
statement, i.e.  labelling that statement, behaves as though declared
in the head of the smallest embracing block, i.e.  the smallest block
whose brackets \keyw{begin} and \keyw{end} enclose that statement.  In
this context a procedure body must be considered as if it were
enclosed by \keyw{begin} and \keyw{end} and treated as a block.

Since a statement of a block may again itself be a block the concepts
local and non-local to a block must be understood recursively.  Thus
an identifier, which is non-local to a block A, may or may not be
non-local to the block B in which A is one statement.



\subsection{Assignment statements}


\subsubsection{Syntax}
\label{LblAssignmentStatementSyntax}

\begin{flushleft}
\vspace{0.2em}\texttt{<left part> ::= <variable> := | <procedure identifier> :=}\\

\vspace{0.2em}\texttt{<left part list> ::= <left part> | <left part list> <left part>}\\

\vspace{0.2em}\texttt{<assignment statement> ::=\\
  \hspace{1.0cm}<left part list> <arithmetic expression> |\\
  \hspace{1.0cm}<left part list> <Boolean expression>}\\
\end{flushleft}


\subsubsection{Examples}

\begin{flushleft}
\texttt{s:=p[0]:=n:=n+1+s\\
n:=n+1\\
A:=B/C-v-q\mtim{}S\\
S[v,k+2]:=3-arctan(s\mtim{}zeta)\\
V:=Q\mgt{}Y\mland{}Z}\\
\end{flushleft}


\subsubsection{Semantics}
\label{LblAssignmentStatementSemantics}

Assignment statements serve for assigning the value of an expression
to one or several variables or procedure identifiers.  Assignment to a
procedure identifier may only occur within the body of a procedure
defining the value of a function designator (cf. section
\ref{LblValuesOfFuncDesignators}).  The process will in the general
case be understood to take place in three steps as follows:

\paragraph{} \label{LblAssignmentStatementSemanticsStep1} Any subscript
expression occurring in the left part variables are evaluated in
sequence from left to right.

\paragraph{} The expression of the statement is evaluated.

\paragraph{} The value of the expression is assigned to all the left part
variables, with any subscript expressions having values as evaluated
in step \ref{LblAssignmentStatementSemanticsStep1}.

\subsubsection{Types}
\label{LblAssignmentTypes}

The type associated with all variables and procedure identifiers of a
left part list must be the same.  If the type is \keyw{Boolean}, the
expression must likewise be \keyw{Boolean}.  If the type is
\keyw{real} or \keyw{integer}, the expression must be arithmetic.  If
the type of the arithmetic expression differs from that associated
with the variables and procedure identifiers, appropriate transfer
functions are understood to be automatically invoked.  For transfer
from \keyw{real} to \keyw{integer} type the transfer function is
understood to yield a result equivalent to

\begin{flushleft}
\texttt{entier(E+0.5)}\\
\end{flushleft}

where E is the value of the expression.  The type associated with a
procedure identifier is given by the declarator which appears as the
first symbol of the corresponding procedure declaration (cf. section
\ref{LblValuesOfFuncDesignators}).


\subsection{Go to statements}
\label{LblGotoStatements}


\subsubsection{Syntax}
\label{LblGotoStatementsSyntax}

\begin{flushleft}
\vspace{0.2em}\texttt{<go to statement> ::= \keyw{goto} <designational expression>}\\
\end{flushleft}


\subsubsection{Examples}

\begin{flushleft}
\texttt{\keyw{goto} 8\\
\keyw{goto} exit [n+1]\\
\keyw{goto} Town [\keyw{if} y\mlt{}0 \keyw{then} N \keyw{else} N+1]\\
\keyw{goto} \keyw{if} Ab\mlt{}c \keyw{then} 17 \keyw{else} q [\keyw{if} w\mlt{}0 \keyw{then} 2 \keyw{else} n]}\\
\end{flushleft}

\subsubsection{Semantics}
\label{LblGotoStatementsSemantics}

A go to statement interrupts the normal sequence of operations,
defined by the write-up of statements, by defining its successor
explicitly by the value of a designational expression.  Thus the next
statement to be executed will be the one having this value as its
label.

\subsubsection{Restriction}

Since labels are inherently local, no go to statement can lead from
outside into a block.  A go to statement may, however, lead from
outside into a compound statement.

Go to an undefined switch designator.  A go to statement is equivalent
to a dummy statement if the designational expression is a switch
designator whose value is undefined.



\subsection{Dummy statements}
\label{LblDummyStatements}


\subsubsection{Syntax}
\label{LblDummyStatementsSyntax}

\begin{flushleft}
\vspace{0.2em}\texttt{<dummy statement> ::= <empty>}\\
\end{flushleft}


\subsubsection{Examples}

\begin{flushleft}
\texttt{L:\\
\keyw{begin} ....; John: \keyw{end}}\\
\end{flushleft}


\subsubsection{Semantics}
\label{LblDummyStatementsSemantics}

A dummy statement executes no operation.  It may serve to place a
label.




\subsection{Conditional statements}


\subsubsection{Syntax}
\label{LblConditionalStatementsSyntax}

\begin{flushleft}
\vspace{0.2em}\texttt{<if clause> ::= \keyw{if} <Boolean expression> \keyw{then}}\\

\vspace{0.2em}\texttt{<unconditional statement> ::= <basic statement> |\\
  \hspace{1.0cm}<compound statement> | <block>}\\

\vspace{0.2em}\texttt{<if statement> ::= <if clause> <unconditional statement>}\\

\vspace{0.2em}\texttt{<conditional statement> ::= <if statement> |\\
  \hspace{1.0cm}<if statement> \keyw{else} <statement> |\\
  \hspace{1.0cm}<if clause> <for statement> |\\
  \hspace{1.0cm}<label>: <conditional statement>}\\
\end{flushleft}


\subsubsection{Examples}

\begin{flushleft}
\texttt{\keyw{if} x\mgt{}0 \keyw{then} n:=n+1\\
\keyw{if} s\mgt{}u \keyw{then} V: q:=n+m \keyw{else} \keyw{goto} R\\
\keyw{if} s\mlt{}0 \mlor{} P\mle{}Q \keyw{then} AA: \keyw{begin} \keyw{if} q\mlt{}v \keyw{then} a:=v/s\\
\hspace{7.25cm}\keyw{else} y:=2\mtim{}a \keyw{end}\\
\hspace{2.8cm}\keyw{else} \keyw{if} v\mgt{}s \keyw{then} a:=v-q\\
\hspace{5.1cm}\keyw{else} \keyw{if} v\mgt{}s-1 \keyw{then} \keyw{goto} S}\\
\end{flushleft}

\subsubsection{Semantics}
\label{LblConditionalStatementsSemantics}

Conditional statements cause certain statements to be executed or
skipped depending on the running values of specified Boolean
expressions.

\paragraph{If statement}
\label{LblConditionalStatementsSemanticsIfStatement}
The unconditional statement of an if statement will be executed if the
Boolean expression of the if clause is true.  Otherwise it will be
skipped and the operation will be continued with the next statement.

\paragraph{Conditional statement}
\label{LblConditionalStatementsSemanticsCondStatement}
According to the syntax two different forms of conditional statements
are possible.  These may be illustrated as follows:

\begin{flushleft}
\texttt{\keyw{if} B1 \keyw{then} S1 \keyw{else} \keyw{if} B2 \keyw{then} S2 \keyw{else} S3; S4}\\
\end{flushleft}

\begin{flushleft}
\texttt{\keyw{if} B1 \keyw{then} S1 \keyw{else} \keyw{if} B2 \keyw{then} S2 \keyw{else} \keyw{if} B3 \keyw{then} S3; S4}\\
\end{flushleft}

Here B1 to B3 are Boolean expressions, while S1 to S3 are
unconditional statements.  S4 is the statement following the complete
conditional statement.

The execution of a conditional statement may be described as follows:
The Boolean expression of the if clause are evaluated one after the
other in sequence from left to right until one yielding the value
\keyw{true} is found.  Then the unconditional statement following this
Boolean is executed.  Unless this statement defines its successor
explicitly the next statement to be executed will be S4, i.e.  the
statement following the complete complete conditional statement.  Thus
the effect of the delimiter \keyw{else} may be described by saying
that it defines the successor of the statement it follows to be the
statement following the complete conditional statement.

The construction

\begin{flushleft}
\texttt{\keyw{else} <unconditional statement>}
\end{flushleft}

is equivalent to

\begin{flushleft}
\texttt{\keyw{else} \keyw{if} \keyw{true} \keyw{then} <unconditional statement>}
\end{flushleft}

If none of the Boolean expressions of the if clauses is true, the
effect of the whole conditional statement will be equivalent to that
of a dummy statement.

For further explanation the following picture may be useful:

\vspace{1em}
\setlength{\unitlength}{1cm}
\begin{picture}(15,1.9)(0,-0.05)
  \put (0.00,1.00){\texttt{\keyw{if} B1 \keyw{then} S1 \keyw{else} \keyw{if} B2 \keyw{then} S2 \keyw{else} S3; S4}}

  \put (0.60,0.80){\vector(+0,-1){0.40}}
  \put (3.75,0.40){\vector(+0,+1){0.40}}
  \put (0.60,0.40){\line  (+1,+0){3.15}}
  \put (1.50,0.00){\texttt{B1 \keyw{false}}}

  \put (4.30,0.80){\vector(+0,-1){0.40}}
  \put (7.45,0.40){\vector(+0,+1){0.40}}
  \put (4.30,0.40){\line  (+1,+0){3.15}}
  \put (5.20,0.00){\texttt{B2 \keyw{false}}}

  \put (3.10,1.40){\vector(+0,+1){0.40}}
  \put (6.70,1.40){\vector(+0,+1){0.40}}
  \put (8.30,1.80){\vector(+0,-1){0.40}}
  \put (3.10,1.80){\line  (+1,+0){5.20}}
\end{picture}

\paragraph{Go to into a conditional statement}

The effect of a go to statement leading into a conditional statement
follows directly from the above explanation of the effect of
\keyw{else}.




\subsection{For statements}
\label{LblForStatements}

\subsubsection{Syntax}
\label{LblForStatementsSyntax}

\begin{flushleft}
\vspace{0.2em}\texttt{<for list element> ::= <arithmetic expression> | \\
  \hspace{1.0cm}<arithmetic expression> \keyw{step} <arithmetic expression> \\
  \hspace{1.0cm}\keyw{until} <arithmetic expression> |\\
  \hspace{1.0cm}<arithmetic expression> \keyw{while} <Boolean expression>}\\

\vspace{0.2em}\texttt{<for list> ::= <for list element> |\\
  \hspace{1.0cm}<for list> , <for list element>}\\

\vspace{0.2em}\texttt{<for clause> ::= \keyw{for} <variable> := <for list> \keyw{do}}\\

\vspace{0.2em}\texttt{<for statement> ::= <for clause> <statement> |\\
  \hspace{1.0cm}<label>: <for statement>}\\
\end{flushleft}


\subsubsection{Examples}

\begin{flushleft}
\texttt{\keyw{for} q:=1 \keyw{step} s \keyw{until} n \keyw{do} A[q]:=B[q]\\
\keyw{for} k:=1,V1\mtim{}2 \keyw{while} V1\mlt{}N \keyw{do}\\
\keyw{for} j:=I+G,L,1 \keyw{step} 1 \keyw{until} N, C+D \keyw{do} A[k,j]:=B[k,j]}\\
\end{flushleft}


\subsubsection{Semantics}
\label{LblForStatementsSemantics}

A for clause causes the statement S which it precedes to be repeatedly
executed zero or more times.  In addition it performs a sequence of
assignments to its controlled variable.  The process may be visualized
by means of the following picture:

\vspace{1em}
\setlength{\unitlength}{1cm}
\begin{picture}(15,1.5)(0,-0.05)
  \put (0.00,0.60){\texttt{Initialize; test; statement S; advance; successor}}

  \put (2.90,0.40){\vector(+0,-1){0.40}}
  \put (9.00,0.00){\vector(+0,+1){0.40}}
  \put (2.90,0.00){\line  (+1,+0){6.10}}

  \put (2.90,1.40){\vector(+0,-1){0.40}}
  \put (6.90,1.00){\vector(+0,+1){0.40}}
  \put (2.90,1.40){\line  (+1,+0){4.00}}
\end{picture}

In this picture the word initialize means: perform the first
assignment of the for clause.  Advance means: perform the next
assignment of the for clause.  Test determines if the last assignment
has been done.  If so, the execution continues with the successor of
the for statement.  If not, the statement following the for clause is
executed.


\subsubsection{The for list elements}
\label{LblForStatementsElements}

The for list gives a rule for obtaining the values which are
consecutively assigned to the controlled variable.  This sequence of
values is obtained from the for list elements by taking these one by
one in order in which they are written.  The sequence of values
generated by each of the three species of for list elements and the
corresponding execution of the statement S are given by the following
rules:

\paragraph{Arithmetic expression}
\label{LblForStatementsSemanticsArithExpression}

This element gives rise to one value, namely the value of the given
arithmetic expression as calculated immediately before the
corresponding execution of the statement S.

\paragraph{Step-until-element}
\label{LblForStatementsSemanticsStepUntil}

An element of the form A \keyw{step} B \keyw{until} C, where A, B, and
C are arithmetic expressions, gives rise to an execution which may be
described most concisely in terms of additional Algol statement as
follows:

\begin{flushleft}
\texttt{\hspace{1.0cm}V := A\\
L1:  \keyw{if} (V-C)\mtim{}sign(B) \mgt{} 0 \keyw{then} \keyw{goto} ``Element exhausted'';\\
\hspace{1.0cm}Statement S;\\
\hspace{1.0cm}V := V+B;\\
\hspace{1.0cm}\keyw{goto} L1;}\\
\end{flushleft}

where V is the controlled variable of the for clause and `Element
exhausted' points to the evaluation according to the next element in
the for list, or if the step-until-element is the last of the list, to
the next statement in the program.

\paragraph{While-element}
\label{LblForStatementsSemanticsWhileElement}

The execution governed by a for list element of the form E
\keyw{while} F, where E is an arithmetic and F a Boolean expression,
is most concisely described in terms of additional Algol statements as
follows:

\begin{flushleft}
\texttt{L3:  V := E\\
  \hspace{1.0cm}\keyw{if} \mlnot{} F \keyw{then} \keyw{goto} ``Element exhausted'';\\
  \hspace{1.0cm}Statement S;\\
  \hspace{1.0cm}\keyw{goto} L3;}\\
\end{flushleft}

where the notation is the same as in
\ref{LblForStatementsSemanticsStepUntil} above.


\subsubsection{The value of the controlled variable upon exit}

Upon exit out of the statement S (supposed to be compound) through a
go to statement the value of the controlled variable will be the same
as it was immediately preceding the execution of the go to statement.

If the exit is due to exhaustion of the for list, on the other hand,
the value of the controlled variable is undefined after the exit.

\subsubsection{Go to leading into a for statement.}

The effect of a go to statement, outside a for statement, which refers
to a label within the for statement, is undefined.




\subsection{Procedure statements}
\label{LblProcedureStatements}


\subsubsection{Syntax}
\label{LblProcedureStatementsSyntax}

\begin{flushleft}
\vspace{0.2em}\texttt{<actual parameter> ::= <string> | <expression> |\\
  \hspace{1.0cm}<array identifier> | <switch identifier> |\\
  \hspace{1.0cm}<procedure identifier>}\\

\vspace{0.2em}\texttt{<letter string> ::= <letter> | <letter string> <letter>}\\

\vspace{0.2em}\texttt{<parameter delimiter> ::= , | ) <letter string> : (}\\

\vspace{0.2em}\texttt{<actual parameter list> ::= <actual parameter> |\\
  \hspace{1.0cm}<actual parameter list> <parameter delimiter> <actual parameter>}\\

\vspace{0.2em}\texttt{<actual parameter part> ::= <empty> | ( <actual parameter list> )}\\

\vspace{0.2em}\texttt{<procedure statement> ::=\\
  \hspace{1.0cm}<procedure identifier> <actual parameter part>}\\
\end{flushleft}

\subsubsection{Examples}

\begin{flushleft}
\texttt{Spur (A) Order: (7) Result to: (V)\\
Transpose (W, v+1)\\
Absmax (A, N, M, Yy, I, K)\\
Innerproduct (A [t,P,u], B [P], 10, P, Y)}\\
\end{flushleft}

These examples correspond to examples given in section
\ref{LblProcedureDeclarationsExamples}.

\subsubsection{Semantics}
\label{LblProcedureStatementsSemantics}

A procedure statement serves to invoke (call for) the execution of a
procedure body (cf. section \ref{LblProcedureDeclarations} procedure
declarations).  Where the procedure body is a statement written in
Algol the effect of this execution will be equivalent to the effect of
performing the following operations on the program at the time of
execution of the procedure statement.

\paragraph{Value assignment (call by value)}
\label{LblProcedureStatementsSemanticsCallByValue}

All formal parameters quoted in the value part of the procedure
declaration heading are assigned the values (cf. section
\ref{LblValuesAndTypes} Values and types) of the corresponding actual
parameters, these assignments being considers as being performed
explicitly before entering the procedure body.  The effect is as
though an additional block embracing the procedure body were created
in which these assignments were made to variables local to this
fictitious block with types as given in the corresponding
specifications (cf. section
\ref{LblProcedureDeclarationsSpecifications}).  As a consequence,
variables called by value are to be considered as nonlocal to the body
of the procedure, but local to the fictitious block (cf. section
\ref{LblProcedureDeclarationsSemantics}).

\paragraph{Name replacement (call by name)}
\label{LblProcedureStatementsSemanticsCallByName}

Any formal parameter not quoted in the value list is replaced,
throughout the procedure body, by the corresponding actual parameter,
after enclosing this latter in parentheses wherever syntactically
possible.  Possible conflicts between identifiers inserted through
this process and other identifiers already present within the
procedure body will be avoided by suitable systematic changes of the
formal or local identifiers involved.

\paragraph{Body replacement and execution}

Finally the procedure body, modified as above, is inserted in place of
the procedure statement and executed.  if the procedure is called from
a place outside the scope of any non-local quantity of the procedure
bode the conflicts between the identifiers inserted through this
process of body replacement and the identifiers whose declarations are
valid at the place of the procedure statement or function designator
will be avoided through suitable systematic changes of the latter
identifiers.

\subsubsection{Actual-formal correspondence}

The correspondence between the actual parameters of the procedure
statement and the formal parameters of the procedure heading is
established as follows: The actual parameter list of the procedure
statement must have the same number of entries as the formal parameter
list of the procedure declaration heading.  The correspondence is
obtained by taking the entries of these two lists in the same order.

\subsubsection{Restrictions}
\label{LblProcedureStatementsRestrictions}

For a procedure statement to be defined it is evidently necessary that
the operations on the procedure body defined in sections
\ref{LblProcedureStatementsSemanticsCallByValue} and
\ref{LblProcedureStatementsSemanticsCallByName} lead to a correct
Algol statement.

This imposes the restriction on any procedure statement that the kind
and type of each actual parameter to be compatible with the kind and
type of the corresponding formal parameter.  Some important particular
cases of this general rule are the following:

\paragraph{} \label{LblProcedureStatementsRestrictionsP1}
If a string is supplied as an actual parameter in a procedure
statement or function designator, whose defining procedure body is an
Algol 60 statement (as opposed to non-Algol code, cf. section
\ref{LblProcedureBodyExpressedInCode}), then this string can only be
used within the procedure body as an actual parameter in further
procedure calls.  Ultimately it can only be used by a procedure body
expressed in non-Algol code.

\paragraph{} \label{LblProcedureStatementsRestrictionsP2}
A formal parameter which occurs as a left part variable in an
assignment statement within the procedure body and which is not called
by value can only correspond to an actual parameter which is a
variable (special case of expression).

\paragraph{} \label{LblProcedureStatementsRestrictionsP3}
A formal parameter which is used within the procedure body as an array
identifier can only correspond to an actual parameter which is an
array identifier of an array of the same dimensions.  In addition if
the formal parameter is called by value the local array created during
the call will have the same subscript bounds as the actual array.

\paragraph{} \label{LblProcedureStatementsRestrictionsP4}
A formal parameter which is called by value cannot in general
correspond to a switch identifier or a procedure identifier or a
string, because these latter do not possess values (the exception is
the procedure identifier of a procedure declaration which has an empty
formal parameter part (cf. section \ref{LblProcedureDeclarations}) and
which defines the value of a function designator (cf. section
\ref{LblValuesOfFuncDesignators}).  This procedure identifier is in
itself a complete expression).

\paragraph{} \label{LblProcedureStatementsRestrictionsP5}
Any formal parameter may have restrictions on the type of the
corresponding actual parameter associated with it (these restrictions
may, or may not, be given through specifications in the procedure
heading).  In the procedure statement such restrictions must evidently
be observed.

\subsubsection{}

Deleted

\subsubsection{Parameter delimiters}
\label{LblProcedureParameterDelimiters}

All parameter delimiters are understood to be equivalent.  No
correspondence between the parameter delimiters used in a procedure
statement and those used in the procedure heading is expected beyond
their number is the same.  Thus the information conveyed by using the
elaborate ones is entirely optional.

\subsubsection{Procedure body expressed in code}
\label{LblProcedureBodyExpressedInCode}

The restrictions imposed on a procedure statement calling a procedure
having its body expressed in non-Algol code evidently can only be
derived from the characteristics of the code used and the intent of
the user and thus fall outside the scope of the reference language.



\section{Declarations}
\label{LblDeclarations}

Declarations serve to define certain properties of the quantities used
in the program, and to associate them with identifiers.  A declaration
of an identifier is valid for one block.  Outside this block the
particular identifier may be used for other purposes (cf. section
\ref{LblCompoundStatementsSemantics}).

Dynamically this implies the following: at the time of an entry into a
block (through the \keyw{begin} since the labels inside are local and
therefore inaccessible from outside) all identifiers declared for the
block assume the significance implied by the nature of the
declarations given.  If these identifiers had already been defined by
other declarations outside they are for the time being given a new
significance.  Identifiers which are not declared for the block, on
the other hand, retain their old meaning.

At the time of an exit from an block (through \keyw{end}, or by a go
to statement) all identifiers which are declared for the block lose
their local significance.

A declaration my be marked with the additional declarator \keyw{own}.
This has the following effect: upon a reentry into the block, the
values of own quantities will be unchanged from their values at the
last exit, while the values of declared variables which are not marked
as own are undefined.  Apart from labels and formal parameters of
procedure declarations and with the possible exception of those for
standard functions (cf. sections \ref{LblBasicSymbolsIdentifiers} and
\ref{LblBasicSymbolsNumbers}) all identifiers of a program must be
declared.  No identifier may be declared more than once in any one
block head.

Syntax:

\begin{flushleft}
\vspace{0.2em}\texttt{<declaration> ::= <type declaration> | <array declaration> |\\
  \hspace{1.0cm}<switch declaration> | <procedure declaration>}\\
\end{flushleft}


\subsection{Type declarations}
\label{LblTypeDeclarations}

\subsubsection{Syntax}
\label{LblTypeDeclarationsSyntax}

\begin{flushleft}
\vspace{0.2em}\texttt{<type list> ::= <simple variable> | <simple variable> , <type list>}\\

\vspace{0.2em}\texttt{<type> ::= \keyw{real} | \keyw{integer} | \keyw{Boolean}}\\

\vspace{0.2em}\texttt{<local or own type> ::= <type> | \keyw{own} <type>}\\

\vspace{0.2em}\texttt{<type declaration> ::= <local or own type> <type list>}\\
\end{flushleft}

\subsubsection{Examples}

\begin{flushleft}
\texttt{\keyw{integer} p, q, s\\
\keyw{own} \keyw{Boolean} Acryl, n}\\
\end{flushleft}

\subsubsection{Semantics}
\label{LblTypeDeclarationsSemantics}

Type declarations serve to declare certain identifiers to represent
simple variables of a given type.  Real declared variables may only
assume positive or negative values including zero.  Integer declared
variables may only assume positive and negative integral values
including zero.  Boolean declared variables may only assume the values
\keyw{true} and \keyw{false}.

In arithmetic expressions any position which can be occupied by a real
declared variable may be occupied by an integer declared variable.

For the semantics of \keyw{own}, see the fourth paragraph of section 5
above.



\subsection{Array declarations}
\label{LblArrayDeclarations}

\subsubsection{Syntax}
\label{LblArrayDeclarationsSyntax}

\begin{flushleft}
\vspace{0.2em}\texttt{<lower bound> ::= <arithmetic expression>}\\

\vspace{0.2em}\texttt{<upper bound> ::= <arithmetic expression>}\\

\vspace{0.2em}\texttt{<bound pair> ::= <lower bound> : <upper bound>}\\

\vspace{0.2em}\texttt{<bound pair list> ::= <bound pair> |\\
  \hspace{1.0cm}<bound pair list> , <bound pair>}\\

\vspace{0.2em}\texttt{<array segment> ::= <array identifier> [ <bound pair list> ] |\\
  \hspace{1.0cm}<array identifier> , <array segment>}\\

\vspace{0.2em}\texttt{<array list> ::= <array segment> |\\
  \hspace{1.0cm}<array list> , <array segment>}\\

\vspace{0.2em}\texttt{<array declaration> ::= \keyw{array} <array list> |\\
  \hspace{1.0cm}<local or own type> \keyw{array} <array list>}\\
\end{flushleft}

\subsubsection{Examples}

\begin{flushleft}
\texttt{\keyw{array} a, b, c [7:n, 2:m], s [-2:10]\\
\keyw{own} \keyw{integer} \keyw{array} A [\keyw{if} c\mlt{}0 \keyw{then} 2 \keyw{else} 1:20]\\
\keyw{real} \keyw{array} q [-7:-1]}\\
\end{flushleft}

\subsubsection{Semantics}
\label{LblArrayDeclarationsSemantics}

An array declaration declares one or several identifiers to represent
multidimensional arrays of subscripted variables and gives the
dimensions of the arrays, the bound of the subscripts, and the types
of the variables.

\paragraph{Subscript bounds} \label{LblArrayDeclarationsSemanticsP1}
The subscript bounds for any array are given in the first subscript
bracket following the identifier of this array in the form of a bound
pair list.  Each item of this list gives the lower and upper bound of
a subscript in the form of two arithmetic expressions separated by the
delimiter :.  The bound pair list gives the bounds of all subscripts
taken in order from left to right.

\paragraph{Dimensions} \label{LblArrayDeclarationsSemanticsP2}

The dimensions are given as the number of entries in the bound pair
list.

\paragraph{Types} \label{LblArrayDeclarationsSemanticsP3}

All arrays declared in one declaration are of the same quoted type.
If no type declarator is given the type \keyw{real} is understood.


\subsubsection{Lower upper bound expressions}
\label{LblArrayDeclarationsLowerUpperExpressions}

\paragraph{} \label{LblArrayDeclarationsLowerUpperExpressionsP1}
The expressions will be evaluated in the same way as subscript
expressions (cf. section \ref{LblVariableSubscriptsP2}).

\paragraph{} \label{LblArrayDeclarationsLowerUpperExpressionsP2}
The expressions can only depend on variables and procedures which are
non-local to the block for which the array declaration is valid.
Consequently in the outermost block of a program only array
declarations with constant bounds may be declared.

\paragraph{} \label{LblArrayDeclarationsLowerUpperExpressionsP3}
An array identifier id defined only when the values of all upper
subscript bounds are not smaller than those of the corresponding lower
bounds.

\paragraph{} \label{LblArrayDeclarationsLowerUpperExpressionsP4}
The expressions will by evaluated once at each entrace into the block.

\subsubsection{The identity of subscripted variables}
\label{LblArrayDeclarationsIdentityOfSubscriptedVariables}

The identity of a subscripted variable is not related to the subscript
bounds given in the array declaration.  However, even if an array is
declared \keyw{own} the values of the corresponding subscripted
variables will, at any time, be defined only for those of these
variables which have subscripts within the most recently calculated
subscript bounds.


\subsection{Switch declarations}
\label{LblSwitchDeclarations}


\subsubsection{Syntax}
\label{LblSwitchDeclarationsSyntax}

\begin{flushleft}
\vspace{0.2em}\texttt{<switch list> ::= <designational expression> |\\
  \hspace{1.0cm}<switch list> , <designational expression>}\\

\vspace{0.2em}\texttt{<switch declaration> ::=\\
  \hspace{1.0cm}\keyw{switch} <switch identifier> := <switch list>}\\
\end{flushleft}


\subsubsection{Examples}

\begin{flushleft}
\texttt{\keyw{switch} S:=S1,S2,Q[m], \keyw{if} v\mgt{}-5 \keyw{then} S3 \keyw{else} S4\\
\keyw{switch} Q:=p1,w}\\
\end{flushleft}

\subsubsection{Semantics}
\label{LblSwitchDeclarationsSemantics}

A switch declaration defines the set of values of the corresponding
switch designators.  These values are given one by one as the values
of the designational expressions entered in the switch list.  With
each of these designational expressions there is associated a positive
integer, 1, 2, ..., obtained by counting the items in the list from
left to right.  The value of the switch designator corresponding to a
given value of the subscript expression (cf. section
\ref{LblDesignationalExpressions} Designational expressions) is the
value of the designational expression in the switch list having this
given value as its associated integer.

\subsubsection{Evaluation of expressions in the switch list}

An expression in the switch list will be evaluated every time the item
of the list in which the expression occurs is referred to, using the
current values of all variables involved.

\subsubsection{Influence of scopes}

If a switch designator occurs outside the scope of a quantity entering
into a designational expression in the switch list, and an evaluation
of this switch designator selects this designational expression, then
the conflicts between the identifiers for the quantities in this
expression and the identifiers whose declarations are valid at the
place of the switch designator will be avoided through suitable
systematic changes of the latter identifiers.



\subsection{Procedure declarations}
\label{LblProcedureDeclarations}

\subsubsection{Syntax}
\label{LblProcedureDeclarationsSyntax}

\begin{flushleft}
\vspace{0.2em}\texttt{<formal parameter> ::= <identifier>}\\

\vspace{0.2em}\texttt{<formal parameter list> ::= <formal parameter> |\\
  \hspace{1.0cm}<formal parameter list> <parameter delimiter> <formal parameter>}\\

\vspace{0.2em}\texttt{<formal parameter part> ::= <empty> | ( <formal parameter list> )}\\

\vspace{0.2em}\texttt{<identifier list> ::= <identifier> |\\
  \hspace{1.0cm}<identifier list> , <identifier>}\\

\vspace{0.2em}\texttt{<value part> ::= \keyw{value} <identifier list> ; | <empty>}\\

\vspace{0.2em}\texttt{<specifier> ::= \keyw{string} | <type> | \keyw{array} |\\
  \hspace{1.0cm}<type> \keyw{array} | \keyw{label} | \keyw{switch} |\\
  \hspace{1.0cm}\keyw{procedure} | <type> \keyw{procedure}}\\

\vspace{0.2em}\texttt{<specification part> ::= <empty> | <specifier> <identifier list> ; |\\
  \hspace{1.0cm}<specification part> <specifier> <identifier list>}\\

\vspace{0.2em}\texttt{<procedure heading> ::= <procedure identifier> <formal parameter part> ;\\
  \hspace{1.0cm}<value part> <specification part>}\\

\vspace{0.2em}\texttt{<procedure body> ::= <statement> | <code>}\\

\vspace{0.2em}\texttt{<procedure declaration> ::=\\
  \hspace{1.0cm}\keyw{procedure} <procedure heading> <procedure body> |\\
  \hspace{1.0cm}<type> \keyw{procedure} <procedure heading> <procedure body>}\\

\end{flushleft}


\subsubsection{Examples}
\label{LblProcedureDeclarationsExamples}

(See also the examples at the end of the report.)

\begin{flushleft}
\texttt{\keyw{procedure} Spur (a) Order: (n); \keyw{value} n;\\
\keyw{array} a; \keyw{integer} n; \keyw{real} s;\\
\keyw{begin} \keyw{integer} k;\\
\hspace{1.25cm}s:=0;\\
\hspace{1.25cm}\keyw{for} k:=1 \keyw{step} 1 \keyw{until} n \keyw{do} s:=s+a[k,k]\\
\keyw{end}\\
}\end{flushleft}

\begin{flushleft}
\texttt{\keyw{procedure} Transpose (a) Order: (n); \keyw{value} n;\\
\keyw{array} a; \keyw{integer} n;\\
\keyw{begin} \keyw{real} w; \keyw{integer} i, k;\\
\hspace{1.25cm}\keyw{for} i := 1 \keyw{step} 1 \keyw{until} n \keyw{do}\\
\hspace{1.25cm}\keyw{for} k := 1+i \keyw{step} 1 \keyw{until} n \keyw{do}\\
\hspace{1.25cm}\keyw{begin} w:=a[i,k];\\
\hspace{2.50cm}a[i,k]:=a[k,i];\\
\hspace{2.50cm}a[k,i]:=w\\
\hspace{1.25cm}\keyw{end}\\
\keyw{end} Transpose\\
}\end{flushleft}

\begin{flushleft}
\texttt{\keyw{integer} \keyw{procedure} Step (u); \keyw{real} u;\\
Step:=\keyw{if} 0\mle{}u\mland{}u\mle{}1 \keyw{then} 1 \keyw{else} 0\\
}\end{flushleft}

\begin{flushleft}
\texttt{\keyw{procedure} Absmax (a) Size: (n, m) Result: (y) Subscripts: (i, k);\\
\keyw{comment} The absolute greatest element of the matrix a, of size n by m\\
is transferred to y, and the subscripts of this element to i and k;\\
\keyw{array} a; \keyw{integer} n, m, i, k; \keyw{real} y;\\
\keyw{begin} \keyw{integer} p, q;\\
\hspace{1.25cm}y := 0;\\
\hspace{1.25cm}\keyw{for} p:=1 \keyw{step} 1 \keyw{until} n \keyw{do} \keyw{for} q:=1 \keyw{step} 1 \keyw{until} m \keyw{do}\\
\hspace{1.25cm}\keyw{if} abs(a[p,q])\mgt{}y \keyw{then} \keyw{begin} y:=abs(a[p,q]);\\
\hspace{6.75cm}    i:=p; k:=q \keyw{end} \keyw{end} Absmax\\
}\end{flushleft}

\begin{flushleft}
\texttt{\keyw{procedure} Innerproduct (a, b) Order: (k, p) Result: (y); \keyw{value} k;\\
\keyw{integer} k, p; \keyw{real} y, a, b;\\
s:=0;\\
\keyw{for} p:=1 \keyw{step} 1 \keyw{until} k \keyw{do} s:=s+a\mtim{}b;\\
y:=s\\
\keyw{end} Innerproduct
}\end{flushleft}


\subsubsection{Semantics}
\label{LblProcedureDeclarationsSemantics}

A procedure declaration serves to define the procedure associated with
a procedure identifier.  The principal constituent of a procedure
declaration is a statement or a piece of code, the procedure body,
which through the use of procedure statements and/or function
designators may be activated from other parts of the block in the head
of which the procedure declaration appears.  Associated with the body
is a heading, which specifies certain identifiers occurring within the
body to represent formal parameters.  Formal parameters in the
procedure body will, whenever the procedure is activated (cf. section
\ref{LblFunctionDesignators} Function designators and section
\ref{LblProcedureStatements} Procedure statements) be assigned the
values of or replaced by actual parameters.  Identifiers in the
procedure body which are not formal will be either local or non-local
to the body depending on whether they are declared within the body or
not.  Those of them which are non-local to the body may well be local
to the block in the head of which the procedure declaration appears.
The procedure body always acts like a block, whether it has the form
of one or not.  Consequently the scope of any label labelling a
statement within the body or the body itself can never extended beyond
the procedure body.  In addition, if the identifier of a formal
parameter is declared anew within the procedure body (including the
case of its use as a label in section
\ref{LblCompoundStatementsSemantics}), it is thereby given a local
significance and actual parameters which correspond to it are
inaccessible throughout the scope of its inner local quantity.

\subsubsection{Values of function designators}
\label{LblValuesOfFuncDesignators}

For a procedure declaration to define the value of a function
designator there must, within the procedure declaration body, occur
one or more explicit assignment statements with the procedure
identifier in a left part; at least one of these must be executed, and
the type associated with the procedure identifier must be declared
through the appearance of a type declarator as the very first symbol
of the procedure declaration.  The last value so assigned is used to
continue the evaluation of the expression in which the function
designator occurs.  Any occurrence of the procedure identifier within
the body of the procedure other than in a left part in an assignment
statement denotes activation of the procedure.

\subsubsection{Specifications}
\label{LblProcedureDeclarationsSpecifications}

In the heading a specification part, giving information about the
kinds and types of the formal parameters by means of an obvious
notation, may be included.  In this part no formal parameter may occur
more than once.  Specification of formal parameters called by value
(cf. section \ref{LblProcedureStatementsSemanticsCallByValue}) must be
supplied and specifications of formal parameters called by name (cf.
section \ref{LblProcedureStatementsSemanticsCallByName}) may be
omitted.

\subsubsection{Code as procedure body}
\label{LblProcedureDeclarationsCodeAsProcedureBody}

It is understood that the procedure body may be expressed in non-Algol
language.  Since it is intended that the use of this feature should be
entirely a question of hardware representation, no further rules
concerning this code language can be given within the reference
language.



\section{Examples of procedure declarations}

\subsection{Example 1}

\begin{flushleft}
\texttt{\keyw{procedure} euler (fct,sum,eps,tim); \keyw{value} eps,tim; \keyw{integer} tim;\\
\keyw{real} \keyw{procedure} fct; \keyw{real} sum,eps;\\
\keyw{comment} euler computes the sum of fct(i) for i from zero up to\\
infinity by means of a suitably refined euler transformation. The\\
summation is stopped as soon as tim times in succession the absolute\\
value of the terms of the transformed series are found to be less than\\
eps. Hence, one should provide a function fct with one integer argument,\\
an upper bound eps, and an integer tim. The output is the sum sum. euler\\
is particularly efficient in the case of a slowly convergent or\\
divergent alternating series;\\
\keyw{begin} \keyw{integer} i,k,n,t; \keyw{array} m[0:15]; \keyw{real} mn,mp,ds;\\
i:=n; t:=0; m[0]:=fct(0); sum:=m[0]/2;\\
nextterm: i:=i+1; mn:=fct(i);\\
\hspace{1.0cm}\keyw{for} k:=0 \keyw{step} 1 \keyw{until} n \keyw{do}\\
\hspace{1.0cm}\hspace{1.0cm}\hspace{1.0cm}\keyw{begin} mp:=(mn+m[k])/2; m[k]:=mn;\\
\hspace{1.0cm}\hspace{1.0cm}\hspace{1.0cm}\hspace{1.0cm}mn:=mp \keyw{end} means;\\
\hspace{1.0cm}\hspace{1.0cm}\keyw{if} (abs(mn)\mlt{}abs(m[n])) \mland{} (n\mlt{}15) \keyw{then}\\
\hspace{1.0cm}\hspace{1.0cm}\hspace{1.0cm}\keyw{begin} ds:=mn/2; n:=n+1; m[n]:=mn \keyw{end} accept\\
\hspace{1.0cm}\hspace{1.0cm}\keyw{else} ds:=mn;\\
\hspace{1.0cm}\hspace{1.0cm}sum:=sum+ds;\\
\hspace{1.0cm}\hspace{1.0cm}\keyw{if} abs(ds)\mlt{}eps \keyw{then} t:=t+1 \keyw{else} t:=0;\\
\hspace{1.0cm}\hspace{1.0cm}\keyw{if} t\mlt{}tim \keyw{then} \keyw{goto} nextterm\\
\keyw{end} euler\\
}\end{flushleft}



\subsection[Example 2]{Example 2\footnotemark}

\footnotetext{This RK-program contains some new ideas which are
related to ideas of S.  Gill, A process for the step by step
integration of differential equations in an automatic computing
machine.  Proc.  Camb.  Phil.  Soc.  47 (1951) p.  96, and E.
Fr\"oberg, On the solution of ordinary differential equations with
digital computing machines, Fysiograf.  S\ae{}llsk.  Lund, F\"orhd.
20 Nr.  11 (1950) p. 136-152.  It must be clear however that with
respect to computing time and round-off errors it may not be optimal,
nor has it actually been tested on a computer.}

\begin{flushleft}
{\tt \keyw{procedure} RK (x,y,n,FKT,eps,eta,xE,yE,fi); \keyw{value} x,y; \keyw{integer} n;\\
\keyw{Boolean} fi; \keyw{real} x,eps,eta,xE; \keyw{array} y,yE; \keyw{procedure} FKT;\\
\keyw{comment} RK integrates the system y'k=fk(x,y1,y2,...,yn)(k=1,2,...n)\\
of differential equations with the method of Runge-Kutta with automatic\\
search for appropriate length of integration step. Parameters are: The\\
initial values x and y[k] for x and the unknown functions yk(x). The\\
order n of the system. The procedure FKT(x,y,n,z) which represents the\\
system to be integrated, i.e. the set of functions fk. The tolerance values\\
eps and eta which govern the accuracy of the numerical integration. The end\\
of the integration interval xE; The output parameter yE which represents\\
the solution x=xE. The Boolean variable fi, which must always be given \\
the value \keyw{true} for an isolated or first entry into RK. If however the\\
functions y must be available at several meshpoints x0,x1,...,xn, then the\\
procedure must be called repeatedly (with x=xk, xE=x(k+1), for k=0,1,...,n-1)\\
and then the later calls may occur with fi=\keyw{false} which saves computing\\
time. The input parameters of FKT must be x,y,z,n, the output parameter z\\
represents the set of derivatives z[k]=fk(x,y[1],y[2],...,y[n]) for x and\\
the actual y's. A procedure comp enters as a non-local identifier;\\
\vspace{1.0em}
\keyw{begin}\\
\hspace{1.0cm}\keyw{array} z,y1,y2,y3[1:n]; \keyw{real} x1,x2,x3,H; \keyw{Boolean} out;\\
\hspace{1.0cm}\keyw{integer} k,j; \keyw{own} \keyw{real} s,Hs;\\
\hspace{1.0cm}\keyw{procedure} RK1ST (x,y,h,xe,ye); \keyw{real} x,h,xe; \keyw{array} y,ye;\\
\hspace{1.0cm}\hspace{1.0cm}\keyw{comment} RK1ST integrates one single Runge-Kutta step with\\
\hspace{1.0cm}\hspace{1.0cm}initial values x, y[k] which yields the output parameters xe=x+h\\
\hspace{1.0cm}\hspace{1.0cm}and ye[k], the latter being the solution at xe.  Important: the\\
\hspace{1.0cm}\hspace{1.0cm}parameters n, FKT, z enter RK1ST as nonlocal entities;\\
\hspace{1.0cm}\hspace{1.0cm}\keyw{begin}\\
\hspace{1.0cm}\hspace{1.0cm}\hspace{1.0cm}\keyw{array} w[1:n], a[1:5]; \keyw{integer} k,j;\\
\hspace{1.0cm}\hspace{1.0cm}\hspace{1.0cm}a[1]:=a[2]:=a[5]:=h/2; a[3]:=a[4]:=h;\\
\hspace{1.0cm}\hspace{1.0cm}\hspace{1.0cm}xe:=x;\\
\hspace{1.0cm}\hspace{1.0cm}\hspace{1.0cm}\keyw{for} k:=1 \keyw{step} 1 \keyw{until} n \keyw{do} ye[k]:=w[k]:=y[k];\\
\hspace{1.0cm}\hspace{1.0cm}\hspace{1.0cm}\keyw{for} j:=1 \keyw{step} 1 \keyw{until} 4 \keyw{do}\\
\hspace{1.0cm}\hspace{1.0cm}\hspace{1.0cm}\hspace{1.0cm}\keyw{begin}\\
\hspace{1.0cm}\hspace{1.0cm}\hspace{1.0cm}\hspace{1.0cm}\hspace{1.0cm}FKT(xe,w,n,z);\\
\hspace{1.0cm}\hspace{1.0cm}\hspace{1.0cm}\hspace{1.0cm}\hspace{1.0cm}xe:=x+a[j];\\
\hspace{1.0cm}\hspace{1.0cm}\hspace{1.0cm}\hspace{1.0cm}\hspace{1.0cm}\keyw{for} k:=1 \keyw{step} 1 \keyw{until} n \keyw{do}\\
\hspace{1.0cm}\hspace{1.0cm}\hspace{1.0cm}\hspace{1.0cm}\hspace{1.0cm}\keyw{begin}\\
\hspace{1.0cm}\hspace{1.0cm}\hspace{1.0cm}\hspace{1.0cm}\hspace{1.0cm}\hspace{1.0cm}w[k]:=y[k]+a[j]\mtim{}z[k];\\
\hspace{1.0cm}\hspace{1.0cm}\hspace{1.0cm}\hspace{1.0cm}\hspace{1.0cm}\hspace{1.0cm}ye[k]:=ye[k]+a[j+1]\mtim{}z[k]/3\\
\hspace{1.0cm}\hspace{1.0cm}\hspace{1.0cm}\hspace{1.0cm}\hspace{1.0cm}\keyw{end} k\\
\hspace{1.0cm}\hspace{1.0cm}\hspace{1.0cm}\hspace{1.0cm}\keyw{end} j\\
\hspace{1.0cm}\hspace{1.0cm}\keyw{end} RK1ST;\\
\vspace{1.0em}
Begin of program:\\
\vspace{1.0em}
\hspace{1.0cm}\keyw{if} fi \keyw{then} \keyw{begin} H:=xE-x; s:=0 \keyw{end} \keyw{else} H:=Hs;\\
\hspace{1.0cm}out:=\keyw{false};\\
\vspace{1.0em}
AA: \keyw{if} (x+2.01\mtim{}H-xE)\mgt{}0) \mleqv{} (H\mgt{}0) \keyw{then}\\
\hspace{1.0cm}\keyw{begin} Hs:=H; out:=\keyw{true}; H:=(xE-x)/2 \keyw{end} if;\\
\hspace{1.0cm}RK1ST (x,y,2\mtim{}H,x1,y1);\\
\vspace{1.0em}
BB: RK1ST (x,y,H,x2,y2); RK1ST (x2,y2,H,x3,y3);\\
\hspace{1.0cm}\keyw{for} k:=1 \keyw{step} 1 \keyw{until} n \keyw{do}\\
\hspace{1.0cm}\hspace{1.0cm}\keyw{if} comp (y1[k],y3[k],eta)\mgt{}eps \keyw{then} \keyw{goto} CC;\\
\hspace{1.0cm}\keyw{comment} comp(a,b,c) is a function designator, the value of\\
\hspace{1.0cm}which is the absolute value of the difference of the mantissae of a\\
\hspace{1.0cm}and b, after the exponents of these quantities have been made equal\\
\hspace{1.0cm}to the largest of the exponents of the originally given parameters\\
\hspace{1.0cm}a, b, c;\\
\hspace{1.0cm}x:=x3; \keyw{if} out \keyw{then} \keyw{goto} DD;\\
\hspace{1.0cm}\keyw{for} k:=1 \keyw{step} 1 \keyw{until} n \keyw{do} y[k]:=y3[k];\\
\hspace{1.0cm}\keyw{if} s=5 \keyw{then} \keyw{begin} s:=0; H:=2\mtim{}H \keyw{end} if;\\
\hspace{1.0cm}s:=s+1; \keyw{goto} AA;\\
\vspace{1.0em}
CC: H:=0.5\mtim{}X; out:=\keyw{false}; x1:=x2;\\
\hspace{1.0cm}\keyw{for} k:=1 \keyw{step} 1 \keyw{until} n \keyw{do} y1[k]:=y2[k];\\
\hspace{1.0cm}\keyw{goto} BB;\\
\vspace{1.0em}
DD: \keyw{for} k:=1 \keyw{step} 1 \keyw{until} n \keyw{do} yE[k]:=y3[k]\\
\keyw{end} RK\\
}\end{flushleft}








\section{Alphabetic index of definitions of concepts and syntactic units}

All references are given through section numbers.  The references are
given in three groups:

\begin{description}

\item[def] Following the abbreviation ``def'', reference to the
             syntactic definition (if any) is given.

\item[synt] Following the abbreviation ``synt'', references to the
             occurrences in metalinguistic formulae are given.
             References already quoted in the def-group are not
             repeated.

\item[text] Following the word ``text'', the references to definitions
             given in the text are given. 

\end{description}

The basic symbols represented by signs other than underlined words
have been collected at the beginning.  The examples have been ignored
in compiling the index.


+ see: plus

- see: minus

\mtim{} see: multiply

/ \mdiv{} see: divide

\mpow{} see: exponentiation

\mlt{} \mle{} \meq{} \mge{} \mgt{} \mne{} see: \texttt{<relational operator>}

\mleqv{} \mlimp{} \mlor{} \mland{} \mlnot{} see: \texttt{<logical operator>}

, see: comma

. see: decimal point

\ten{} see: ten

: see: colon

; see: semicolon

:= see: colon equal

\blankchar{} see: space

( ) see: parentheses

[ ] see: subscript bracket

` ' see: string quote


\texttt{<actual parameter>},
def
\ref{LblFunctionDesignatorsSyntax},
\ref{LblProcedureStatementsSyntax}\*

\texttt{<actual parameter list>},
def
\ref{LblFunctionDesignatorsSyntax},
\ref{LblProcedureStatementsSyntax}\*

\texttt{<actual parameter part>},
def
\ref{LblFunctionDesignatorsSyntax},
\ref{LblProcedureStatementsSyntax}\*

\texttt{<adding operator>},
def
\ref{LblArithmeticExpressionsSyntax}\*

alphabet,
text
\ref{LblBasicSymbolsLetters}\*

arithmetic,
text
\ref{LblArithmeticExpressionsRealQuantities}\*

\texttt{<arithmetic expression>},
def
\ref{LblArithmeticExpressionsSyntax},
synt
\ref{LblExpressions},
\ref{LblVariablesSyntax},
\ref{LblBooleanExpressionsSyntax},
\ref{LblAssignmentStatementSyntax},
\ref{LblForStatementsSyntax},
\ref{LblArrayDeclarationsSyntax}
text
\ref{LblArithmeticExpressionsSemantics}\*

\texttt{<arithmetic operator>},
def
\ref{LblBasicSymbolsDelimiters}
text
\ref{LblOperatorsAndTypes}\*

\keyw{array},
synt
\ref{LblBasicSymbolsDelimiters},
\ref{LblArrayDeclarationsSyntax},
\ref{LblProcedureDeclarationsSyntax}\*

array,
text
\ref{LblVariableSubscriptsP1}\*

\texttt{<array declaration>},
def
\ref{LblArrayDeclarationsSyntax}
synt
\ref{LblDeclarations}
text
\ref{LblArrayDeclarationsSemantics}\*

\texttt{<array identifier>},
def
\ref{LblVariablesSyntax}
synt
\ref{LblFunctionDesignatorsSyntax},
\ref{LblProcedureStatementsSyntax},
\ref{LblArrayDeclarationsSyntax}
text
\ref{LblValuesAndTypes}\*

\texttt{<array list>},
def
\ref{LblArrayDeclarationsSyntax}\*

\texttt{<array segment>},
def
\ref{LblArrayDeclarationsSyntax}\*

\texttt{<assignment statement>},
def
\ref{LblAssignmentStatementSyntax}
synt
\ref{LblCompoundStatementsSyntax}
text
\ref{LblStructureOfTheLanguage},
\ref{LblAssignmentStatementSemantics}\*

\texttt{<basic statement>},
def
\ref{LblCompoundStatementsSyntax}
synt
\ref{LblConditionalStatementsSyntax}\*

\texttt{<basic symbol>},
def
\ref{LblBasicSymbols}\*

\keyw{begin},
synt
\ref{LblBasicSymbolsDelimiters},
\ref{LblCompoundStatementsSyntax}\*

\texttt{<block>},
def
\ref{LblCompoundStatementsSyntax}
synt
\ref{LblConditionalStatementsSyntax}
text
\ref{LblStructureOfTheLanguage},
\ref{LblCompoundStatementsSemantics},
\ref{LblDeclarations}\*

\texttt{<block head>},
def
\ref{LblCompoundStatementsSyntax}\*

\keyw{Boolean},
synt
\ref{LblBasicSymbolsDelimiters},
\ref{LblTypeDeclarationsSyntax}
text
\ref{LblTypeDeclarationsSemantics}\*

\texttt{<Boolean expression>},
def
\ref{LblBooleanExpressionsSyntax}
synt
\ref{LblExpressions},
\ref{LblArithmeticExpressionsSyntax},
\ref{LblAssignmentStatementSyntax},
\ref{LblConditionalStatementsSyntax},
\ref{LblForStatementsSyntax}
text
\ref{LblBooleanExpressionsSemantics}\*

\texttt{<Boolean factor>},
def
\ref{LblBooleanExpressionsSyntax}\*

\texttt{<Boolean primary>},
def
\ref{LblBooleanExpressionsSyntax}\*

\texttt{<Boolean secondary>},
def
\ref{LblBooleanExpressionsSyntax}\*

\texttt{<Boolean term>},
def
\ref{LblBooleanExpressionsSyntax}\*

\texttt{<bound pair>},
def
\ref{LblArrayDeclarationsSyntax}\*

\texttt{<bound pair list>},
def
\ref{LblArrayDeclarationsSyntax}\*

\texttt{<bracket>},
def
\ref{LblBasicSymbolsDelimiters}\*

\texttt{<code>},
synt
\ref{LblProcedureDeclarationsSyntax}
text
\ref{LblProcedureBodyExpressedInCode},
\ref{LblProcedureDeclarationsCodeAsProcedureBody}\*

colon:,
synt
\ref{LblBasicSymbolsDelimiters},
\ref{LblFunctionDesignatorsSyntax},
\ref{LblCompoundStatementsSyntax},
\ref{LblConditionalStatementsSyntax},
\ref{LblForStatementsSyntax},
\ref{LblProcedureStatementsSyntax},
\ref{LblArrayDeclarationsSyntax}\*

colon equal :=,
synt
\ref{LblBasicSymbolsDelimiters},
\ref{LblAssignmentStatementSyntax},
\ref{LblForStatementsSyntax},
\ref{LblSwitchDeclarationsSyntax}\*

comma , ,
synt
\ref{LblBasicSymbolsDelimiters},
\ref{LblVariablesSyntax},
\ref{LblFunctionDesignatorsSyntax},
\ref{LblForStatementsSyntax},
\ref{LblProcedureStatementsSyntax},
\ref{LblTypeDeclarationsSyntax},
\ref{LblArrayDeclarationsSyntax},
\ref{LblSwitchDeclarationsSyntax},
\ref{LblProcedureDeclarationsSyntax}\*

\keyw{comment},
synt
\ref{LblBasicSymbolsDelimiters}\*

comment convention,
\ref{LblBasicSymbolsDelimiters}\*

\texttt{<compound statement>},
def
\ref{LblCompoundStatementsSyntax}
synt
\ref{LblConditionalStatementsSyntax}
text
\ref{LblStructureOfTheLanguage}\*

\texttt{<compound tail>},
def
\ref{LblCompoundStatementsSyntax}\*

\texttt{<conditional statement>},
def
\ref{LblConditionalStatementsSyntax}
synt
\ref{LblCompoundStatementsSyntax}
text
\ref{LblConditionalStatementsSemantics}\*

\texttt{<decimal fraction>},
def
\ref{LblBasicSymbolsNumbersSyntax}\*

\texttt{<decimal number>},
def
\ref{LblBasicSymbolsNumbersSyntax}
text
\ref{LblBasicSymbolsNumbersSemantics}\*

decimal point . ,
synt
\ref{LblBasicSymbolsDelimiters},
\ref{LblBasicSymbolsNumbersSemantics}\*

\texttt{<declaration>},
def
\ref{LblDeclarations}
synt
\ref{LblCompoundStatementsSyntax}
text
\ref{LblStructureOfTheLanguage},
\ref{LblDeclarations} (complete section)\*

\texttt{<declarator>},
def
\ref{LblBasicSymbolsDelimiters}\*

\texttt{<delimiter>},
def
\ref{LblBasicSymbolsDelimiters}
synt
\ref{LblBasicSymbols}

\texttt{<designational expression>},
def
\ref{LblDesignationalExpressionsSyntax}
synt
\ref{LblExpressions},
\ref{LblGotoStatementsSyntax},
\ref{LblSwitchDeclarationsSyntax}
text
\ref{LblDesignationalExpressionsSemantics}\*

\texttt{<digit>},
def
\ref{LblBasicSymbolsDigits}
synt
\ref{LblBasicSymbols},
\ref{LblBasicSymbolsIdentifiersSyntax},
\ref{LblBasicSymbolsNumbersSyntax}\*

dimension,
text
\ref{LblArrayDeclarationsSemanticsP2}\*

divide / \mdiv{},
synt
\ref{LblBasicSymbolsDelimiters},
\ref{LblArithmeticExpressionsSyntax}
text
\ref{LblOperatorsAndTypesP2}\*

\texttt{<dummy statement>},
def
\ref{LblDummyStatementsSyntax}
synt
\ref{LblCompoundStatementsSyntax}
text
\ref{LblCompoundStatementsSemantics}\*

\keyw{else},
synt
\ref{LblBasicSymbolsDelimiters},
\ref{LblArithmeticExpressionsSyntax},
\ref{LblBooleanExpressionsSyntax},
\ref{LblDesignationalExpressionsSyntax},
\ref{LblConditionalStatementsSyntax}
text
\ref{LblCompoundStatementsSemantics}\*

\texttt{<empty>},
def
\ref{FormalismForSyntacticDescription}
synt
\ref{LblBasicSymbolsStringsSyntax},
\ref{LblFunctionDesignatorsSyntax},
\ref{LblDummyStatementsSyntax},
\ref{LblProcedureStatementsSyntax},
\ref{LblProcedureDeclarationsSyntax}\*

\keyw{end},
synt
\ref{LblBasicSymbolsDelimiters},
\ref{LblCompoundStatementsSyntax}\*

entier,
text
\ref{LblTransferFunctions}\*

exponentiation \mpow{},
synt
\ref{LblBasicSymbolsDelimiters},
\ref{LblArithmeticExpressionsSyntax}
text
\ref{LblOperatorsAndTypesP3}\*

\texttt{<expression>},
def
\ref{LblExpressions}
synt
\ref{LblFunctionDesignatorsSyntax},
\ref{LblProcedureStatementsSyntax}
text
\ref{LblExpressions} (complete section)\*

\texttt{<exponential part>},
def
\ref{LblBasicSymbolsNumbersSyntax}
text
\ref{LblBasicSymbolsNumbersSemantics}\*

\texttt{<factor>},
def
\ref{LblArithmeticExpressionsSyntax}\*

\keyw{false},
synt
\ref{LblBasicSymbolsLogical}\*

\texttt{<for clause>},
def
\ref{LblForStatementsSyntax}
text
\ref{LblForStatementsSemantics}\*

\texttt{<for list>},
def
\ref{LblForStatementsSyntax}
text
\ref{LblForStatementsElements}\*

\texttt{<for list element>},
def
\ref{LblForStatementsSyntax}
text
\ref{LblForStatementsSemanticsArithExpression},
\ref{LblForStatementsSemanticsStepUntil},
\ref{LblForStatementsSemanticsWhileElement}\*

\texttt{<formal parameter>},
def
\ref{LblProcedureDeclarationsSyntax}
text
\ref{LblProcedureDeclarationsSemantics}\*

\texttt{<formal parameter list>},
def
\ref{LblProcedureDeclarationsSyntax}\*

\texttt{<formal parameter part>},
def
\ref{LblProcedureDeclarationsSyntax}\*

\texttt{<for statement>},
def
\ref{LblForStatementsSyntax}
synt
\ref{LblCompoundStatementsSyntax},
\ref{LblConditionalStatementsSyntax}
text
\ref{LblForStatements} (complete section)\*

\texttt{<function designator>},
def
\ref{LblFunctionDesignatorsSyntax}
synt
\ref{LblArithmeticExpressionsSyntax},
\ref{LblBooleanExpressionsSyntax}
text
\ref{LblFunctionDesignatorsSemantics},
\ref{LblValuesOfFuncDesignators}\*

\keyw{goto},
synt
\ref{LblBasicSymbolsDelimiters},
\ref{LblGotoStatementsSyntax}\*

\texttt{<go to statement>},
def
\ref{LblBooleanExpressionsSyntax}
synt
\ref{LblCompoundStatementsSyntax}
text
\ref{LblGotoStatementsSemantics}\*

\texttt{<identifier>},
def
\ref{LblBasicSymbolsIdentifiersSyntax}
synt
\ref{LblVariablesSyntax},
\ref{LblFunctionDesignatorsSyntax},
\ref{LblDesignationalExpressionsSyntax},
\ref{LblProcedureDeclarationsSyntax}
text
\ref{LblBasicSymbolsIdentifiersSemantics}\*

\texttt{<identifier list>},
def
\ref{LblProcedureDeclarationsSyntax}\*

\keyw{if},
synt
\ref{LblBasicSymbolsDelimiters},
\ref{LblArithmeticExpressionsSyntax},
\ref{LblConditionalStatementsSyntax}\*

\texttt{<if clause>},
def
\ref{LblArithmeticExpressionsSyntax},
\ref{LblConditionalStatementsSyntax}
synt
\ref{LblBooleanExpressionsSyntax},
\ref{LblDesignationalExpressionsSyntax}
text
\ref{LblArithmeticExpressionsSemantics},
\ref{LblConditionalStatementsSemanticsCondStatement}\*

\texttt{<if statement>},
def
\ref{LblConditionalStatementsSyntax}
text
\ref{LblConditionalStatementsSemanticsIfStatement}\*

\texttt{<implication>},
def
\ref{LblBooleanExpressionsSyntax}\*

\keyw{integer},
synt
\ref{LblBasicSymbolsDelimiters},
\ref{LblTypeDeclarationsSyntax}
text
\ref{LblTypeDeclarationsSemantics}\*

\texttt{<integer>},
def
\ref{LblBasicSymbolsNumbersSyntax}
text
\ref{LblBasicSymbolsNumbersTypes}\*

\keyw{label},
synt
\ref{LblBasicSymbolsDelimiters},
\ref{LblProcedureDeclarationsSyntax}\*

\texttt{<label>},
def
\ref{LblDesignationalExpressionsSyntax}
synt
\ref{LblCompoundStatementsSyntax},
\ref{LblConditionalStatementsSyntax},
\ref{LblForStatementsSyntax}
text
\ref{LblStructureOfTheLanguage},
\ref{LblCompoundStatementsSemantics}\*

\texttt{<left part>},
def
\ref{LblAssignmentStatementSyntax}\*

\texttt{<left part list>},
def
\ref{LblAssignmentStatementSyntax}\*

\texttt{<letter>},
def
\ref{LblBasicSymbolsLetters},
synt
\ref{LblBasicSymbols},
\ref{LblBasicSymbolsIdentifiersSyntax},
\ref{LblFunctionDesignatorsSyntax},
\ref{LblProcedureStatementsSyntax}\*

\texttt{<letter string>},
def
\ref{LblFunctionDesignatorsSyntax},
\ref{LblProcedureStatementsSyntax}\*

local, text
\ref{LblCompoundStatementsSemantics}\*

\texttt{<local or own type>},
def
\ref{LblTypeDeclarationsSyntax}
synt
\ref{LblArrayDeclarationsSyntax}\*

\texttt{<logical operator>},
def
\ref{LblBasicSymbolsDelimiters}
synt
\ref{LblBooleanExpressionsSyntax}
text
\ref{LblBooleanExpressionsOperators}\*

\texttt{<logical value>},
def
\ref{LblBasicSymbolsLogical}
synt
\ref{LblBasicSymbols},
\ref{LblBooleanExpressionsSyntax}\*

\texttt{<lower bound>},
def
\ref{LblArrayDeclarationsSyntax}
text
\ref{LblArrayDeclarationsLowerUpperExpressions}\*

minus -,
synt
\ref{LblBasicSymbolsDelimiters},
\ref{LblBasicSymbolsNumbersSyntax},
\ref{LblArithmeticExpressionsSyntax}
text
\ref{LblOperatorsAndTypesP1}\*

multiply \mtim{},
synt
\ref{LblBasicSymbolsDelimiters},
\ref{LblArithmeticExpressionsSyntax}
text
\ref{LblOperatorsAndTypesP1}\*

\texttt{<multiplying operator>},
def
\ref{LblArithmeticExpressionsSyntax}\*

non-local,
text
\ref{LblCompoundStatementsSemantics}\*

\texttt{<number>},
def
\ref{LblBasicSymbolsNumbersSyntax}
text
\ref{LblBasicSymbolsNumbersSemantics},
\ref{LblBasicSymbolsNumbersTypes}\*

\texttt{<open string>},
def
\ref{LblBasicSymbolsStringsSyntax}\*

\texttt{<operator>},
def
\ref{LblBasicSymbolsDelimiters}\*

\keyw{own},
synt
\ref{LblBasicSymbolsDelimiters},
\ref{LblTypeDeclarationsSyntax}
text
\ref{LblDeclarations},
\ref{LblArrayDeclarationsIdentityOfSubscriptedVariables}\*

\texttt{<parameter delimiter>},
def
\ref{LblFunctionDesignatorsSyntax},
\ref{LblProcedureStatementsSyntax}
synt
\ref{LblProcedureDeclarationsSyntax}
text
\ref{LblProcedureParameterDelimiters}\*

parentheses (),
synt
\ref{LblBasicSymbolsDelimiters},
\ref{LblFunctionDesignatorsSyntax},
\ref{LblArithmeticExpressionsSyntax},
\ref{LblBooleanExpressionsSyntax},
\ref{LblDesignationalExpressionsSyntax},
\ref{LblProcedureStatementsSyntax},
\ref{LblProcedureDeclarationsSyntax}
text
\ref{LblArithmeticExpressionsPrecedenceP2}\*

plus +,
synt
\ref{LblBasicSymbolsDelimiters},
\ref{LblBasicSymbolsNumbersSyntax},
\ref{LblArithmeticExpressionsSyntax}
text
\ref{LblOperatorsAndTypesP1}\*

\texttt{<primary>},
def
\ref{LblArithmeticExpressionsSyntax}\*

\keyw{procedure},
synt
\ref{LblBasicSymbolsDelimiters},
\ref{LblProcedureDeclarationsSyntax}\*

\texttt{<procedure body>},
def
\ref{LblProcedureDeclarationsSyntax}\*

\texttt{<procedure declaration>},
def
\ref{LblProcedureDeclarationsSyntax}
synt
\ref{LblDeclarations}
text
\ref{LblSwitchDeclarations}\*

\texttt{<procedure heading>},
def
\ref{LblProcedureDeclarationsSyntax}
text
\ref{LblProcedureDeclarationsSemantics}\*

\texttt{<procedure identifier>},
def
\ref{LblFunctionDesignatorsSyntax}
synt
\ref{LblFunctionDesignatorsSyntax},
\ref{LblProcedureStatementsSyntax},
\ref{LblProcedureDeclarationsSyntax}
text
\ref{LblProcedureStatementsRestrictionsP4}\*

\texttt{<procedure statement>},
def
\ref{LblProcedureStatementsSyntax}
synt
\ref{LblCompoundStatementsSyntax}
text
\ref{LblProcedureStatementsSemantics}\*

\texttt{<program>},
def
\ref{LblCompoundStatementsSyntax}
text
\ref{LblStructureOfTheLanguage}\*

\texttt{<proper string>},
def
\ref{LblBasicSymbolsStringsSyntax}\*

\keyw{real},
synt
\ref{LblBasicSymbolsDelimiters},
\ref{LblTypeDeclarationsSyntax}
text
\ref{LblTypeDeclarationsSemantics}\*

\texttt{<relation>},
def
\ref{LblBooleanExpressionsSyntax}
text
\ref{LblBooleanExpressionsOperators}\*

\texttt{<relational operator>},
def
\ref{LblBasicSymbolsDelimiters},
\ref{LblBooleanExpressionsSyntax}\*

scope,
text
\ref{LblQuantitiesKindsAndScopes}\*

semicolon ; ,
synt
\ref{LblBasicSymbolsDelimiters},
\ref{LblCompoundStatementsSyntax},
\ref{LblProcedureDeclarationsSyntax}\*

\texttt{<separator>},
def
\ref{LblBasicSymbolsDelimiters}\*

\texttt{<sequential operator>},
def
\ref{LblBasicSymbolsDelimiters}\*

\texttt{<simple arithmetic expression>},
def
\ref{LblArithmeticExpressionsSyntax}
text
\ref{LblArithmeticExpressionsSemantics}\*

\texttt{<simple Boolean>},
def
\ref{LblBooleanExpressionsSyntax}\*

\texttt{<simple designational expression>},
def
\ref{LblDesignationalExpressionsSyntax}\*


\texttt{<simple variable>},
def
\ref{LblVariablesSyntax}
synt
\ref{LblTypeDeclarationsSyntax}
text
\ref{LblBasicSymbolsIdentifiersSemantics}\*

space \blankchar{},
synt
\ref{LblBasicSymbolsDelimiters}
text
\ref{LblBasicSymbolsDelimiters},
\ref{LblBasicSymbolsStringsSemantics}\*

\texttt{<specification part>},
def
\ref{LblProcedureDeclarationsSyntax}
text
\ref{LblProcedureDeclarationsSpecifications}\*

\texttt{<specificator>},
def
\ref{LblBasicSymbolsDelimiters}\*

\texttt{<specifier>},
def
\ref{LblProcedureDeclarationsSyntax}\*

standard function,
text
\ref{LblStandardFunctions},
\ref{LblTransferFunctions}\*

\texttt{<statement>},
def
\ref{LblCompoundStatementsSyntax}
synt
\ref{LblConditionalStatementsSyntax},
\ref{LblForStatementsSyntax},
\ref{LblProcedureDeclarationsSyntax}
text
\ref{LblStatements} (complete section)\*

statement bracket see \keyw{begin} \keyw{end}\*

\keyw{step},
synt
\ref{LblBasicSymbolsDelimiters},
\ref{LblForStatementsSyntax}
text
\ref{LblForStatementsSemanticsStepUntil}\*

\keyw{string},
synt
\ref{LblBasicSymbolsDelimiters},
\ref{LblProcedureDeclarationsSyntax}\*

\texttt{<string>},
def
\ref{LblBasicSymbolsStringsSyntax}
synt
\ref{LblFunctionDesignatorsSyntax},
\ref{LblProcedureStatementsSyntax}
text
\ref{LblBasicSymbolsStringsSemantics}\*

string quotes `',
synt
\ref{LblBasicSymbolsDelimiters},
\ref{LblBasicSymbolsStringsSyntax}
text
\ref{LblBasicSymbolsStringsSemantics}\*

subscript,
text
\ref{LblVariableSubscriptsP1}\*

subscript bound,
text
\ref{LblArrayDeclarationsSemanticsP1}\*

subscript brackets [],
synt
\ref{LblBasicSymbolsDelimiters},
\ref{LblVariablesSyntax},
\ref{LblDesignationalExpressionsSyntax},
\ref{LblArrayDeclarationsSyntax}\*

\texttt{<subscripted variable>},
def
\ref{LblVariablesSyntax}
text
\ref{LblVariableSubscriptsP1}\*

\texttt{<subscript expression>},
def
\ref{LblVariablesSyntax}
synt
\ref{LblDesignationalExpressionsSyntax}\*

\texttt{<subscript list>},
def
\ref{LblVariablesSyntax}\*

successor,
text
\ref{LblStatements}\*

\keyw{switch},
synt
\ref{LblBasicSymbolsDelimiters},
\ref{LblSwitchDeclarationsSyntax},
\ref{LblProcedureDeclarationsSyntax}\*

\texttt{<switch declaration>},
def
\ref{LblSwitchDeclarationsSyntax}
synt
\ref{LblDeclarations}
text
\ref{LblSwitchDeclarationsSemantics}\*

\texttt{<switch designator>},
def
\ref{LblDesignationalExpressionsSyntax}
text
\ref{LblDesignationalExpressionsSemantics}\*

\texttt{<switch identifier>},
def
\ref{LblDesignationalExpressionsSyntax}
synt
\ref{LblFunctionDesignatorsSyntax},
\ref{LblProcedureStatementsSyntax},
\ref{LblSwitchDeclarationsSyntax}\*

\texttt{<switch list>},
def
\ref{LblSwitchDeclarationsSyntax}\*

\texttt{<term>},
def
\ref{LblArithmeticExpressionsSyntax}\*

ten \ten{},
synt
\ref{LblBasicSymbolsDelimiters},
\ref{LblBasicSymbolsNumbersSyntax}\*

\keyw{then},
synt
\ref{LblBasicSymbolsDelimiters},
\ref{LblArithmeticExpressionsSyntax},
\ref{LblConditionalStatementsSyntax}\*

transfer function,
text
\ref{LblTransferFunctions}\*

\keyw{true},
synt
\ref{LblBasicSymbolsLogical}\*

\texttt{<type>},
def
\ref{LblTypeDeclarationsSyntax}
synt
\ref{LblProcedureDeclarationsSyntax}
text
\ref{LblValuesAndTypes}\*

\texttt{<type declaration>},
def
\ref{LblTypeDeclarationsSyntax}
synt
\ref{LblDeclarations}
text
\ref{LblTypeDeclarationsSemantics}\*

\texttt{<type list>},
def
\ref{LblTypeDeclarationsSyntax}\*

\texttt{<unconditional statement>},
def
\ref{LblCompoundStatementsSyntax},
\ref{LblConditionalStatementsSyntax}\*

\texttt{<unlabelled basic statement>},
def
\ref{LblCompoundStatementsSyntax}\*

\texttt{<unlabelled block>},
def
\ref{LblCompoundStatementsSyntax}\*

\texttt{<unlabelled compound>},
def
\ref{LblCompoundStatementsSyntax}\*

\texttt{<unsigned integer>},
def
\ref{LblBasicSymbolsNumbersSyntax},
\ref{LblDesignationalExpressionsSyntax}\*

\texttt{<unsigned number>},
def
\ref{LblBasicSymbolsNumbersSyntax}
synt
\ref{LblArithmeticExpressionsSyntax}\*

\keyw{until},
synt
\ref{LblBasicSymbolsDelimiters},
\ref{LblForStatementsSyntax}
text
\ref{LblForStatementsSemanticsStepUntil}\*

\texttt{<upper bound>},
def
\ref{LblArrayDeclarationsSyntax}
text
\ref{LblArrayDeclarationsLowerUpperExpressions}\*

\keyw{value},
synt
\ref{LblBasicSymbolsDelimiters},
\ref{LblProcedureDeclarationsSyntax}\*

value,
text
\ref{LblValuesAndTypes},
\ref{LblArithmeticExpressionsSemantics}\*

\texttt{<value part>},
def
\ref{LblProcedureDeclarationsSyntax}
text
\ref{LblProcedureStatementsSemanticsCallByValue}\*

\texttt{<variable>},
def
\ref{LblVariablesSyntax}
synt
\ref{LblArithmeticExpressionsSyntax},
\ref{LblBooleanExpressionsSyntax},
\ref{LblAssignmentStatementSyntax},
\ref{LblForStatementsSyntax}
text
\ref{LblVariablesSemantics}\*

\texttt{<variable identifier>},
def
\ref{LblVariablesSyntax}\*

\keyw{while},
synt
\ref{LblBasicSymbolsDelimiters},
\ref{LblForStatementsSyntax}
text
\ref{LblForStatementsSemanticsWhileElement}

\vspace{4em}

Note.  This report is published in Numerische Mathematik, in the
Communications of the ACM, and in the Journal of the British Computer
Soc.  Reproduction of this report for any purpose is explicitly
permitted; reference should be made to this issue of Numerische
Mathematik and to the respective issues of the Communications and the
Journal of the British Computer Soc.  as the source.

\vspace{1em}

Technical University Delft\*
Delft, Holland\*
W. L. van der Poel,\*
(Chairman of Working Group 2.1 on Algol of the\*
International Federation for Information Processing)


\end{document}
