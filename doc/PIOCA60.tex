\documentclass[a4paper,11pt]{article}
\usepackage{a4wide}
\usepackage[latin2]{inputenc}
\usepackage[
  bookmarks=true,
  bookmarksopen=true,
  colorlinks=false,
  urlbordercolor={0 0 0},
  pdfhighlight=/I,
  pdftitle=A\ Proposal\ For\ InputOutput\ Conventions\ In\ ALGOL\ 60,
  pdfauthor=Petr\ Machata,
  pdfstartview=FitH,
  bookmarksopenlevel=2
  ]{hyperref}
\usepackage{subscript}
\pagestyle{empty}
\topmargin 0in
\begin{document}
\setcounter{secnumdepth}{5}

\title{A Proposal For Input-Output Conventions In ALGOL 60\\
A Report of the Subcommittee on ALGOL of the ACM Programming Languages Committee.}

\author{S. Gorn, Editor; R. W. Bemer, Asst. Editor, Glossary \& Terminology\\
  J. Green, Asst. Editor, Programming Languages\\
  E. Lohse, Asst. Editor, Information Interchange.\\
  D. E. Knuth\footnote{California Institute of Technology, Pasadena, Calif.}, Chairman\\
  L. L. Bumgarner, P.Z. Ingerman, J.N. Merner\\
  D. E. Hamilton, M.P. Lietzke, D.W. Ross}

\date{May 1964}

\maketitle

\def\keyw#1{{\tt \bf \underline{#1}}}
\def\blankchar{\leavevmode\hbox{\tt\char`\ }}

\def\program#1{{\tt\begin{tabbing}#1\end{tabbing}}}

\def\Algol{{\sc Algol}\space}

\def\mleqv{$\equiv$}
\def\mlnot{$\neg$}
\def\mland{$\wedge$}
\def\mlor{$\vee$}
\def\mlimp{$\Rightarrow$}
\def\mlt{$<$}
\def\mle{$\leq$}
\def\mgt{$>$}
\def\mge{$\geq$}
\def\meq{$=$}
\def\mne{$\neq$}
\def\mpow{$\uparrow$}
\def\mtim{$\times$}
\def\mdiv{$\div$}
\def\ten{\textsubscript{10}}

\def\mychapter#1{{\Huge \bf \vspace{1em}#1}\vspace{1.0em}}
\def\mysection#1{{\Large \bf \vspace{1em}#1}\vspace{1.0em}}
\def\mysubsection#1{{\large \bf \vspace{1em}#1}\vspace{1.0em}}

\def\note#1{{{\it Note.\space}#1}}

The \Algol 60 language as first defined made no explicit reference to
input and output processes. Such processes appeared to be quite
dependent on the computer used, and so it was difficult to obtain
agreement on those matters. As time has passed, a great many \Algol
compilers have come into use, and each compiler has incorporated some
input-output facilities. Experience has shown that such facilities can
be introduced in a manner which is compatible and consistent with the
\Algol language, and which (more importantly) is almost completely
machine-independent. However, the existing implementations have taken
many different approaches to the subject, and this has hampered the
interchange of programs between installations. The ACM \Algol
committee has carefully studied the various proposals in an attempt to
define a set of conventions for doing input and output which would be
suitable for use on most computers.  The present report constitutes
the recommendations of that committee.

The input-output conventions described here do not involve extensions
or changes to the \Algol 60 language.  Hence they can be incorporated
into existing processors with a minimum of effort. The conventions
take the form of a set of procedures\footnote{Throughout this report,
names of system procedures are in lower case, and names of procedures
used in illustrative examples are in UPPER CASE.}, which are to be
written in code for the various machines; this report discusses the
function and use of these procedures. The material contained in this
proposal is intended to supplement the procedures {\it in real}, {\it
out real}, {\it in symbol}, {\it out symbol} which have been defined
by the international \Algol committee; the procedures described here
could, with trivial exceptions, be expressed in terms of these
four\footnote{Defined at meeting of IFIP/WG2.1--\Algol in Delft during
September, 1963. The definition of these procedures is not available
for publication at this time.--J. G., Ed.}.

The first part of this report describes the methods by which formats
are represented; then the calls on the input and output procedures
themselves are discussed. The primary objective of the present report
is to describe the proposal concisely and precisely, rather than to
give a programmer's introduction to the input-output conventions.  A
simpler and more intuitive (but less exact) description can be written
to serve as a teaching tool.

Many useful ideas were suggested by input-output conventions of the
compilers listed in the references below.  We are also grateful for
the extremely helpful contributions of F. L. Bauer, M. Paul,
H. Rutishauser, K. Samelson, G. Seegm\"uller, W. L. v. d. Poel, and
other members of the European computing community, as well as
A. Evans, Jr., R. W. Floyd, A. G. Grace, J. Green, G. E. Haynam, and
W. C. Lynch of the USA.

\section{Formats}
\label{Lbl++Formats}

In this section a certain type of string, which specifies the format
of quantities to be input or output, is defined, and its meaning is
explained.

\subsection[Number Formats]{Number Formats (cf. ALGOL Report 2.5)}

\subsubsection{Syntax}

Basic components :

\begin{flushleft}
\vspace{0.2em}\texttt{<replicator> ::= <unsigned integer> | X }\\

\vspace{0.2em}\texttt{<insertion> ::= B | <replicator> B | <string>}\\

\vspace{0.2em}\texttt{<insertion sequence> ::= <empty> | <insertion sequence> <insertion>}\\

\vspace{0.2em}\texttt{<Z> ::= Z | <replicator> Z | Z <insertion sequence> C |\\
  \hspace{1.0cm}<replicator> Z <insertion sequence> C}\\

\vspace{0.2em}\texttt{<Z part> ::= <Z> | <Z part> <Z> | <Z part> <insertion>}\\

\vspace{0.2em}\texttt{<D> ::= D | <replicator> D | D <insertion sequence> C |\\
  \hspace{1.0cm}<replicator> D <insertion sequence> C}\\

\vspace{0.2em}\texttt{<D part> ::= <D> | <D part> <D> | <D part> <insertion>}\\

\vspace{0.2em}\texttt{<T part> ::= <empty> | T <insertion sequence>}\\

\vspace{0.2em}\texttt{<sign part> :: = <empty> | <insertion sequence> $+$ |\\
  \hspace{1.0cm}<insertion sequence> $-$}\\

\vspace{0.2em}\texttt{<integer part> :: = <Z part> | <D part> | <Z part> <D part>}\\
\end{flushleft}

Format structures:

\begin{flushleft}
\vspace{0.2em}\texttt{<unsigned integer format> ::= <insertion sequence> <integer part>}\\

\vspace{0.2em}\texttt{<decimal fraction format> ::= $.$ <insertion sequence> <D part> <T part> |\\
  \hspace{1.0cm}V <insertion sequence> <D part> <T part>}\\

\vspace{0.2em}\texttt{<exponent part format> ::= \ten<sign part> <unsigned integer format>}\\

\vspace{0.2em}\texttt{<decimal number format> ::= <unsigned integer format> <T part> |\\
  \hspace{1.0cm}<insertion sequence> <decimal fraction format> |\\
  \hspace{1.0cm}<unsigned integer format> <decimal fraction format>}\\

\vspace{0.2em}\texttt{<number format> ::= <sign part> <decimal number format> |\\
  \hspace{1.0cm}<decimal number format> + <insertion sequence> |\\
  \hspace{1.0cm}<decimal number format> - <insertion sequence> |\\
  \hspace{1.0cm}<sign part> <decimal number format> <exponent part format>}\\
\end{flushleft}

\note{This syntax could have been described more simply, but the
rather awkward constructions here have been formulated so that no
syntactic ambiguities (in the sense of formal hmguage theory) will
exist.}

\subsubsection{Examples}

Examples of number formats appear in Figure \ref{Fig++ExamplesOfNumberFormats}.
\begin{figure}

\begin{tabular*}{15.0cm}{p{5cm}|p{5cm}|p{5cm}}

Number format &
Result from $-13.296$ &
Result from $1007.999$ \\

\cline{1-3}

{\tt +ZZZCDDD.DD} &
{\tt -013.30} &
{\tt +1,008.00} \\

{\tt +3ZC3D.2D} &
{\tt -013.30} &
{\tt +1,008.00} \\

{\tt -3D2B3D.2DT}   &
{\tt -000\blankchar{}\blankchar{}013.29} &
{\tt 001\blankchar{}\blankchar{}007.99} \\

{\tt 5Z.5D-} &
{\tt 13.29600-} &
{\tt 1007.99900} \\

{\tt `integer\blankchar{}part\blankchar{}'-4ZV `,\blankchar{}fraction'B3D} &
{\tt integer\blankchar{}part\blankchar{}\blankchar{}\blankchar{}-13 ,\blankchar{}fraction\blankchar{}296} &
{\tt integer\blankchar{}part\blankchar{}1007 ,\blankchar{}fraction\blankchar{}999} \\

{\tt -.5D\ten{}+2D`...'} &
{\tt -.13296\ten{}+02...} &
{\tt .10080\ten{}+04...} \\

{\tt +ZD\ten{}2Z} &
{\tt -13} &
{\tt +10\ten{}\blankchar{}2} \\

{\tt +D.DDBDDBDDB\ten+DD} &
{\tt -1.32\blankchar{}96\blankchar{}00\blankchar{}\ten{}+01} &
{\tt +1.00\blankchar{}79\blankchar{}99\blankchar{}\ten{}+03} \\

{\tt XB.XD\ten{}-DDD} &
{\tt (depends on call)} &
{\tt (depends on call)} \\
\end{tabular*}
\caption{Examples of number formats}
\label{Fig++ExamplesOfNumberFormats}
\end{figure}

\subsubsection{Semantics}

The above syntax defines the allowable strings which can comprise a
``number format.''  We will first describe the interpretation to be
taken during {\it output}.

\paragraph{Replicators}
\label{Lbl++Formats+Number+Semantics+Replicators}
An unsigned integer {\tt n} used as replicator means the quantity is
repeated n times; thus {\tt 3B} is equivalent to {\tt BBB}. The
character {\tt X} as replicator means a number of times which will be
specified when the format is called (see Section
\ref{Lbl++IoProc+Layout+Format}).

\paragraph{Insertions}
The syntax has been set up so that strings, delimited by string
quotes, may be inserted anywhere within a number format. The
corresponding information in the strings (except for the outermost
string quotes) will appear inserted in the same place with respect to
the rest of the number. Similarly, the letter {\tt B} may be inserted
anywhere within a number format, and it stands for a blank space.

\paragraph{Sign, zero, and comma suppression}
\label{Lbl++Formats+Number+Semantics+SignZeroComma}

The portion of a number to the left of the decimal point consists of
an optional sign, then a sequence of {\tt Z}'s and a sequence of {\tt
D}'s, with possible {\tt C}'s following a {\tt Z} or a {\tt D}, plus
possible insertion characters.

The convention on signs is the following: (a) if no sign appears, the
number is assumed to be positive, and the treatment of negative
numbers is undefined; (b) if a plus sign appears, the sign will appear
as $+$ or $-$ on the external medium; and (c) if a minus sign appears,
the sign will appear if minus, and will be suppressed if plus.

The letter {\tt Z} stands for zero suppression, and the letter {\tt D}
stands for digit printing {\it without} zero suppression.  Each {\tt
Z} and {\tt D} stands for a single digit position; a zero digit
specified by {\tt Z} will be suppressed, i.e. replaced by a blank
space, when all digits to its left are zero.  A digit specified by
{\tt D} will always be printed.  Note that the number zero printed
with all {\tt Z}'s in the format will give rise to all blank spaces,
so at least one {\tt D} should usually be given somewhere in the
format.

The letter {\tt C} stands for a comma.  A comma following a {\tt D}
will always be printed; a comma following a {\tt Z} will be printed
except when zero suppression takes place at that {\tt Z}. Whenever
zero or comma suppression takes place, the sign (if any) is printed in
place of the rightmost character suppressed.

\paragraph{Decimal points}

The position of the decimal point is indicated either by the character
``.'' or by the letter {\tt V}. In the former case, the decimal point
appears on the external medium; in the latter case, the decimal point
is ``implied,'' i.e., it takes up no space on the external
medium. (This feature is most commonly used to save time and space
when preparing input data.) Only {\tt D}'s (no {\tt Z}'s) may appear
to the right of the decimal point.

\paragraph{Truncation}

On output, nonintegral numbers are usually rounded to fit the format
specified.  If the letter {\tt T} is used, however, truncation takes
place instead.  Rounding and truncation of a number {\tt X} to {\tt d}
decimal places are defined as follows:

\begin{tabular*}{15.0cm}{rp{10.0cm}}
Rounding   & $10^{-d}\cdot entier(10^d X + .5)$\\
Truncation & $10^{-d}\cdot sign(X)\cdot entier(10^d abs(X))$\\
\end{tabular*}

\paragraph{Exponent part}

The number following a ``10'' is treated exactly the same as the
portion of a number to the left of a decimal point (Section
\ref{Lbl++Formats+Number+Semantics+SignZeroComma}), except if the ``D
part'' of the exponent is empty, i.e. no {\tt D}'s appear, and if the
exponent is zero, the ``10'' and the sign are deleted.

\paragraph{Two types of numeric format}

Number formats are of two principal kinds:

\begin{enumerate}

\item[(a)] Decimal number with no exponent.  In this case, the number
  is aligned according to the decimal point with the picture in the
  format, and it is then truncated or rounded to the appropriate
  number of decimal places.  The sign may precede or follow the
  number.

\item[(b)] Decimal number with exponent.  In this case, the number is
  transforlned into the format of the decimal number with its most
  significant digit nonzero; the exponent is adjusted accordingly.  If
  the nmnber is zero, both the decimal part and the exponent part are
  output as zero.

\end{enumerate}

If in case (a) the number is too large to be output in the specified
form, or if in case (b) the exponent is too large, an overflow error
occurs.  The action which takes place on overflow is undefined; it is
recommended that the number of characters used in the output be the
same as if no overflow had occurred, and that as much significant
information as possible be output.

\paragraph{Input}
\label{Lbl++Formats+Number+Semantics+Input}

A number input with a particular format specification should in
general be the same as the number which would be output with the same
format, except less error checking occurs. The rules are, more
precisely:

\begin{enumerate}

\item[(a)] Leading zeros and commas may appear even though {\tt Z}'s
  are used in the format. Leading spaces may appear even if {\tt D}'s
  are used.  In other words, no distinction between {\tt Z} and {\tt
  D} is made on input.

\item[(b)] Insertions take the same amount of space in the same
  positions, but the characters appearing there are ignored on input.
  In other words, an insertion specifies only the number of characters
  to ignore, when it appears in an input format.

\item[(c)] If the format specifies a sign at the left, the sign may
  appear in any {\tt Z}, {\tt D} or {\tt C} position as long as it is
  to the left of the number.  A sign specified at the right must
  appear in place.

\item[(d)] The following things are checked: The positions of commas,
  decimal points, ``10'', and the presence of digits in place of {\tt
  D} or {\tt Z} after the first significant digit.  If an error is
  detected in the data, the result is undefined; it is recommended
  that the input procedure attempt to reread the data as if it were in
  standard format (Section \ref{Lbl++Formats+StandardFormat}) and also to give some error indication
  compatible with the system being used.  Such an error indication
  might be suppressed at the programmer's option if the data became
  meaningful when it was reread in standard format.

\end{enumerate}


\subsection{Other Formats}

\subsubsection{Syntax}

\begin{flushleft}
\vspace{0.2em}\texttt{<S> ::= S | <replicator> S}\\

\vspace{0.2em}\texttt{<string format> ::= <insertion sequence> <S> | <string format> <S> |\\
  \hspace{1.0cm}<string format> <insertion>}\\

\vspace{0.2em}\texttt{<A> ::= A | <replicator> A}\\

\vspace{0.2em}\texttt{<alpha format> ::= <insertion sequence> <A> | <alpha format> <A> |\\
  \hspace{1.0cm}<alpha format> <insertion>}\\

\vspace{0.2em}\texttt{<nonformat> ::= I | R | L}\\

\vspace{0.2em}\texttt{<Boolean part> ::= P | 5F | FFFFF | F}\\

\vspace{0.2em}\texttt{<Boolean format> ::=\\
  \hspace{1.0cm}<insertion sequence> <Boolean part> <insertion sequence>}\\

\vspace{0.2em}\texttt{<title format> ::= <insertion> | <title format> <insertion>}\\

\vspace{0.2em}\texttt{<alignment mark> ::= / | \mpow{} | <replicator>$/$ | <replietor>\mpow{}}\\

\vspace{0.2em}\texttt{<format item 1> ::= <number format> | <string format> |\\
  \hspace{1.0cm}<alpha format> | <nonformat> | <Boolean format> | <title format> |\\
  \hspace{1.0cm}<alignment mark> <format item 1>}\\

\vspace{0.2em}\texttt{<format item> ::= <format item 1> | <alignment mark> |\\
  \hspace{1.0cm}<format item> <alignment mark>}\\

\end{flushleft}


\subsubsection{Examples}

\program{\mpow{}5Z.5D///\\
3S`='6S4B\\
AA`='\\
\mpow{}R\\
P\\
/`Execution.'\mpow{}}


\subsubsection{Semantics}

\paragraph{String format}

A string format is used for output of string quantities.  Each of the
{\tt S}-positions in the format corresponds to a single character in
the string which is output.  If the string is longer than the number
of {\tt S}'s, the leftmost characters are transferred; if the string
is shorter, \blankchar{}-symbols are effectively added at the right of
the string.

The word ``character'' as used in this report refers to one unit of
information on the external input or output medimn; if \Algol basic
symbols are used in strings which do not have a single-character
representation on the external medium being used, the result is
undefined.

\paragraph{Alpha format}
\label{Lbl++Formats+Other+Semantics+Alpha}

Each letter {\tt A} means one character is to be transmitted; this is
the same as {\tt S}-format, except the \Algol equivalent of the
alphabetics is of type integer rather than a string.  The translation
between the external and internal code will vary from one machine to
another, and so programmers should refrain from using this feature in
a machine-dependent manner.  Each implelnentor should specify the
maximum number of characters which can be used for a single integer
variable.  The following operations are undefined for quantities which
have been input using alpha format: arithmetic operations, relations
except ``\meq'' and ``\mne'', and output using a different number of
{\tt A}'s in the output format.  If the integer is output using the
same nmnber of {\tt A}'s, the same string will be output as was input.

A programmer may work with these alphabetic quantities in a
machine-independent manner by using the transfer function {\it
equiv(S)} where {\it S} is a string; the value of {\it equiv(S)} is of
type integer, and it is defined to have exactly the same value as if
the string {\it S} had been input using alpha format.  For example,
one may write\vspace{0.5em}\\
{\tt\keyw{if} X = equiv(`ALPHA') \keyw{then} \keyw{go to} PROCESS ALPHA;}\vspace{0.5em}\\
where the value of {\it X} has been input using the format ``AAAAA''.

\paragraph{Nonformat}

An {\tt I}, {\tt R} or {\tt L} is used to indicate that the value of a
single variable of integer, real, or Boolean type, respectively, is to
be input or output from or to an external medium, using the internal
machine representation.  If a value of type integer is output with
{\tt R}-format or if a value of type real is input with {\tt
I}-format, the appropriate transfer function is invoked.  The precise
behavior of this format, and particularly its interaction with other
formats, is undefined in general.

\paragraph{Boolean format}

When Boolean quantities are input or output, the format {\tt P}, {\tt
F}, {\tt 5F} or {\tt FFFFF} must be used.  The correspondence is
defined as follows:\vspace{0.5em}\\
\begin{tabular*}{15.0cm}{cccc}
Internal to \Algol & {\tt P} & {\tt F} & {\tt 5F} \meq\space {\tt FFFFF}     \\
\cline{1-4}
\keyw{true}        & {\tt 1} & {\tt T} & {\tt TRUE\blankchar} \\
\keyw{false}       & {\tt 0} & {\tt F} & {\tt FALSE}          \\
\end{tabular*}\vspace{0.5em}\\
On input, anything failing to be in the proper form is undefined.


\paragraph{Title format}

All formats discussed so far have given a correspondence between a
single \Algol real, integer, Boolean, or string quantity and a number
of characters in the input or output.  A title format item consists
entirely of insertions and alignment marks, and so it does not require
a corresponding \Algol quantity.  On input, it merely causes skipping
of the characters, and on output it causes emission of the insertion
characters it contains. (If titles are to be input, alpha format
should be used; see Section \ref{Lbl++Formats+Other+Semantics+Alpha})


\paragraph{Alignment marks}

The characters ``/'' and ``\mpow{}'' in a format item indicate line
and page control actions.  The precise definition of these actions
will be given later (see Section \ref{Lbl++IoProc+IoCalls}); they have
the following intuitive interpretation: (a) ``/'' means go to the next
line, in a manner similar to the ``carriage return'' operation on a
typewriter. (b) ``\mpow{}'' means do a /-operation and then skip to
the top of the next page.

Two or more alignment marks indicates the number of times the
operations are to be performed; for example, ``//'' on output means
the current line is completed and the next line is effectively set to
all blanks.  Alignment marks at the left of a format item cause
actions to take place before the regular format operation, and if they
are at the right they take place afterwards.

\note{On machines which do not have the character \mpow{} in their
character set, it is recommended that some convenient character such
as an asterisk be substituted for \mpow{} in format strings.}


\subsection{Format Strings}
\label{Lbl++Formats+FormatStrings}

The format items mentioned above are combined into format strings
according to the rules in this section.

\subsubsection{Syntax}

\begin{flushleft}
\vspace{0.2em}\texttt{<format primary> ::= <format item> |\\
  \hspace{1.0cm}<replicator> (<format secondary>) | (<format secondary>)}\\

\vspace{0.2em}\texttt{<format secondary> ::= <format primary> |\\
  \hspace{1.0cm}<format secondary>, <format primary>}\\

\vspace{0.2em}\texttt{<format string> ::= `<format secondary>' | `'}\\
\end{flushleft}

\subsubsection{Examples}

\program{`4(15ZD),//'\\
`\mpow{}'\\
`.5D\ten{}+D, X(2(20B.8D\ten{}+D), 10S)'\\
``...This\blankchar{}is\blankchar{}a\blankchar{}peculiar\blankchar{}`format\blankchar{}string'''}

\subsubsection{Semantics}

A format string is simply a list of format items, which are to be
interpreted from left to right.  The construction ``{\tt <replicator>
(<format secondary>)}'' is simply an abbreviation for ``replicator''
repetitions of the parenthesized quantity (see Section
\ref{Lbl++Formats+Number+Semantics+Replicators}).  The construction
``{\tt (<format secondary>)}'' is used to specify an infinite
repetition of the parenthesized quantity.

All spaces within a format string except those which are part of
insertion substrings are irrelevant.

It is recommended that the \Algol compiler check the syntax of strings
which (from their context) are known to be format strings as the
program is compiled.  In most cases it will also be possible for the
compiler to translate format strings into an intermediate code
designed for highly efficient input-output processing by the other
procedures.


\subsection{Summary of Format Codes}

\begin{tabular*}{15.0cm}{rp{12.0cm}}
A    & alphabetic character represented as integer \\
B    & blank space \\
C    & comma\\
D    & digit\\
F    & Boolean TRUE or FALSE\\
I    & integer untranslated\\
L    & Boolean untranslated \\
P    & Boolean bit \\
R    & real untranslated \\
S    & string character \\
T    & truncation \\
V    & implied decimal point\\
X    & arbitrary replicator \\
Z    & zero suppression \\
$+$  & print the sign \\
$-$  & print the sign if it is minus \\
\ten & exponent part indicator \\
()   & delimiters of replicated format secondaries \\
,    & separates format items\\
/    & line alignment\\
\mpow{}& page alignment\\
`'   & delimiters of inserted string\\
.    & decimal point \\
\end{tabular*}


\subsection[Standard Format]{``Standard'' Format}
\label{Lbl++Formats+StandardFormat}

There is a format available {\it without} specification (cf. Section
\ref{Lbl++IoProc+IoCalls}) which has the following characteristics.

\begin{enumerate}

\item[(a)] On input, any number written according to the \Algol syntax
  for {\tt <number>} is accepted with the conventional meaning.  These
  are of {\it arbitrary} length, and they are delimited at the right
  by the following conventions:

  \begin{enumerate}
    \item[(i)] A letter or character other than a decimal point, sign,
      digit, or ``10'' occurring to the right of a decimal point,
      sign, digit, or ``10'' is a delinfiter.

    \item[(ii)] A sequence of $k$ or more blank spaces serves as a
      delimiter as in (i); a sequence of less than $k$ blank spaces is
      ignored. This number $k\geq1$ is specified by the implementor
      (and the implementor may choose to let the programmer specify k
      on a control card of some sort).

    \item[(iii)] If the number contains a decimal point, sign, digit,
      or ``10'' on the line where the number begins, the right-hand
      margin of that line serves as a delimiter of the number.
      However, if the first line of a field contains no such
      characters, the number is determined by reading several lines
      until finding a definfiter of type (i) or (ii).  In other words,
      a number is not usually split across more than one line, unless
      its first line contains nothing but spaces or characters which
      do not enter into the number itself.  (See Section
      \ref{Lbl++IoProc+IoCalls} for further discussion of standard
      input format.)
  \end{enumerate}

\item[(b)] On output, a number is given in the form of a decimal
  number with an exponent. This decimal number has the amount of
  significant figures which the machine can represent; it is suitable
  for reading by the standard input format.  Standard output format
  takes a fixed number of characters on the output medium; this size
  is specified by each \Algol installation.  Standard output format
  can also be used for the output of strings, and in this case the
  number of characters is equal to the length of the string.

\end{enumerate}


\section{Input and Output Procedures}

\subsection{General Characteristics}
\label{Lbl++IoProc+General}

The over-all approach to input and output which is provided by the
procedures of this report will be introduced here by means of a few
examples, and the precise definition of the procedures will be given
later.

Consider first a typical case, in which we want to print a line
containing the values of the integer variables $N$ and $M$, each of
which is nonnegative, with at most five digits; also the value of
$X[M]$, in the form of a signed number with a single nonzero digit to
the left of the decimal point, and with an exponent indicated; and
finally the value of $cos(t)$, using a format with a fixed decimal
point and no exponent.  The following might be written for this case:

\program{output 4 (6, `2(BBBZZZZD),3B,+D.DDDDDD\ten{}+DDD,3B,-Z.DDDDBDDDD/',\\
  \hspace{1.0cm}N, M, X[M], cos(t))}

This example has the following significance. (a) The ``4'' in output 4
means four values are being output.  (b) The ``6'' means that output
is to go to unit number 6.  This is the logical unit number, i.e., the
programmer's number for that unit, and it does not necessarily mean
physical unit number 6.  See Section
\ref{Lbl++IoProc+General+UnitNumbers}, for further discussion of unit
numbers.  (c) The next parameter, `2(BBB \dots DDDD/', is the format
string which specifies a format for outputting the four values.  (d)
The last four parameters are the values being printed.  If $N = 500$,
$M = 0$, $X[0] = 18061579$, and $t = 3.1415926536$, we obtain the
line\vspace{0.5em}\\ {\tt
\blankchar\blankchar\blankchar\blankchar\blankchar500\blankchar\blankchar\blankchar\blankchar\blankchar\blankchar\blankchar{}0\blankchar\blankchar\blankchar+1.806158\ten{}+007\blankchar\blankchar\blankchar-1.0000\blankchar{}0000}\vspace{0.5em}\\
as output.

Notice the ``/'' used in the above format; this symbol signifies the
end of a line.  If it had not been present, more numbers could have
been placed on the same line in a future output statement.  The
programmer may build the contents of a line in several steps, as his
algorithm proceeds, without automatically starting a new line each
time output is called.  For example, the above could have been
written\vspace{0.5em}\\
{\tt output 1 (6, 'BBBZZZZD', N);\\
output 1 (6, 'BBBZZZZD', M);\\
output 2 (6,'3B,+D.DDDDDD\ten{}+DDD,3B,-Z.DDDDBDDDD',X[M],cos(t));\\
output 0 (6, '/');}\vspace{0.5em}\\
with equivalent results.

In the example above, a line of 48 characters was output.  If for some
reason these output statements are used with a device incapable of
printing 48 characters on a single line, the output would actually
have been recorded on two or more lines, according to a rule which
automatically keeps from breaking numbers between two consecutive
lines wherever possible. (The exact rule appears in Section
\ref{Lbl++IoProc+IoCalls})

Now let us go to a slightly more complicated example: the real array
$A[1:n, 1:n]$ is to be printed, starting on a new page.  Supposing
each element is printed with the format ``{\tt BB-ZZZZ.DD}'', which
uses ten characters per item, we could write the following program:

\program{output 0 (6, `\mpow');\\
\keyw{for} i := 1 \keyw{step} 1 \keyw{until} n \keyw{do}\\
\keyw{begin} \=\keyw{for} j := 1 \keyw{step} 1 \keyw{until} n \keyw{do}\\
\>\hspace{1.0cm}output 1 (6, `BB-ZZZZ.DD', A[i, j]);\\
\>output 0 (6, '//') \keyw{end}.}

If $10n$ characters will fit on one line, this little program will
print $n$ lines, double spaced, with $n$ values per line; otherwise
$n$ groups of $k$ lines separated by blank lines are produced, where
$k$ lines are necessary for the printing of $n$ values.  For example,
if $n = 10$ and if the printer has 120 character positions, 10
double-spaced lines are produced.  If, however, a 72-character printer
is being used, 7 values are printed on the first line, 3 on the next,
the third is blank, then 7 more values are printed, etc.

There is another way to achieve the above output and to obtain more
control over the page format as well.  The subject of page format will
be discussed further in Section \ref{Lbl++IoProc+HorizVertControl},
and we will indicate here the manner in which the above operation can
be done conveniently using a single output statement.  The procedures
{\it output O}, {\it output 1}, etc. mentioned above provide only for
the common cases of output, and they are essentially a special
abbreviation for certain calls on the more general procedure {\it out
list}.  This more general procedure could be used for the above
problem in the following manner:\vspace{0.5em}\\ {\tt out list (6,
LAYOUT, LIST)}\vspace{0.5em}\\ Here LAYOUT and LIST are the names of
procedures which appear below.  The first parameter of out list is the
logical unit number as described above. The second parameter is the
name of a so-called ``layout procedure''; general layout procedures
are discussed in Section \ref{Lbl++IoProc+Layout}.  The third
parameter of {\it out list} is the name of a so-called ``list
procedure''; general list procedures are discussed in Section
\ref{Lbl++IoProc+ListProc}.

In general, a layout procedure specifies the format control of the
input or output. For the case we are considering, we could write a
simple layout procedure (named ``LAYOUT'') as follows:
\program{\keyw{procedure} LAYOUT; format 1 (`\mpow{},(X(BB-ZZZZ.DD),//)', n)}
\noindent{}The $1$ in {\it format 1} means a format string containing
one {\tt X} is given. The format string is
\program{\mpow{},(X(BB-ZZZZ.DD),//)}
\noindent{}which means skip to a new page, then repeat the format {\tt
X(BB-ZZZZ.DD),//} until the last value is output.  The latter format
means that {\tt BB-ZZZZ.DD} is to be used X times, then skip to a new
line. Finally, {\it format 1} is a procedure which effectively inserts
the value of $n$ for the letter {\tt X} appearing in the format
string.

A {\it list procedure} serves to specify a list of quantities.  For the
problem under consideration, we could write a simple list procedure
(named ``LIST'') as follows:
\program{\keyw{procedure} LIST(ITEM);\\
\keyw{for} i := \keyw{step} 1 \keyw{until} n \keyw{do}\\
\hspace{1.0cm}\keyw{for} j := 1 \keyw{step} 1 \keyw{until} n \keyw{do}\\
\hspace{2.0cm}ITEM(A[i, j])}
\noindent{}Here ``{\tt ITEM(A[i, j])}'' means that $A[i, j]$ is the
next item of the list. The procedure {\it ITEM} is a formal parameter
which might have been given a different name such as {\it PIECE} or
{\it CHUNK}; list procedures are discussed in more detail in Section
\ref{Lbl++IoProc+ListProc}.

The declarations of {\it LAYOUT} and {\it LIST} above, together with
the procedure statement {\tt out list (6, LAYOUT, LIST)}, accomplish
the desired output of the array $A$.

Input is done in a manner dual to output, in such a way that it is the
exact inverse of the output process wherever possible. The procedures
{\it in list} and {\it input n} correspond to {\it out list} and {\it
output n} ($n = 0, 1, \dots$).

Two other procedures, {\it get} and {\it put}, are introduced to
facilitate storage of intermediate data on external devices.  For
example, the statement {\it put(100, LIST)} would cause the values
specified in the list procedure named {\it LIST} to be recorded in the
external medium with an identification number of 100.  The subsequent
statement {\it get(100, LIST)} would restore these values.  The
external medium might be a disk file, a drum, a magnetic tape, etc.;
the type of device and the format in which data is stored there is of
no concern to the programmer.

\subsubsection{Unit numbers}
\label{Lbl++IoProc+General+UnitNumbers}

The first parameter of input and output procedures is the logical unit
number, i.e. some number which the programmer has chosen to identify
some input or output device.  The connection between logical unit
numbers and the actual physical unit numbers is specified by the
programmer {\it outside} of the \Algol language, by means of ``control
cards'' preceding or following his program, or in some other way
provided by the \Algol implementor.  The situation which arises if the
same physical unit is being used for two different logical numbers, or
if the same physical unit is used both for input and for output, is
undefined in general.

It is recommended that the internal computer memory (e.g. the core
memory) be available as an ``input-output device,'' so that data may
be edited by means of input and output statements.


\subsection{Horizontal and Vertical Control}
\label{Lbl++IoProc+HorizVertControl}

This section deals with the way in which the sequence of characters,
described by the rules of formats in Section \ref{Lbl++Formats}, is
mapped onto input and output devices.  This is done in a manner which
is essentially independent of the device being used, in the sense that
with these specifications the programmer can anticipate how the input
or output data will appear on virtually any device.  Some of the
features of this description will, of course, be more appropriately
used on certain devices than on others.

We will begin by assuming we are doing {\it output} to a {\it
printer}.  This is essentially the most difficult case to handle, and
we will discuss the manner in which other devices fit into the same
general framework.

The page format is controlled by specifying the horizontal and the
vertical layout.  Horizontal layout is controlled in essentially the
same manner as vertical layout, and this symmetry between the
horizontal and vertical dimensions should be kept in mind for easier
understanding of the concepts in this section.

Refer to Figure \ref{Fig++HorizVertControl}; the horizontal format is
described in terms of three parameters ($L$, $R$, $P$), and the
vertical format has corresponding parameters ($L'$, $R'$, $P'$). The
parameters $L$, $L'$ and $R$, $R'$ indicate left and right margins,
respectively; Figure \ref{Fig++HorizVertControl} shows a case where $L
= L' = 4$ and $R = R' = 12$. Only positions $L$ through $R$ of a
horizontal line are used, and only lines $L'$ and $R'$ of the page are
used; we require that $1 \leq L \leq R$ and $1 \leq L' \leq R'$.  The
parameter $P$ is the nmnber of characters per line, and $P'$ is the
number of lines per page.  Although $L$, $R$, $L'$, and $R'$ are
chosen by the programmer, the values of $P$ and $P'$ are
characteristics of the device and they are usually out of the
programmer's control.  For those devices on which $P$ and $P'$ can
vary (for example, some printers have two settings, one on which there
are 66 lines per page, and another on which there are 88), the values
are specified to the system in some manner external to the \Algol
program, e.g. on control cards.  For certain devices, values of $P$ or
$P'$ might be essentially infinite.

\begin{figure}
  \setlength{\unitlength}{1cm}
  \centering
  \begin{picture}(10.0,9.5)(0,-0.25)
    %vertical line
    \put ( 0.25,0.00){\line (+1,+0){0.25}}
    \put ( 0.50,0.00){\line (+0,+1){9.00}}
    \put ( 0.25,9.00){\line (+1,+0){0.25}}
    %...with dots
    \multiput ( 0.50,0.50)(0.00,0.50){17}{\circle*{0.1}}
    %...and numbers
    \put ( 0.00,8.50){\makebox(0.50,0.50){1}}
    \put ( 0.00,8.00){\makebox(0.50,0.50){2}}
    \put ( 0.00,7.50){\makebox(0.50,0.50){3}}
    \put ( 0.00,7.00){\makebox(0.50,0.50){4}}
    \put ( 0.00,6.50){\makebox(0.50,0.50){5}}
    \put ( 0.00,6.00){\makebox(0.50,0.50){6}}
    \put ( 0.00,5.50){\makebox(0.50,0.50){7}}
    \put ( 0.00,5.00){\makebox(0.50,0.50){8}}
    \put ( 0.00,4.50){\makebox(0.50,0.50){9}}
    \put ( 0.00,4.00){\makebox(0.50,0.50){10}}
    \put ( 0.00,3.50){\makebox(0.50,0.50){11}}
    \put ( 0.00,3.00){\makebox(0.50,0.50){12}}
    \put ( 0.00,2.50){\makebox(0.50,0.50){13}}
    \put ( 0.00,2.00){\makebox(0.50,0.50){14}}
    \put ( 0.00,1.50){\makebox(0.50,0.50){15}}
    \put ( 0.00,1.00){\makebox(0.50,0.50){16}}
    \put ( 0.00,0.50){\makebox(0.50,0.50){17}}
    \put ( 0.00,0.00){\makebox(0.50,0.50){18}}
    %...and description
    \put ( 1.00,7.33){\makebox(0.50,0.50){$L'$}}
    \put ( 0.50,7.25){\line (+2,+1){0.55}}
    \put ( 1.00,2.75){\makebox(0.50,0.50){$R'$}}
    \put ( 0.50,3.25){\line (+2,-1){0.55}}
    \put ( 1.00,-0.25){\makebox(0.50,0.50){$P'$}}
    \put ( 0.50,0.25){\line (+2,-1){0.55}}

    %horizontal line
    \put ( 1.00,4.25){\line (+0,+1){0.25}}
    \put ( 1.00,4.50){\line (+1,+0){9.00}}
    \put (10.00,4.25){\line (+0,+1){0.25}}
    %...with dots
    \multiput ( 1.50,4.50)(0.50,0.00){17}{\circle*{0.1}}
    %...and numbers
    \put ( 1.00,4.00){\makebox(0.50,0.50){1}}
    \put ( 1.50,4.00){\makebox(0.50,0.50){2}}
    \put ( 2.00,4.00){\makebox(0.50,0.50){3}}
    \put ( 2.50,4.00){\makebox(0.50,0.50){4}}
    \put ( 3.00,4.00){\makebox(0.50,0.50){5}}
    \put ( 3.50,4.00){\makebox(0.50,0.50){6}}
    \put ( 4.00,4.00){\makebox(0.50,0.50){7}}
    \put ( 4.50,4.00){\makebox(0.50,0.50){8}}
    \put ( 5.00,4.00){\makebox(0.50,0.50){9}}
    \put ( 5.50,4.00){\makebox(0.50,0.50){10}}
    \put ( 6.00,4.00){\makebox(0.50,0.50){11}}
    \put ( 6.50,4.00){\makebox(0.50,0.50){12}}
    \put ( 7.00,4.00){\makebox(0.50,0.50){13}}
    \put ( 7.50,4.00){\makebox(0.50,0.50){14}}
    \put ( 8.00,4.00){\makebox(0.50,0.50){15}}
    \put ( 8.50,4.00){\makebox(0.50,0.50){16}}
    \put ( 9.00,4.00){\makebox(0.50,0.50){17}}
    \put ( 9.50,4.00){\makebox(0.50,0.50){18}}
    %...and description
    \put ( 2.25,5.00){\makebox(0.50,0.50){$L$}}
    \put ( 2.75,4.50){\line (-1,+2){0.25}}
    \put ( 6.75,5.00){\makebox(0.50,0.50){$R$}}
    \put ( 6.75,4.50){\line (+1,+2){0.25}}
    \put ( 9.75,5.00){\makebox(0.50,0.50){$P$}}
    \put ( 9.75,4.50){\line (+1,+2){0.25}}
  \end{picture}
  \caption{Margins in horizontal and vertical control}
  \label{Fig++HorizVertControl}
\end{figure}

Although Figure \ref{Fig++HorizVertControl} shows a case where $P \geq
R$ and $P' \geq R'$, it is of course quite possible that $P < R$ or
$P' < R'$ (or both) might occur, since $P$ and $P'$ are in general
unknown to the programmer.  In such cases, the algorithm described in
Section \ref{Lbl++IoProc+IoCalls} is used to break up logical lines
which are too wide to fit on a physical line, and to break up logical
pages which are too large to fit a physical page.  On the other hand,
the conditions $L \leq P$ and $L' \leq P'$ are insured by setting $L$
or $L'$ equal to 1 automatically if they happen to be greater than $P$
or $P'$, respectively.

Characters determined by the output values are put onto a horizontal
line; there are three conditions which cause a transfer to the next
line: (a) normal line alignment, specified by a ``/'' in the format;
(b) $R$-overflow, which occurs when a group of characters is to be
transmitted which would pass position $R$; and (c) $P$-overflow, which
occurs when a group of characters is to be transmitted which would not
cause $R$-overflow but would pass position $P$.  When any of these
three things occurs, control is transferred to a procedure specified
by the programmer in case special action is desired (e.g. a change of
margins in case of overflow; see Section
\ref{Lbl++IoProc+Layout+EndControl}).

Similarly, there are three conditions which cause a transfer to the
next page: (a') normal page alignment, specified by a ``\mpow{}'' in the
format; (b') $R'$-overflow, which occurs when a group of characters is
to be transmitted which would appear on line $R'+1$; and (c')
$P'$-overflow, which occurs when a group of characters is to be
transmitted which would appear on line $P'+1 < R'+1$. The programmer
may indicate special procedures to be executed at this time if he
wishes, e.g. to insert a page heading, etc.

Further details concerning pages and lines will be given later. Now we
will consider how devices other than printers can be thought of in
terms of the ideas above.

A typewriter is, of course, very much like a printer and it requires
no further comment.

Punched cards with, say, 80 columns, have $P = 80$ and $P' = \infty$.
Vertical control would appear to have little meaning for punched
cards, although the implementor might choose to interpret ``\mpow{}'' to
mean the insertion of a coded or blank card.

With paper tape, we might again say that vertical control has little
or no meaning; in this case, $P$ could be the number of characters
read or written at a time.

On magnetic tape capable of writing arbitrarily long blocks, we have
$P = P' = \infty$.  We might think of each page as being a ``record,''
i.e., an amount of contiguous information on the tape which is read or
written at once.  The lines are subdivisions of a record, and $R'$
lines form a record; $R$ characters are in each line.  In this way we
can specify so-called ``blocking of records.''  Other interpretations
might be more appropriate for magnetic tapes at certain installations,
e.g. a format which would correspond exactly to printer format for
future offline listing, etc.

These examples are given merely to indicate how the concepts described
above for printers can be applied to other devices.  Each implementor
will decide what method is most appropriate for his particular
devices, and if there are choices to be made they can be given by the
programmer by means of control cards.  The manner in which this is
done is of no concern in this report; our procedures are defined
solely in terms of $P$ and $P'$.

\subsection{Layout Procedures}
\label{Lbl++IoProc+Layout}

Whenever input or output is done, certain ``standard'' operations are
assumed to take place, unless otherwise specified by the programmer.
Therefore one of the parameters of the input or output procedure is a
so-called ``layout'' procedure, which specifies all of the nonstandard
operations desired.  This is achieved by using any or all of the six
``descriptive procedures'' {\it format}, {\it h end}, {\it v end},
{\it h lim}, {\it v lim}, {\it no data} described in this section.

The precise action of these procedures can be described in terms of
the mythical concept of six ``hidden variables,'' H1, H2, H3, H4, H5,
H6.  The effect of each descriptive procedure is to set one of these
variables to a certain value; and as a matter of fact, that may be
regarded as the sum total of the effect of a descriptive procedure.
The programmer normally has no other access to these hidden variables
(see, however, Section \ref{Lbl++IoProc+Other}).  The hidden variables
have a scope which is local to {\it in list} and to {\it out list}.

\subsubsection{Format procedures}
\label{Lbl++IoProc+Layout+Format}

The descriptive procedure call \program{format (string)}
\noindent{}has the effect of setting the hidden variable H1 to
indicate the string parameter. This parameter may either be a string
explicitly written, or a formal parameter; but in any event, the
string it refers to must be a format string, which satisfies the
syntax of Section \ref{Lbl++Formats+FormatStrings}, and it must have
no ``{\tt X}'' replicators.

The procedure {\it format} is just one of a class of procedures which
have the names {\it format n}, ($n = 0, 1, \dots$).  The name {\it
format} is equivalent to {\it format O}.  In general, the procedure
{\it format n} is used with format strings which have exactly $n$ {\tt
X}-replicators.  The call is
\program{format n (string, X$_1$, X$_2$, \dots{}, X$n$)}
\noindent{}where each {\tt X$_i$} is an integer parameter called by
value.  The effect is to replace each {\tt X} of the format string by
one of the {\tt X$_i$}, with the correspondence defined from left to
right.  Each {\tt X$_i$} must be nonnegative.

For example,
\program{format 2 (`XB.XD\ten{}+DD', 5, 10)}
\noindent{}is equivalent to
\program{format (`5B.10D\ten{}+DD').}

\subsubsection{Limits}

The descriptive procedure call
\program{h lim (L, R)}
\noindent{}has the effect of setting the hidden variable H2 to
indicate the two parameters L and R. Similarly,
\program{v lim (L', R')}
\noindent{}sets H3 to indicate L' and R'.  These parameters have the
significance described in Section \ref{Lbl++IoProc+HorizVertControl}.
If {\it h lim} and {\it v lim} are not used, $L = L' = 1$ and $R = R'
= \infty$.


\subsubsection{End control}
\label{Lbl++IoProc+Layout+EndControl}

The descriptive procedure calls
\program{h end (P\textsubscript{N}, P\textsubscript{R} , P\textsubscript{P});\\
  v end (P\textsubscript{N'}, P\textsubscript{R'}, P\textsubscript{P'});}
\noindent{}have the effect of setting the hidden variables H4 and H5,
respectively, to indicate their parameters.  The parameters $P_{N}$,
$P_{R}$, $P_{P}$, $P_{N'}$, $P_{R'}$, $P_{P'}$ are names of procedures
(ordinarily dummy statements if {\it h end} and {\it v end} are not
specified) which are activated in case of normal line alignment,
$R$-overflow, $P$-overflow, normal page alignment, $R'$-overflow, and
$P'$-overflow, respectively.

\subsubsection{End of data}

The descriptive procedure call
\program{no data (L);}
\noindent{}has the effect of setting the hidden variable H6 to
indicate the parameter {\it L}.  Here {\it L} is a label.  End of data
as defined here has meaning only on input, and it does not refer to
any specific hardware features; it occurs when data is requested for
input but no more data remains on the corresponding input medium.  At
this point, a transfer to the statement labeled {\it L} will occur.
If the procedure {\it{}no data} is not used, transfer will occur to a
``label'' which has effectively been inserted just before the final
\keyw{end} in the \Algol program, thus terminating the program.  (In
this case the implementor may elect to provide an appropriate error
comment.)


\subsubsection{Examples}

A layout procedure might look as follows:
\program{\keyw{procedure} LAYOUT;\\
\keyw{begin} \=format(`/');\\
\>\keyw{if} B \keyw{then} \keyw{begin} \=format 1(`XB', Y + 10);\\
\>\>no data (L32) \keyw{end};\\
\>h lim(\keyw{if} B \keyw{then} 1 \keyw{else} 10, 30) \keyw{end};}
\noindent{}Note that layout procedures never have formal parameters;
this procedure, for example, refers to three global quantities, $B$,
$Y$ and $L32$.  Suppose Y has the value 3; then this layout
accomplishes the following:

\vspace{0.5em}
\begin{tabular*}{15.0cm}{cccc}
Hidden variable & Procedure & \keyw{if} B = \keyw{true} & \keyw{if} B = \keyw{false}\\
\cline{1-4}
H1 & {\it format}  & `13B'        & `/'          \\
H2 & {\it h lim}   & (1, 30)      & (10, 30)     \\
H3 & {\it v lim}   & (1, $\infty$)& (1, $\infty$)\\
H4 & {\it h end}   & ( , , )      & ( , , )      \\
H5 & {\it v end}   & ( , , )      & ( , , )      \\
H6 & {\it no data} & $L32$        & {\it end program}\\
\end{tabular*}
\vspace{0.5em}

As a more useful example, we can take the procedure {\it LAYOUT} of
Section \ref{Lbl++IoProc+General} and rewrite it so that the
horizontal margins (11,110) are used on the page, except that if
$P$-overflow or $R$-overflow occurs we wish to use the margins (16,
105) for overflow lines.
\program{\keyw{procedure} LAYOUT;\\
\keyw{begin} \=format 1 (`\mpow{},(X(BB-ZZZZ.DD),//)', n);\\
\>h lim (11,110); h end (K, L, L) \keyw{end};\\
\keyw{procedure} K; h lim (11,110);\\
\keyw{procedure} L; h lim (16,105);}
\noindent{}This causes the limits (16, 105) to be set whenever
overflow occurs, and the ``/'' in the format will reinstate the
original margins when it causes procedure {\tt K} to be called.  (If
the programmer wishes a more elaborate treatment of the overflow case,
depending on the value of $P$, he may do this using the procedures of
Section \ref{Lbl++IoProc+ControlProc})

\subsection{List Procedures}
\label{Lbl++IoProc+ListProc}

\subsubsection{General characteristics}
\label{Lbl++IoProc+ListProc+General}

The concept of a list procedure is quite important to the input-output
conventions described in this report, and it may also prove useful in
other applications of \Algol.  It represents a specialized application
of the standard features of ALGOL which permit a procedure identifier,
$L$, to be given as an actual parameter of a procedure, and which
permit procedures to be declared within procedures.

The purpose of a list procedure is to describe a sequence of items
which is to be transmitted for input or output.  A procedure is
written in which the name of each item $V$ is written as the argument
of a procedure, say {\it ITEM}, thus: {\it ITEM(V)}.  When the list
procedure is called by an input-output system procedure, another
procedure (such as the internal system procedure out item) will be
``substituted'' for {\it ITEM}, $V$ will be called by name, and the
value of $V$ will be transmitted for input or output. The standard
sequencing of \Algol statements in the body of the list procedure
determines the sequence of items in the list.

A simple form of list procedure might be written as follows:
\program{\keyw{procedure} LIST (ITEM);\\
  \keyw{begin} ITEM(A); ITEM(B); ITEM(C) \keyw{end}}
\noindent{}which says that the values of $A$, $B$ and $C$ are to be
transmitted.  A more typical list procedure might be:
\program{\keyw{procedure} PAIRS (ELT);\\
\keyw{for} i := 1 \keyw{step} 1 \keyw{until} n \keyw{do}\\
\keyw{begin} \=ELT(A[i]);\\
\>ELT(B[i]) \keyw{end}}
\noindent{}This procedure says that the values of the list of items
$A[1]$, $B[1]$, $A[2]$, $B[2]$, \dots, $A[n]$, $B[n]$ are to be
transmitted, in that order. Note that if $n \leq 0$ no items are
transmitted at all.

The parmneter of the ``item'' procedure (i.e. the parameter of {\it
ITEM} or {\it ELT} in the above examples) is called by name.  It may
be an arithmetic expression, a Boolean expression, or a string, in
accordance with the format which will be associated with the item.
Any of the ordinary features of \Algol may be used in a list
procedure, so there is great flexibility.

Unlike layout procedures which simply run through their statements and
set up hidden variables H1 through H6, a list procedure is executed
step by step with the input or output procedure, with control
transferring back and forth.  This is accomplished by special system
procedures such as {\it in item} and {\it out item} which are
``interlaced'' with the list procedure, as described in Sections
\ref{Lbl++IoProc+ListProc+Other} and \ref{Lbl++IoProc+IoCalls}.  The
list procedure is called with {\it in item} (or {\it out item}) as
actual parameter, and whenever this procedure is called within the
list procedure, the actual input or output is taking place.  Through
the interlacing, special format control, including the important
device-independent overflow procedures, can take place during the
transmission process.  Note that a list procedure may change the
hidden variables by calling a descriptive procedure; this can be a
valuable characteristic, e.g. when changing the format, based on the
value of the first item which is input.

\subsubsection{Other applications}
\label{Lbl++IoProc+ListProc+Other}

List procedures can actually be used in many ways in \Algol besides
their use with input or output routines; they are useful for
manipulating linear lists of items of a quite general nature.  To
illustrate this fact, and to point out how the interlacing of control
between list and driver procedures can be accomplished, here is an
example of a procedure which calculates the sum of all of the elements
in a list (assuming all elements are of integer or real type):
\program{\keyw{procedure} ADD(Y, Z);\\
\keyw{begin} \=\keyw{procedure} A(X); Z := Z + X;\\
\>Z := 0; Y(A) \keyw{end}}
\noindent{}The call {\it ADD(PAIRS, SUM)} will set the value of $SUM$
to be the sum of all of the items in the list $PAIRS$ defined in
Section \ref{Lbl++IoProc+ListProc+General}.  The reader should study
this example carefully to grasp the essential significance of list
procedures.  It is a simple and instructive exercise to write a
procedure which sets all elements of a list to zero.

\subsection{Input and Output Calls}
\label{Lbl++IoProc+IoCalls}

Here procedures are described which cause the actual transmission of
input or output to take place.

\subsubsection{Output}
\label{Lbl++IoProc+IoCalls+Output}

An output process is initiated by the call:
\program{out list (unit, LAYOUT, LIST)}
\noindent{}Here unit is an integer parameter called by value, which is
the number of an output device (cf. Section
\ref{Lbl++IoProc+General+UnitNumbers}). The parameter $LAYOUT$ is the
name of a layout procedure (Section \ref{Lbl++IoProc+Layout}), and
$LIST$ is the name of a list procedure (Section
\ref{Lbl++IoProc+ListProc}).

There is also another class of procedures, named {\it output n}, for
$n = 0, 1, 2, \dots$, which is used for output as follows:
\program{output n (unit, format string, e\textsubscript{1}, e\textsubscript{2}, \dots, e\textsubscript{n})}
\noindent{}Each of these latter procedures can be defined in terms of
out list as follows:
\program{\keyw{procedure} output n(unit, format string, e\textsubscript{1}, e\textsubscript{2}, \dots, e\textsubscript{n});\\
\keyw{begin} \={}\keyw{procedure} A; format (format string);\\
\>{}\keyw{procedure} B(P); begin P(e\textsubscript{1}); P(e\textsubscript{2}); \dots; P(e\textsubscript{n}) \keyw{end};\\
\>{}out list (unit, A, B) \keyw{end}}
\noindent{}We will therefore assume in the following rules that out
list has been called.

Let the variables $p$ and $p'$ indicate the current position in the
output for the unit under consideration, i.e. lines 1, 2, \dots, $p'$
of the current page have been completed, as well as character
positions 1, 2, \dots, $p$ of the current line (i.e., of line $p'+1$).
At the beginning of the program, $p = p' = 0$. The symbols $P$ and
$P'$ denote the line size and page size (see Section
\ref{Lbl++IoProc+HorizVertControl}). Output takes place according to
the following algorithm:

\begin{enumerate}
  \item[Step 1.] The hidden variables are set to standard values:\\
    H1 is set to the ``standard'' format `'\\
    H2 is set so that $L = 1, R = \infty$.\\
    H3 is set so that $L' = 1, R' = \infty$.\\
    H4 is set so that $P_N$, $P_R$, $P_P$ are all effectively equal to
    the $DUMMY$ procedure defined as follows: ``{\tt \keyw{procedure}
    DUMMY;;}''.\\
    H5 is set so that $P_{N'}$, $P_{R'}$, $P_{P'}$ are all effectively
    equal to DUMMY.\\
    H6 is set to terminate the program in case the data ends (this has
    meaning only on input).

    \item[Step 2.] The layout procedure is called; this may change
      some of the variables H1, H2, H3, H4, H5, H6.

    \item[Step 3.] The next format item of the format string is
      examined.  (\note{After the format string is exhausted,
      ``standard'' format, Section \ref{Lbl++Formats+StandardFormat},
      is used from then on until the end of the procedure.  In
      particular, if the format string is `', standard format is used
      throughout.}) Now if the next format item is a title format,
      i.e. requires no data item, we proceed directly to step 4.
      Otherwise, the list procedure is activated; this is done the
      first time by {\it calling} the list procedure, using as actual
      parameter a procedure named {\it out item}; this is done on all
      subsequent times by merely returning from the procedure {\it out
      item}, which will cause the list procedure to be continued from
      the latest {\it out item} call.  (\note{The identifier out item
      has scope local to out list, so a programmer may not call this
      procedure directly.})  After the list procedure has been
      activated in this way, it will either terminate or will call the
      procedure {\it out item}.  In the former case, the output
      process is completed; in the latter case, continue at step 4.

    \item[Step 4.] Take the next item from the format
      string. (\note{If the list procedure was called in step 3, it
      may have called the descriptive procedure {\it format}, thereby
      changing from the format which was examined during step 3. In
      such a case, the new format is used here.  But at this point the
      format item is effectively removed from the format string and
      copied elsewhere; so that the format string itself, possibly
      changed by further calls of {\it format}, will not be
      interrogated until the next occurrence of step 3.  If the list
      procedure has substituted a title format for a nontitle format,
      the ``item'' it specifies will not be output, since a title
      format consists entirely of insertions and alignment marks.})

      Set ``$toggle$'' to \keyw{false}. (This is used to control the
      breaking of entries between lines.) The alignment marks, if any,
      at the left of the format item, now cause process A (below) to
      be executed for each ``/'', and process B for each ``\mpow{}''.
      If the format item consists entirely of alignment marks, then go
      immediately to step 3.  Otherwise the size of the format
      (i.e. the number of characters specified in the output medium)
      is determined.  Let this size be denoted by $S$.  Continue with
      step 5.

    \item[Step 5.] Execute process C, to ensure proper page alignment.

    \item[Step 6.] Line alignment: If $p < L-l$, effectively insert
      blank spaces so that $p = L - 1$.  Now if $toggle =
      \keyw{true}$, go to step 9; otherwise, test for line overflow as
      follows: If $p + S > R$, perform process D, then call $P_R$ and
      go to step 8; otherwise, if $p + S > P$, perform process D, call
      $P_P$ , and go to step 8.

    \item[Step 7.] Evaluate the next output item and output it
      according to the rules given in Section \ref{Lbl++Formats}; in
      the case of a title format, this is simply a transmission of the
      insertions without the evaluation of an output item.  The
      pointer $p$ is set to $p + S$.  Any alignment marks at the right
      of the format item now cause activation of process A for each
      ``/'' and of process B for each ``\mpow{}''.  Return to step 3.

    \item[Step 8.] Set $toggle$ to \keyw{true}. Prepare a formatted
      output item as in step 7, but do not record it on the output
      medium yet (this is done in step 9).  Go to step 5.  (It is
      necessary to re-examine page and line alignment, which may have
      been altered by the overflow procedure; hence we go to step 5
      rather than proceeding immediately to step 9.)

    \item[Step 9.] Transfer as many characters of the current output
      item as possible into positions $p+1$, \dots, without exceeding
      position $P$ or $R$.  Adjust $p$ appropriately.  If the output
      of this item is still unfinished, execute process D again, call
      $P_R$ (if $R \leq P$) or $P_P$ (if $P < R$), and return to step
      5.  The entire item will eventually be output, and then we
      process alignment characters as in step 7, finally returning to
      step 3.
\end{enumerate}


\begin{enumerate}
    \item[Process A.] (``/'' operation) Check page alignment with
      process C, then execute process D and call procedure $P_N$

    \item[Process B.] (``\mpow{}'' operation) If $p > 0$, execute
      process A.  Then execute process E and call procedure $P_{N'}$.

    \item[Process C.] (Page alignment)\\
      If $p'<L'-1$ and $p > 0$: execute process D, call procedure
      $P_N$ , and repeat process C.\\
      If $p'<L'-1$ and $p = 0$: execute process D until $p' = L'-1$.\\
      If $p'+1>R'$: execute process E, call procedure $P_{R'}$, and
      repeat process C.\\
      If $p'+1>P'$: execute process E, call procedure $P_{P'}$, and
      repeat process C.

    \item[Process D.] Skip the output medium to the next line, set
      $p=0$, and set $p'=p'+1$.

    \item[Process E.] Skip the output medium to the next page, and set
      $p'=0$.
\end{enumerate}


\subsubsection{Input}

The input process is initiated by the call:
\program{in list (unit, LAYOUT, LIST)}
\noindent{}The parameters have the same significance as they did in
the case of output, except that {\it unit} is in this case the number
of an input device.  There is a class of procedures {\it input n}
which stand for a call with a particularly simple type of layout and
list, just as discussed in Section \ref{Lbl++IoProc+IoCalls+Output}
for the case of output.  In the case of input, the parameters of the
``item'' procedure within the list must be variables.

The various steps which take place during the execution of {\it in
list} are very much the same as those in the case of {\it out list},
with obvious changes.  Instead of transferring characters of title
format, the characters are ignored on input.  If the data is improper,
some standard error procedure is used (cf. Section
\ref{Lbl++Formats+Number+Semantics+Input}).

The only significant change occurs in the case of standard input
format, in which the number $S$ of the above algorithm cannot be
determined in step 4.  The tests $p+S>R$ and $p+S>P$ now become a test
on whether positions $p+l$, $p+2$, \dots, $min(R, P)$ have any numbers
in them or not.  If so, the first number, up to its delimiter, is
used; the $R$ and $P$ positions serve as delimiters here.  If not,
however, overflow occurs, and subsequent lines are searched until a
number is found (possibly causing additional overflows). The right
boundary $min(R, P)$ will not count as a delimiter in the case of
overflow.  This rule has been made so that the process of input is
dual to that of output: an input item is not split across more than
one line unless it has overflowed twice.

Notice that the programmer has the ability to determine the presence
or absence of data on a card when using standard format, because of
the way overflow is defined.  The following program, for example, will
count the number $n$ of data items on a single input card and will
read them into $A[1]$, $A[2]$, \dots, $A[n]$. (Assume unit 5 is a card
reader.)

\program{\keyw{procedure} LAY; h end(EXIT, EXIT, EXIT);\\
\keyw{procedure} LIST(ITEM); ITEM(A[n + 1]);\\
\keyw{procedure} EXIT; \keyw{go to} L2;\\
\\
\hspace{1.0cm}\={}n := O;\\
L1:\>in list(5, LAY, LIST);\\
\>n := n - 1;\\
\>\keyw{go to} L1;\\
L2:\>\keyw{comment} mission accomplished;}


\subsubsection{Skipping}

Two procedures are available which achieve an effect similar to that
of the ``tab'' key on a typewriter:
\program{h skip (position, OVERFLOW)\\
v skip (position, OVERFLOW)}
\noindent{}where position is an integer variable called by value, and
$OVERFLOW$ is the name of a procedure.  These procedures are defined
only if they are called within a list procedure during an {\it in
list} or {\it out list} operation.  For {\it h skip}, if $p <
position$, set $p = position$; but if $p<position$, call the procedure
$OVERFLOW$. For {\it v skip}, an analogous procedure is carried out:
if $p'<position$, effectively execute process A of Section
\ref{Lbl++IoProc+IoCalls+Output} $(position-p')$ times; but if
$p'<position$, call the procedure $OVERFLOW$.

\subsubsection{Intermediate data storage}
The procedure call
\program{put (n, LIST)}
\noindent{}where $n$ is an integer parameter called by value and
$LIST$ is the name of a list procedure (Section
\ref{Lbl++IoProc+ListProc}), takes the values specified by the list
procedure and stores them, together with the identification number
$n$.  Anything previously stored with the same identification number
is lost.  The variables entering into the list do not lose their
values.

The procedure call
\program{get (n, LIST)}
\noindent{}where $n$ is an integer parameter called by value and $LIST$ is the
name of a list procedure, is used to retrieve the set of values which
has previously been put away with identification number $n$.  The
items in $LIST$ must be variables.  The stored values are retrieved in
the same order as they were placed, and they must be compatible with
the type of the elements specified by $LIST$; transfer functions may
be invoked to convert from real to integer type or vice versa.  If
fewer items are in $LIST$ than are associated with $n$, only the first
are retrieved; if $LIST$ contains more items, the situation is
undefined.  The values associated with $n$ in the external storage are
not changed by {\it get}.

\subsection{Control Procedures}
\label{Lbl++IoProc+ControlProc}

The procedure calls
\program{out control (unit, xl, x2, x3, x4)\\
in control (unit, xl, x2, x3, x4)}
\noindent{}may be used by the programmer to determine the values of
normally ``hidden'' system parameters, in order to have finer control
over input and output.  Here unit is the number of an output or input
device, and $xl$, $x2$, $x3$, $x4$ are variables.  The action of these
procedures is to set $xl$, $x2$, $x3$, $x4$ equal to the current
values of $p$, $P$, $p'$, $P'$, respectively, corresponding to the
device specified.

\subsection{Other Procedures}
\label{Lbl++IoProc+Other}

Other procedures which apply to specific input or output devices may
be defined at certain installations, e.g. {\it tape skip} and {\it
rewind} for controlling magnetic tape, etc.  An installation may also
define further descriptive procedures (thus introducing further hidden
variables); e.g., a procedure might be added to name a label to go to
in case of an input error.  Procedures for obtaining the current
values of hidden variables might also be incorporated.



\section{An Example}

A simple example follows, which is to print the first 20 lines of
Pascal's triangle in triangular form:

\begin{verbatim}
   1
  1 1
 1 2 1
1 3 3 1
\end{verbatim}

These first 20 lines involve numbers which are at most five digits in
magnitude.  The output is to begin a new page, and it is to be
double-spaced and preceded by the title ``PASCALS TRIANGLE''.  We
assume that unit number 3 is a line printer.

Two solutions of the problem are given, each of which uses slightly
different portions of the input-output conventions.

\program{\keyw{begin} \=\keyw{integer} N, K, printer;\\
\>\keyw{integer} \keyw{array} A [0:19];\\
\>\keyw{procedure} AK(ITEM); ITEM(A[K]);\\
\>\keyw{procedure} TRIANGLE; \keyw{begin} format('6Z');\\
\>\hspace{1.0cm}h lim(58 - 3 \mtim N, 63 + 3 \mtim N) \keyw{end};\\
printer := 3;\\
output 0 (printer, `\mpow{}`PASCALS\blankchar{}TRIANGLE'//');\\
\keyw{for} N := 0 \keyw{step} 1 \keyw{until} 19 \keyw{do}\\
\keyw{begin} \=A[N] := 1;\\
\>\keyw{for} K := N - 1 \keyw{step} -1 \keyw{until} 1 \keyw{do}\\
\>\hspace{1.0cm}A[K] := A[K - 1] + A[K];\\
\>\keyw{for} K := 0 \keyw{step} 1 \keyw{until} N \keyw{do}\\
\>\hspace{1.0cm}out list (printer, TRIANGLE, AK);\\
\>output 0 (printer, '//') \keyw{end} \keyw{end}}

\program{\keyw{begin} \=\keyw{integer} N, IK, printer;\\
\>\keyw{integer} \keyw{array} A [0:19];\\
\>\keyw{procedure} LINES; format 2 (`XB,X(6Z),//', 57 - 3 \mtim N, N+1);\\
\>\keyw{procedure} LIST(Q); \keyw{for} K := 0 \keyw{step} 1 \keyw{until} N \keyw{do} Q(A[K]);\\
printer := 8;\\
output 1 (printer, `\mpow{}20S//', `PASCALS\blankchar{}TRIANGLE');\\
\keyw{for} N := 0 \keyw{step} 1 \keyw{until} 19 \keyw{do}\\
\keyw{begin} \=A[N] := 1;\\
\>\keyw{for} K := N - 1 \keyw{step} -1 \keyw{until} 1 \keyw{do} A[K] := A[K - 1] + A[K];\\
\>out list (printer, LINES, LIST) \keyw{end} \keyw{end}}



\section{Machine-dependent Portions}

Since input-output processes must be machine-dependent to a certain
extent, the portions of this proposal which are machine-dependent are
summarized here.

\begin{enumerate}
\item The values of P and P' for the input and output devices.
\item The treatment of I, L, and R (unformatted) format.
\item The number of characters in standard output format.
\item The internal representation of alpha format.
\item The number of spaces, K, which will serveto delilnit standard
input format values.
\end{enumerate}


\end{document}
